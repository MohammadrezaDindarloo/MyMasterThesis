% !TeX root=../main.tex

\chapter{مقدمه}

این پایان‌نامه به بررسی موضوع کالیبراسیون و مکان‌یابی ربات‌ها می‌پردازد. در این فصل، ابتدا به بیان مسئله و مفاهیم مرتبط با این زمینه‌ها خواهیم پرداخت. شناخت دقیق این مفاهیم، نقشی کلیدی در حل مسائل و دستیابی به اهداف تحقیق ایفا می‌کند. هدف اصلی این تحقیق، نه تنها ایجاد ارتباطی معنادار بین این دو حوزه مهم رباتیکی است، بلکه فراهم‌سازی بستری مناسب برای گسترش این ارتباط به سایر موضوعات مرتبط نیز می‌باشد. اگر بتوانیم این ارتباط معنادار را برقرار کنیم، می‌توانیم این موضوعات را در یک نقطه مشترک به هم پیوند دهیم. این پیوند، تأثیر قابل توجهی بر نتایج هر یک از این عملیات‌ها خواهد داشت، که تاکنون کمتر به آن پرداخته شده است. در ادامه این فصل، به بررسی این تأثیرات و اهمیت آن‌ها خواهیم پرداخت. در نهایت، نگاهی به روش‌هایی که می‌توانند در ایجاد این ارتباط مؤثر باشند خواهیم داشت و به یک جمع‌بندی کلی از این مباحث دست خواهیم یافت.


\section{بیان مسئله}
کالیبراسیون در رباتیک فرآیندی است که به منظور بهبود دقت مکان‌یابی و حرکت ربات‌ها از طریق به‌روزرسانی پارامترهای سیستم رباتیکی انجام می‌شود. این فرآیند معمولاً با اصلاح نرم‌افزار کنترل ربات صورت می‌گیرد و نیاز به تغییرات مکانیکی در طراحی ربات ندارد. کالیبراسیون به شناسایی و تصحیح خطاهایی که ناشی از سایش قطعات، تحمل‌ها و سایر عدم قطعیت‌ها در ساختار ربات هستند، کمک می‌کند. هدف اصلی کالیبراسیون این است که رابطه بین خوانش‌های حسگرهای مفصل و مکان واقعی پتجه ربات را تا حد ممکن دقیق کند 
\cite{roth1987overview}.
روش‌های کالیبراسیون ربات را می‌توان به سه سطح اصلی تقسیم کرد که هر کدام بر جنبه‌های مختلف عملکرد ربات تمرکز دارند
\cite{elatta2004overview, roth1987overview}.

\begin{itemize}
	\item سطح اول، کالیبراسیون در سطح مفصل می‌باشد. این سطح اطمینان حاصل می‌کند که خوانش‌های حسگرهای مفصل به درستی بیانگر تغییرات واقعی مفصل‌ها هستند. مدل‌های ساده‌ای که معمولاً برای این سطح استفاده می‌شوند، رابطه بین سیگنال حسگر و حرکت مفصل را توضیح می‌دهند و اغلب نیاز به کالیبراسیون مجدد دارند، به ویژه پس از تعمیرات یا خاموش شدن ربات.
	
	
	\item  سطح دوم، کالیبراسیون سینماتیکی ربات است. کالیبراسیون سینماتیکی بر بهبود مدل سینماتیکی ربات تمرکز دارد که روابط مکانی بین مفصل‌ها و لینک‌های ربات را توصیف می‌کند. این روش با اطمینان از اینکه مدل استفاده‌شده در سیستم کنترل با ساختار فیزیکی واقعی ربات مطابقت دارد، دقت کلی ربات را بهبود می‌بخشد. 
	
	
	\item  سطح سوم، کالیبراسیون غیرسینماتیکی نام‌گذاری شده است. کالیبراسیون غیرسینماتیکی خطاهایی را که از عوامل غیرهندسی مانند انعطاف‌پذیری مفصل‌ها، اصطکاک و انعطاف‌پذیری لینک‌ها ناشی می‌شوند، مد نظر قرار می‌دهد. این سطح پیچیده‌تر است و به جنبه‌های دینامیکی عملکرد ربات می‌پردازد که برای وظایفی که نیاز به دقت بالا تحت شرایط عملیاتی متغیر دارند، بسیار مهم است.
	
	
\end{itemize}
تمامی این روش‌های بیان‌شده می‌توانند به یک مسئله‌‌ بهینه‌سازی با تابع‌هزینه‌های ریاضی مناسب در راستای کم شدن خطای کالیبراسیون تبدیل شوند. این مسئله‌های بهینه سازی در ربات‌های مختلف با توجه به سطح کالیبراسیون می توانند به مقادیر خاصی از حسگرها، شرایط هندسی مشخصی از ربات، قیدهای غیرهندسی، و یا ترکیبی از این قیدها، مقید شوند. 

علاوه بر کالیبراسیون، مکان‌یابی ربات‌ها یک مبحث حیاتی در حوزه رباتیک است که به تخمین موقعیت ربات نسبت به نقشه محیط می‌پردازد و برای انجام وظایف مختلف ربات ضروری است. حسگرهایی که برای مکان‌یابی ربات‌ها استفاده می‌شوند، با توجه به محیط، به دو دسته اصلی تقسیم می‌شوند: حسگرهای داخلی مانند ژیروسکوپ‌ها و شتاب‌سنج‌ها که حرکت ربات را بر اساس داده‌های داخلی تخمین می‌زنند، و حسگرهای خارجی مانند دوربین‌ها و لیزرهای مسافت‌یاب که از اطلاعات محیطی برای تعیین دقیق موقعیت ربات بهره می‌برند. این حسگرها می‌توانند به‌صورت مستقل یا ترکیبی استفاده شوند تا دقت مکان‌یابی را افزایش دهند
\cite{malagon2015mobile, burgard1997active}.
به علت خطاهای جمع‌شونده‌ای که در این فرآیند پراستفاده ممکن است ایجاد شود، برخی از تحقیقات اخیر اطلاعاتی به عنوان نشانگر
\footnote{landmark}
به سیستم مکان‌یابی اضافه می‌کنند
\cite{betke1997mobile}.
افزودن این اطلاعات منجر به مقید کردن فرآیند مکان‌یابی می‌شود. حسگر‌های بینایی، به عنوان یکی از مهم‌ترین و پرکاربردترین حسگر‌های اخیر در مکان‌یابی ربات‌ها مورد استفاده قرار گرفته‌اند، اگرچه قید‌هایی بر مقیاس تصاویر در این فرآیند وجود دارند که باید در یک مسئله‌ی بهینه‌سازی حل شود
\cite{Forster2014ICRA}.
 بدین ترتیب نگاه به مسئله مکان‌یابی نه تنها انعکاسی از یک مسئله بهینه‌سازی را در ذهن ما ایجاد می‌کند، بلکه در بسیاری از موارد این مسئله مقید خواهد بود. علاوه بر همه‌ی این موارد، نامعینی‌ حسگر‌ها و از کارافتادگی برخی در شرایط خاص، باعث پیش‌روی محققان این حوزه به سمت ترکیب این حسگر شده‌است. این کار مانند آنچه برای کالیبراسیون بیان شد، می‌تواند افزایش دقت را نتیجه دهد. 
 
حال با نگاهی به بیان تعریف شده از کالیبراسیون، مکان‌یابی و نقشه برداری و قید‌هایی که در هر یک از این وظایف بر ربات حاکم است، می توانیم همگی را در ساختار مسئله‌های بهینه‌سازی فرمول‌بندی کنیم. تعریف وظایف مختلف ربات به عنوان مسائل بهینه‌سازی، می تواند نقطه عطفی در بین این مسئله‌ها و ترکیب وظایف مختلف رباتیک ایجاد کند. در تحقیقات گذشته زمینه‌های جدیدی از این ترکیب‌ها معرفی شده است. به عنوان مثال الگوریتم مکان‌یابی و نقشه‌یابی ربات 
\footnote{\lr{Simultaneous Localization and Mapping (SLAM)}}
به صورت همزمان که به عنوان SLAM شناخته می‌شود. 

الگوریتم SLAM به دلیل نیاز همزمان به مکان‌یابی دقیق و ایجاد نقشه از محیط اطراف، یکی از مهم‌ترین و پیچیده‌ترین چالش‌های رباتیک محسوب می‌شود. در یک محیط ناشناخته، ربات باید بتواند با استفاده از حسگرهای خود، مکان فعلی‌اش را تخمین بزند (مکان‌یابی) و همزمان با این تخمین، نقشه‌ای از محیط پیرامون خود ایجاد کند. این دو وظیفه به شدت به هم وابسته‌اند؛ چرا که برای مکان‌یابی دقیق، نیاز به نقشه‌ای دقیق از محیط است و برای ایجاد یک نقشه دقیق، نیاز به مکان‌یابی دقیق ربات است. عدم ترکیب صحیح این دو فرایند می‌تواند منجر به خطاهای تجمعی در هر دو بخش شود، که در نهایت به کاهش دقت و کارایی ربات در مسیریابی و انجام وظایفش منجر خواهد شد. بنابراین، ترکیب مکان‌یابی و نقشه‌برداری در یک چارچوب SLAM باعث می‌شود که ربات بتواند به طور همزمان و با دقت بالا، هم مکان خود را در محیط تعیین کند و هم نقشه‌ای از محیط بسازد، که این امر برای کاربردهای مختلف رباتیک، از جمله ناوبری خودکار و جستجو و نجات، حیاتی است
\cite{grisetti2010tutorial}.

کالیبراسیون همزمان حسگرها و مکان‌یابی ربات‌ها یک فرآیند حیاتی برای افزایش دقت و بهبود کارایی سیستم‌های رباتیک است. در طی ناوبری، ربات‌ها برای اصلاح خطاهای ناشی از مسافت‌پیمایی و بهبود دقت مکان‌یابی، نیازمند ترکیب داده‌های حسگرها و تخمین دقیق موقعیت خود هستند.  توجه به اینکه خطاهای قابل پیش‌بینی و غیرقابل پیش‌بینی در طول زمان و با پیمایش مسافت‌های طولانی افزایش می‌یابند، کالیبراسیون مداوم برای کاهش این خطاها و اطمینان از صحت محاسبات مکان، ضروری است.
 به عنوان مثال، استفاده از فیلتر کالمن افزوده 
\footnote{\lr{Augmented Kalman Filter (AKF)}}
برای تخمین پارامترهای خطای سیستماتیک در کنار به‌روزرسانی مکان ربات در طول حرکت، به حفظ دقت مکان‌یابی کمک شایانی می‌کند. این رویکرد نه تنها به کاهش خطاهای مسافت‌پیمایی کمک می‌کند، بلکه امکان اصلاح مداوم مدل‌های خطای حسگرها را نیز فراهم می‌سازد. این امر به خصوص در محیط‌های داخلی که دقت مکان‌یابی بسیار اهمیت دارد، کاربردی و ضروری است
\cite{martinelli2007simultaneous}.

ترکیب‌های متفاوت در زمینه کالیبراسیون و مکان‌یابی ربات‌ها دارای مزایای متعددی از جمله بهبود دقت و نتایج نهایی می‌باشند. در این پایان‌نامه، همچون روش‌های بیان شده پیشین، گامی مهم در راستای ترکیب این وظایف اساسی ربات‌ها برداشته‌ شده‌است. این حرکت از همان نقطه عطفی آغاز می‌شود که در روش‌های قبلی با نگاه به وظایف ربات به عنوان مسئله‌ای بهینه‌سازی مطرح شده بود، و ما بر آنیم تا با گسترش این رویکرد، به ترکیبی جامع‌تر و کارآمدتر از این وظایف دست یابیم. آنچه کار ما را نسبت به موارد گذشته متمایز می‌کند، جامعیت بخشیدن به فرمول‌بندی ارائه شده به گونه‌ای است که نه‌تنها کالیبراسیون حسگرها، بلکه کالیبراسیون ربات در سطوح مختلف به‌طور همزمان با مکان‌یابی انجام شود. همچنین، قیود متفاوت و متعددی که در این مسائل وجود دارد، ما را از روش‌های مرسوم و فیلترمبنا دور می‌کند. در واقع، روش‌های اخیر برای ترکیب حسگرها با توجه به پیچیدگی‌های موجود، بیشتر از روش‌های فیلترمبنا فاصله گرفته‌اند. هدف ما نیز پیدا کردن روشی سریع و منعطف است که نه تنها توانایی حل بالای مسائل را داشته باشد، بلکه با افزودن قیود مختلف به سیستم، فرمول‌بندی سایر اجزا را تغییر ندهد. چنین فرمول‌بندی‌ای باعث می‌شود که روش ارائه شده به صورت ماژولار بتواند به دیگر مسائل این حوزه تعمیم داده‌شود. 

برای انجام هر یک از سطوح کالیبراسیون مطرح شده، روش‌های خاصی نیز ارائه شده است. یکی از این روش‌ها برای بهبود کالیبراسیون، افزایش مدل است که در آن پدیده‌های پیشتر مدل‌نشده، مانند ساختار قرقره و مکانیزم‌های پولی، به مدل ربات افزوده می‌شوند. این روش دقت را افزایش می‌دهد، اما ممکن است به دلیل نیاز به حسگرهای ویژه و فرضیات خاصی که در مدل‌سازی به کار می‌رود، محدودیت‌هایی داشته باشد. رویکرد دیگر شامل استفاده از انواع حسگرها برای کالیبراسیون است. برای مثال، از ابزارهای اندازه‌گیری طول یک‌بعدی و نیرو‌سنج‌ها برای جمع‌آوری داده‌ها به منظور کالیبراسیون سینماتیکی استفاده شده است. این روش‌ها می‌توانند نیاز به اندازه‌گیری‌های دقیق مکان پنجه را کاهش دهند، اما ممکن است توسط محدودیت‌های حسگرها به‌ویژه در سناریوهای واقعی محدود شوند
\cite{elatta2004overview, roth1987overview}. 

کالیبراسیون همچنان یکی از اجزای اساسی در رباتیک به شمار می‌آید که تضمین می‌کند ربات‌ها با دقت و اطمینان بالا عمل کنند. با وجود روش‌های گوناگونی که هر یک مزایای خاص خود را دارند، انتخاب روش کالیبراسیون معمولاً به کاربرد خاص و محدودیت‌های عملیاتی ربات بستگی دارد. تمرکز بر ترکیب این روش‌ها می‌تواند به ایجاد فرآیندهای کالیبراسیون قوی‌تر و سازگارتر منجر شود.

روش‌های مکان‌یابی به‌طور کلی به دو دسته تقسیم می‌شوند: اول، روش‌های مبتنی بر مدل‌سازی احتمالی و دوم، روش‌های مبتنی بر ترکیب اطلاعات حسگرها. روش‌های مبتنی بر مدل‌های احتمالی، توزیع احتمال مکان ربات را در فضای حالت محیط حفظ کرده و به‌صورت بازگشتی با استفاده از فیلترهای مختلف این توزیع را به‌روز می‌کنند. در مقابل، روش‌های مبتنی بر ترکیب اطلاعات حسگرها، از ترکیب داده‌های دریافتی از حسگرهای مختلف برای افزایش دقت تخمین مکان بهره می‌برند.

یکی از روش‌های اصلی مکان‌یابی، استفاده از فیلترهای احتمالی مانند فیلتر کالمن و فیلتر ذره‌ای است. این روش‌ها با بهره‌گیری از توزیع احتمالی، مکان ربات را تخمین می‌زنند. فیلتر کالمن به‌طور گسترده‌ای برای ترکیب اطلاعات حسگرها به کار می‌رود و به‌ویژه در محیط‌هایی که نویز سیستم گوسی است، کارایی بالایی دارد. از سوی دیگر، فیلتر ذره‌ای که یکی از روش‌های غیرپارامتریک است، توانایی تخمین مکان ربات در محیط‌های با نویز غیرگوسی را دارد. این فیلتر با نمونه‌برداری از فضای حالت، توزیع احتمالی مکان ربات را تخمین می‌زند.

یکی از روش‌های اصلی مکان‌یابی، استفاده از فیلترهای احتمالی مانند فیلتر کالمن و فیلتر ذره‌ای است. این روش‌ها با بهره‌گیری از توزیع احتمالی، مکان ربات را تخمین می‌زنند. فیلتر کالمن به‌طور گسترده‌ای برای ترکیب اطلاعات حسگرها به کار می‌رود و به‌ویژه در محیط‌هایی که نویز سیستم گوسی است، کارایی بالایی دارد. از سوی دیگر، فیلتر ذره‌ای که یکی از روش‌های غیرپارامتریک است، توانایی تخمین مکان ربات در محیط‌های با نویز غیرگوسی را دارد. این فیلتر با نمونه‌برداری از فضای حالت، توزیع احتمالی مکان ربات را تخمین می‌زند. علاوه بر این، فیلتر کالمن خنثی
\footnote{\lr{Unscented Kalman Filter (UKF)}}
، که قادر به تخمین حالات غیرخطی و سیستم‌های پیچیده است، با استفاده از مجموعه‌ای از نقاط سیگما، دقت بالاتری در مدل‌سازی سیستم‌های غیرخطی ارائه می‌دهد. همچنین فیلتر بیزین
\footnote{\lr{Bayesian Filter}}
 غیرخطی به عنوان یک روش پیشرفته‌تر، با ترکیب اطلاعات حسگرها به صورت غیرخطی و محاسبه توزیع بیزین در فضای حالت، مکان‌یابی دقیق‌تری را در محیط‌های نامطمئن و پیچیده فراهم می‌آورد.

روش پیشنهادی ما برای ایجاد یک مسئله یکپارچه که ویژگی‌های بیان شده را فعال کند، استفاده از گراف است. استفاده از این روش‌ها می‌تواند قیود و فرمول‌بندی‌های موجود در کالیبراسیون و مکان‌یابی را به صورت یکپارچه و با استفاده از توابع احتمالاتی در کنار یکدیگر قرار دهد. با ادغام تمامی این تابع‌های هزینه، می‌توانیم به یک راه‌حل یکپارچه برای مسئله مورد نظر دست یابیم. این گونه نمایش‌ مسئله ما را به مسیری در حوزه رباتیک هدایت می‌کند که اخیراً محققان ارزش زیادی برای آن قائل شده‌اند. این رویکرد با نام ”گراف عامل“ در این حوزه شناخته می‌شود و به عنوان یک روش کارآمد برای حل مسائل پیچیده و چندوجهی در رباتیک مطرح شده است. 

\section{ربات انتخابی برای بررسی نتایج تحقیق}

در این تحقیق، ربات مورد استفاده جهت ارزیابی و بررسی نتایج الگوریتم گراف‌مبنا، یک ربات کابلی است. این نوع ربات‌ها به دلیل ساختار ویژه‌شان، از جمله انعطاف‌پذیری کابل‌ها و تأثیر نیروهای خارجی بر عملکردشان، نیازمند تحلیل‌های دقیق در شرایط مختلف هستند. به همین دلیل، تحقیقات انجام شده برای کالیبراسیون این نوع ربات‌ها بسیار محدود بوده است. از آنجایی که ادعای روش پیشنهادی حل مسائل کالیبراسیون و مکان‌یابی همزمان با سرعت بالا برای طیف وسیعی از ربات‌ها است، ربات‌های کابلی به‌عنوان گزینه‌ای مناسب برای محک زدن این روش انتخاب شده‌اند.

در این تحقیق، الگوریتم گراف‌مبنا برای کالیبراسیون و مکان‌یابی همزمان ربات در دو حالت مختلف به کار گرفته شده است. در حالت اول، با فرض صلب بودن کابل‌ها، اثر شکم‌دهی کابل‌ها نادیده گرفته می‌شود. این رویکرد باعث ساده‌تر شدن فرمول‌بندی مسئله و کاهش پیچیدگی‌های محاسباتی می‌گردد.

در حالت دوم، شکم‌دهی کابل‌ها به‌عنوان بخشی از مدل در نظر گرفته شده است. این تحلیل پیچیده‌تر است و از قابلیت‌های الگوریتم گراف‌مبنا برای مدیریت داده‌ها و قیود غیرخطی بهره می‌برد. با استفاده از این رویکرد، کالیبراسیون و مکان‌یابی ربات کابلی در شرایط واقعی‌تر مورد ارزیابی قرار گرفته و نتایج به‌دست‌آمده می‌توانند به بهبود دقت و کارایی ربات در سناریوهای عملی منجر شوند. یکی دیگر از اهداف انتخاب این نوع ربات‌ها، بررسی میزان انعطاف‌پذیری فرمول‌بندی مسئله است، به‌گونه‌ای که فرمول‌بندی حالت دوم با افزودن قیود دینامیکی به فرمول‌بندی حالت اول ایجاد می‌شود.

این دو حالت، که به کمک گراف‌های عامل مدل‌سازی و تحلیل شده‌اند، به ما امکان می‌دهند تا تأثیرات شکم‌دهی کابل‌ها را بر دقت کالیبراسیون و مکان‌یابی ربات بررسی کرده و راه‌حل‌هایی برای کاهش خطاهای ناشی از این پدیده ارائه دهیم. نتایج این تحلیل‌ها به‌طور مفصل در فصل نتایج مورد بحث قرار خواهند گرفت.


\section{اهمیت تحقیق}

کالیبراسیون و مکان‌یابی دقیق ربات‌های کابلی از اهمیت بالایی برخوردار است، زیرا این دو فرآیند تأثیر مستقیمی بر عملکرد و دقت این نوع ربات‌ها دارند. ربات‌های مختلف با توجه به کاربردهای خاص خود، در بسیاری از صنایع کاربرد گسترده‌ای دارند. با این حال، هرگونه عدم دقت در کالیبراسیون و مکان‌یابی می‌تواند منجر به بروز خطاهای جدی در عملکرد ربات‌ها شود.

این تحقیق با هدف ارائه یک رویکرد نوین برای کالیبراسیون و مکان‌یابی همزمان در حضور قیدهای رباتیکی همچون سینماتیکی، دینامیکی، فیزیکی و کنترلی، از اهمیت بالایی برخوردار است. یکی از نقاط قوت این تحقیق، استفاده از گراف‌های عامل برای مدل‌سازی و حل مسئله است. گراف‌های عامل به دلیل توانایی‌شان در مدل‌سازی مسائل پیچیده و چندبعدی، ابزار مناسبی برای حل مسائل کالیبراسیون و مکان‌یابی ربات‌های کابلی هستند. این رویکرد به ما این امکان را می‌دهد که پیچیدگی‌های دینامیکی و سینماتیکی ربات‌ها را در نظر بگیریم و دقت بیشتری در کالیبراسیون و مکان‌یابی به دست آوریم. همچنین، ایجاد یک روش فرمول‌بندی مبنا برای تعریف این مسئله، یکی از ویژگی‌های مهم این تحقیق است. 


\section{اهداف تحقیق}
این تحقیق به دنبال ارائه یک رویکرد نوین برای کالیبراسیون و مکان‌یابی همزمان ربات‌ها است که بتواند دقت و کارایی لازم را در این فرآیندها فراهم کند. اهداف اصلی این تحقیق به شرح زیر هستند:

\begin{itemize}
	\item \textbf{توسعه رویکرد گراف‌مبنا برای کالیبراسیون و مکان‌یابی همزمان:} یکی از اهداف اصلی این تحقیق، ارائه یک رویکرد گراف‌مبنا است که بتواند فرآیندهای کالیبراسیون و مکان‌یابی را به‌صورت همزمان و با دقت بالا انجام دهد. این رویکرد با استفاده از گراف‌های عامل، به مدل‌سازی دقیق روابط بین پارامترهای مختلف ربات و بهینه‌سازی این پارامترها می‌پردازد.
	
	\item \textbf{بررسی اثرات ادغام کالیبراسیون و مکان‌یابی:} هدف دیگر این تحقیق، بررسی اثرات ادغام فرآیندهای کالیبراسیون و مکان‌یابی بر دقت و کارایی ربات است. این بررسی شامل تحلیل نتایج حاصل از ادغام این دو فرآیند و مقایسه آن با روش‌های مرسوم است.
	
	\item \textbf{ارزیابی روش پیشنهادی در پیاده سازی بر روی ربات‌های کابلی واقعی:} هدف سوم این تحقیق، ارزیابی امکان استفاده از روش پیشنهادی در شرایط واقعی بر  روی ربات‌های کابلی است. با توجه به نتایج حاصل از پیاده‌سازی عملی، این تحقیق به بررسی قابلیت استفاده از این روش در محیط‌های واقعی پرداخته و مزایا و چالش‌های احتمالی آن را تحلیل می‌کند.
	
	\item \textbf{ارائه فرمول‌بندی جامع  گراف‌مبنا جهت کالیبراسیون و مکان‌یابی ربات‌های کابلی با/بدون در نظر گرفتن شکم‌دهی کابل:} هدف چهارم این تحقیق، تکمیل و ارائه نتایج الگوریتم مد نظر بر روی ربات‌های کابلی به صورت منبع باز برای استفاده‌های گوناگون می ‌باشد. روش ارائه شده در انتها شامل انجام این فرآیند برای ربات‌ها با در نظر گرفتن و همچنین بدون در نظر گرفتن جرم کابل و اثر شکم‌دهی کابل در ریاضیات مسئله می‌باشد.
	
\end{itemize}


\section{روش تحقیق}
 روش تحقیق این پایان‌نامه شامل چندین مرحله اساسی است که هر یک به طور مستقل و دقیق بررسی شده‌اند. 
\begin{itemize}
	
	\item \textbf{بررسی جامع ادبیات و تحلیل مشکلات موجود:} در گام اول، به منظور شناسایی مشکلات و چالش‌های موجود در زمینه کالیبراسیون و مکان‌یابی ربات‌ها، مرور جامعی بر روی ادبیات موضوع انجام شده است. این مرور شامل بررسی روش‌های سنتی و نوین کالیبراسیون و مکان‌یابی و تحلیل نقاط قوت و ضعف هر یک از این روش‌ها است. همچنین، از آخرین دستاوردهای علمی در حوزه رباتیک برای بهبود روش‌های پیشنهادی بهره گرفته شده است. از آنجایی که ربات انتخابی ما برای پیاده‌سازی یک ربات کابلی است، قسمتی از این فصل به بررسی این موضوعات در زمینه ربات‌های کابلی خواهد بود.
	
	\item \textbf{توسعه مدل ریاضی و فرمول‌بندی مسئله:}
در گام دوم، یک مدل ریاضی دقیق برای کالیبراسیون و مکان‌یابی همزمان ربات‌ها توسعه داده شده است. این مدل با استفاده از گراف‌های عامل و داده‌های سینماتیکی و بینایی، به بهینه‌سازی پارامترهای مختلف ربات پرداخته و یک فرمول‌یندی از ریاضیات ربات ارائه می‌دهد. در این فرمول‌بندی، قیدهای دینامیکی و سینماتیکی به‌صورت جامع در نظر گرفته شده‌اند. توسعه این مدل‌های ریاضی، برای مورد مطالعه، بر روی ربات‌های کابلی به‌صورت کامل مورد بررسی قرار گرفته است.
	
	\item \textbf{پیاده‌سازی عملی و ارزیابی نتایج:} در گام سوم، پیاده‌سازی عملی روش پیشنهادی بر روی یک ربات کابلی واقعی انجام شده است. این پیاده‌سازی با استفاده از داده‌های جمع‌آوری شده از ربات و با بهره‌گیری از الگوریتم‌های کالیبراسیون و مکان‌یابی انجام شده است. نتایج حاصل از این پیاده‌سازی مورد ارزیابی و تحلیل قرار گرفته‌اند تا دقت و کارایی روش پیشنهادی مورد بررسی قرار گیرد.
	
	\item \textbf{تحلیل آماری و ارائه پیشنهادات برای بهبود روش‌ها:} در گام آخر، نتایج حاصل از پیاده‌سازی و ارزیابی‌ها به‌صورت آماری تحلیل شده‌اند. این تحلیل‌ها شامل بررسی اثرات ادغام فرآیندهای کالیبراسیون و مکان‌یابی و مقایسه آن با روش‌های سنتی است. همچنین، بر اساس نتایج به‌دست‌آمده، پیشنهاداتی برای بهبود و توسعه روش‌های موجود در زمینه کالیبراسیون و مکان‌یابی ارائه شده است.
	
\end{itemize}


\section{دستاورد‌های تحقیق}
در این تحقیق، فرمول‌بندی یکپارچه‌ای در فضای گراف ارائه شده است که به طور همزمان به حل مسائل کالیبراسیون و مکان‌یابی ربات‌ها به صورت هم‌زمان می‌پردازد. برای بررسی کارایی این فرمول‌بندی، با استفاده از گراف‌های عامل، برای اولین بار مسئله پیچیده کالیبراسیون و مکان‌یابی را برای ربات‌های کابلی در دو حالت مختلف بررسی شده است: حالت اول، با فرض صلب بودن کابل‌ها و نادیده‌گرفتن شکم‌دهی، و حالت دوم، با در نظر گرفتن شکم‌دهی کابل‌ها به عنوان بخشی از مدل. این رویکرد نوآورانه امکان تحلیل دقیق‌تر و واقعی‌تر از عملکرد ربات‌های کابلی را فراهم کرده است و نشان داده که الگوریتم پیشنهادی می‌تواند به طور مؤثری در هر دو حالت به کار گرفته شود، و دقت و کارایی بالاتری در کالیبراسیون و مکان‌یابی ربات‌ها ارائه دهد.

\section{معرفی سایر فصل‌ها}
در این پایان‌نامه، پس از بررسی کلیات و اهداف تحقیق در فصل اول، به مرور ادبیات و مبانی نظری در فصل دوم پرداخته‌ شده‌ است. در این فصل، مسائل بهینه‌سازی در حوزه رباتیک و کاربردهای مختلف آن‌ها مورد بررسی قرار گرفته و به تحلیل روش‌های مختلف برای کالیبراسیون، مکان‌یابی، و حل مسائل پیچیده در ربات‌ها پرداخته شده است.

فصل سوم به معرفی یک رویکرد نوین بر مبنای گراف‌های عاملی برای حل مسائل کالیبراسیون و مکان‌یابی همزمان ربات‌ها اختصاص یافته است. در این فصل، با بررسی مشکلات و محدودیت‌های روش‌های مرسوم، رویکردی جدید ارائه شده که ضمن افزایش دقت و کارایی، انعطاف‌پذیری بیشتری برای اعمال قیود مختلف دارد. این فصل با معرفی گراف عامل پیشنهادی که به‌طور جامع مسائل مکان‌یابی و کالیبراسیون را حل می‌کند، به پایان می‌رسد.

در فصل چهارم، به پیاده‌سازی و ارزیابی عملی روش پیشنهادی در یک ربات کابلی با فرض کابل‌های صلب پرداخته شده است. این پیاده‌سازی شامل بررسی تأثیر قیود سینماتیکی بر دقت مکان‌یابی و کالیبراسیون خودکار ربات بوده و نتایج حاصل از این پیاده‌سازی نشان‌دهنده عملکرد مطلوب روش پیشنهادی از نظر سرعت و دقت است.

فصل پنجم به توسعه و ارتقای روش مطرح‌شده در فصل چهارم اختصاص دارد. در این فصل، پیچیدگی‌های بیشتری به مدل افزوده شده و قیدهای دینامیکی کابل‌های شکم‌دار در نظر گرفته شده است. نتایج حاصل از شبیه‌سازی‌ها در این فصل نشان می‌دهد که مدل‌سازی دقیق‌تر کابل‌ها، بهبود قابل‌توجهی در دقت کالیبراسیون و مکان‌یابی به همراه دارد.

در نهایت، فصل ششم با ارائه نتایج کلی و جمع‌بندی مطالب پایان‌نامه به پایان می‌رسد. در این فصل همچنین پیشنهاداتی برای ادامه تحقیقات در این حوزه ارائه شده است. این پیشنهادات به بررسی زمینه‌های مختلفی پرداخته‌اند که می‌توانند بهبود‌های بیشتری در الگوریتم‌ها و کاربردهای رباتیک به ارمغان آورند.




