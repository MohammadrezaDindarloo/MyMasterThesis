% !TeX root=../main.tex
\chapter{مروری بر پژوهش‌ها}
این فصل، ابتدا مروری بر آخرین تحقیقات انجام‌شده در زمینه کالیبراسیون و روش‌های پیشنهادی آن‌ها خواهد داشت. در انتهای این بخش، تمرکز بیشتری بر روی ربات‌هایی خواهد بود که در این پایان‌نامه مورد تحقیق قرار گرفته‌اند. پس از مرور مباحث کالیبراسیون، به بررسی روش‌های مختلف مکان‌یابی پرداخته و نگاه دقیق‌تری به ربات‌های کابلی در این زمینه خواهیم داشت. در نهایت، مطالعه‌ای بر تحقیقات ارائه شده که به ترکیب وظایف رباتیکی پرداخته‌اند، انجام خواهد شد.

\section{مروری بر پژوهش‌های انجام شده در حوزه کالیبراسیون} 



همانطور که بیان شد، کالیبراسیون نقشی کلیدی در بهبود دقت و کارایی ربات‌ها ایفا می‌کند. کالیبراسیون ربات‌ها با هدف بهبود دقت مکان‌یابی آن‌ها، از اهمیت ویژه‌ای برخوردار است. روش‌های مختلفی برای کالیبراسیون پارامترهای ربات‌ها در زمینه‌های مختلف ارائه شده است که هر یک دارای مزایا و معایب خاص خود هستند. علاوه بر روش ارائه شده در فصل قبل در دسته‌بندی سطوح کالیبراسیون، دیدگاهی دیگر در این زمینه وجود دارد که برای این مرور ادبیات برای ما مفید خواهد بود.

در این دیدگاه، اگر کالیبراسیون صرفاً از حسگرهای داخلی ربات استفاده کند، به آن خودکالیبراسیون\footnote{self-calibration} گفته می‌شود. و اگر از حسگر‌های خارجی اضافی برای کالیبراسیون استفاده شود، به آن کالیبراسیون خارجی\footnote{external-calibration} می‌گویند. در حالی که کالیبراسیون خارجی دقت بیشتری را به همراه دارد، اما از نظر هزینه و زمان‌بری، عملکرد ضعیف‌تری نسبت به خودکالیبراسیون دارد. در ادامه نگاهی به کالیبراسیون در ربات‌های مختلف خواهیم داشت.


در ربات‌های صنعتی، روش‌های مبتنی بر بهینه‌سازی فیلتر برای بهبود دقت پارامترهای سینماتیکی به‌طور گسترده‌ای مورد استفاده قرار گرفته‌اند. به‌عنوان مثال، \cite{gan2019calibration} روشی با استفاده از حسگرهای کششی ارائه کرده است. \cite{park2011laser} از حسگر لیزری و یک دوربین ثابت بهره گرفته و اعتبار روش خود را از طریق آزمایش بر روی بازوی رباتیک تأیید کرده‌اند. همچنین، روش‌هایی مانند \cite{li2011optimal} بر بهینه‌سازی تعداد حالت‌های اندازه‌گیری جهت بهبود شناسایی پارامترها تمرکز داشته‌اند. این دسته از روش‌های کالیبراسیون عمدتاً مبتنی بر فیلترها هستند و نیاز به استفاده از سنسورهای خارجی دارند، که در دسته‌ی کالیبراسیون خارجی قرار می‌گیرند.

دسته دیگری از ربات‌ها که اخیراً تحقیقات گسترده‌تری بر روی آنها صورت گرفته است، ربات‌های پادار هستند. کالیبراسیون در این ربات‌ها نیز به‌ویژه برای بهبود دقت حرکتی اهمیت فراوانی دارد. \cite{yang2022online} روشی برای کالیبراسیون آنلاین پارامترهای سینماتیکی مبتنی بر فیلتر کالمن ارائه کرده است. یکی از مواردی که در این نوع ربات‌ها بسیار مورد توجه قرار گرفته، استفاده همزمان از داده‌های حسگرهای داخلی و خارجی برای بهبود دقت تخمین پارامترهای حرکتی است \cite{blochliger2017foot}. ترکیب داده‌های مختلف در این نوع ربات‌ها و ربات‌های خودران اخیراً موضوع داغ‌تری بوده است و تلاش‌های زیادی برای حل چالش‌های مرتبط با آن انجام شده است. الگوریتم‌هایی مانند SLAM تأثیر قابل‌توجهی بر ترکیب داده‌ها داشته‌اند، اما موضوع کالیبراسیون به اندازه کافی مورد توجه قرار نگرفته است.

علاوه بر موارد فوق، کالیبراسیون در ربات‌های جراح به دلیل نیاز به دقت بالا امری حیاتی است. روش‌هایی نظیر \cite{wang2017vision} بر کالیبراسیون ابزارهای جراحی با استفاده از تصویربرداری دوبعدی و سه‌بعدی تمرکز کرده‌اند. در این تحقیقات، کالیبراسیون با استفاده از سنسورهای خارجی انجام شده و یک فرآیند جداگانه پیش از هر عمل جراحی برای اطمینان از دقت کافی ربات ضروری بوده است. \cite{roberti2020improving} نیز روشی برای کالیبراسیون با استفاده از دوربین RGB-D و ربات‌های جراحی da Vinci ارائه کرده است که دقت بیشتری نسبت به روش‌های مرسوم دارد. این روش کالیبراسیون شامل سه مرحله اصلی و مجزا است: کالیبراسیون بازوهای رباتیک، کالیبراسیون دوربین، و کالیبراسیون دست-چشم\footnote{\lr{Hand-Eye-Calibration}}.
نکته‌ای که در اینجا از اهمیت بالایی برخوردار است، مجزا بودن این مراحل است. این جداسازی نه تنها نیازمند تلاش‌های جداگانه برای کالیبراسیون سنسورها و ربات است، بلکه تأثیر هر یک بر دیگری مانع از دستیابی به پاسخ بهینه می‌شود. به‌عنوان مثال، برای انجام کالیبراسیون ربات، ابتدا باید کالیبراسیون دست-چشم به‌درستی انجام شده باشد؛ در حالی که خود کالیبراسیون دست-چشم بر اساس روش‌های سینماتیکی صورت می‌گیرد که پیش از کالیبراسیون ربات مقادیر بهینه‌ای ندارند. شاید پیوند دادن این دو کالیبراسیون به یکدیگر و حتی کالیبراسیون حسگرها و تبدیل آن‌ها به یک مسئله واحد، ما را به پاسخ بهینه نزدیک کند و از انجام فرآیندهای جداگانه جلوگیری نماید. اما تا کنون این موضوع در رباتیک به‌طور دقیق مورد بررسی قرار نگرفته است. یکی از دلایل ممکن برای نپرداختن به این مسئله، احتمالاً سختی فرمول‌بندی و پیچیدگی‌های محاسباتی بالا در روش‌های مرسوم بوده است. 


در ادامه مروری بر پژوهش های انجام شده در کالیبراسیون برای ربات‌های کابلی موازی\footnote{\lr{Cable-Driven Parallel Robot (CDPR)}}
خواهیم داشت. این ربات‌ها نیز از اهمیت ویژه‌ای برخوردار هستند زیرا با استفاده از کابل‌های سبک و اقتصادی به انتقال انرژی و کنترل اجسام متحرک می‌پردازند. با این حال، یکی از چالش‌های اساسی در استفاده از کابل‌ها به عنوان رابط انتقال، پیچیدگی‌های ناشی از افتادگی کابل‌ها تحت تأثیر وزن آن‌ها است. این پدیده، کالیبراسیون دقیق این ربات‌ها را دشوار می‌سازد و مانع از کاربرد گسترده آن‌ها می‌شود. به همین دلیل در بسیاری از مطالعات انجام‌شده در کالیبراسیون این ربات‌ها، کابل‌ها را بدون وزرن به عنوان اجسام صلب در نظر گرفته‌اند. از آنجا که ربات‌های کابلی در بسیاری از کاربردها از جمله ماموریت‌های نجات، واقعیت مجازی، و عملیات در محیط‌های خطرناک به کار گرفته می‌شوند، داشتن کالیبراسیونی دقیق برای اطمینان از دقت و کارایی این ربات‌ها ضروری است. انتخاب استراتژی مناسب برای کالیبراسیون، چه با استفاده از حسگرهای داخلی ربات (خودکالیبراسیون) و چه با استفاده از ابزارهای مرجع خارجی، نقش کلیدی در بهبود عملکرد و دقت این سیستم‌ها ایفا می‌کند. 
 
نویسندگان در \cite{jin2018geometric} یک ابزار اندازه‌گیری طول تک‌بعدی را برای کالیبراسیون معرفی کرده‌اند. با این حال، محدودیت‌های عملکردی این حسگر باعث کاهش قابلیت کاربرد این روش در سناریوهای دنیای واقعی شده است. این روش یک کالیبراسیون خارجی را معرفی می کند. همانطور که بیان گردید، کالیبراسیون خودکار برای ربات‌های کابلی بزرگ‌مقیاس که با مشکل افتادگی کابل مواجه هستند، به‌طور مستقیم کمتر مورد بررسی قرار گرفته است. 

در \cite{borgstrom2009nims}، نویسندگان یک روش خودکالیبراسیون برای تخمین مکان اولیه یک ربات با چهار کابل و دو درجه آزادی (DoFs) معرفی کرده‌اند. به همین ترتیب، در \cite{ida2019automatic}، مکان اولیه یک ربات کابلی معلق و فروتحریک با استفاده از معادلات ایستا تخمین زده می‌شود. علاوه بر این، در \cite{pott2013cable_ForceSensorCalib} یک روش خودکالیبراسیون با استفاده از حسگرهای نیرو توسعه داده شده است، در حالی که نویسندگان \cite{darvin2018initial} روشی را برای کالیبراسیون مکان اولیه ربات و مقدار طول اولیه با استفاده از انکودرها پیشنهاد کرده‌اند. 
 
تنها تحقیقی که تاکنون برای خودکالیبراسیون ربات‌های کابلی با در نظر گرفتن اثر شکم‌دهی کابل‌ها منتشر شده است، توسط \cite{an2022all} ارائه شده است. روش بیان‌شده در این مقاله برای حل مسئله کالیبراسیون، از روش‌های مرسوم استفاده می‌کند. با این حال، در این تحقیق، پاسخ‌پذیری معادلات در قالب فرمول‌بندی ارائه شده برای ربات‌های با ابعاد بزرگ که در آن‌ها اثر شکم‌دهی نمود پیدا می‌کند، بررسی نشده است.



\section{مروری بر پژوهش‌های انجام شده در حوزه مکان‌یابی} 

مکان‌یابی یکی از مسائل اساسی در رباتیک است که هدف آن تعیین مکان یک ربات در محیط مشخصی است. در طول سال‌ها، روش‌های مختلفی توسعه یافته‌اند که هر کدام دارای نقاط قوت و ضعف خاص خود هستند. این بخش به مرور روش‌های کلیدی مبتنی بر روش‌های احتمالی، ترکیب حسگرها، و استراتژی‌های مکان‌یابی می‌پردازد. در ادامه مروری بر هر یک از این روش‌ها و برخی از تحقیقاتی که در این زمینه انجام شده‌اند خواهیم‌داشت.


\subsection{روش‌های مکان‌یابی احتمالی}

روش‌های احتمالی به دلیل مقاومت در برابر عدم قطعیت\footnote{uncertainty}
 و نویز در اندازه‌گیری حسگرها به طور گسترده‌ای در مکان‌یابی ربات‌های متحرک استفاده می‌شوند. این روش‌ها اغلب شامل استفاده از فیلترهای بیزین\footnote{\lr{Bayesian-Filter}}
 ، از جمله فیلتر کالمن\footnote{\lr{Kalman Filter(KF)}}
، فیلتر کالمن توسعه‌یافته\footnote{\lr{Extended Kalman Filter (EKF)}}
، و فیلترهای ذره‌ای\footnote{\lr{Particle-Filter}}
 هستند.


\textbf {فیلتر کالمن و نسخه‌های آن}

فیلتر کالمن یکی از ابزارهای محبوب برای سیستم‌های خطی است که در آن فرض نویز گاوسی است. این فیلتر با کمینه‌سازی خطای میانگین مربعات، برآورد بهینه‌ای از وضعیت سیستم ارائه می‌دهد. فیلتر کالمن توسعه‌یافته این قابلیت را به سیستم‌های غیرخطی گسترش می‌دهد، با این حال،  فیلتر کالمن توسعه‌یافته در مواجهه با سیستم‌های به شدت غیرخطی یا زمانی که نویز از توزیع گاوسی پیروی نمی‌کند، محدودیت‌هایی دارد
\cite{burgard1997active}
. برای رفع این مشکل، روش‌هایی مانند فیلتر کالمن توسعه‌یافته، فیلتر کالمن توسعه‌یافته مرحله‌ای\footnote{\lr{Iterative Extended Kalman Filter (IEKF)}}
 و فیلتر کالمن نقطه سیگما\footnote{\lr{Sigma-Point Extended Kalman Filter (SPEKF)}}
  پیشنهاد شده‌اند که دقت را با در نظر گرفتن جملات مرتبه بالاتر در سری تیلور بهبود می‌بخشند.
استفاده از فیلترکالمن در طیف وسیعی از ربات‌ها دیده‌ شده‌است. شاید ربات‌های متحرک که در مکان‌یابی از انواع حسگر‌ها و الگوریتم‌ها استفاده می‌کنند، بیشترین‌ استفاد‌ه از فیلترکالمن را برای مکان‌یابی خود برده‌اند
\cite{liu2021cost, lin2018topological, negenborn2003robot}. 

با توجه به ماهیت غیرخطی پدیده‌های دنیای اطراف، استفاده از فیلترکالمن‌های خنثی گسترش بیشتری داشته‌است. 
\cite{zhan2007iterated, xian2016square} 
 روش‌های مکان‌یابی مبتنی بر فیلتر کالمن خنثی\footnote{\lr{Unscented Kalman Filter (UKF)}}
 برای ردیابی و ناوبری اینرسی مورد بررسی قرار گرفته‌اند. این روش‌ها به منظور کشف مکان بلادرنگ اهداف متحرک در محیط‌های اینترنت اشیاءبه کار رفته‌اند. در این فرآیند، اطلاعات تولید شده از گره‌های حسگرهای مورد استفاده در اینترنت اشیا برای مکان‌یابی و ردیابی اهداف متحرک استفاده می‌شود.
  
 
در مقالات 
\cite{huang2013quadratic, kim2008unscented, cheng2014compressed}
، به بررسی روش‌های مختلف برای مسئله مکان‌یابی و نقشه‌برداری همزمان\footnote{\lr{Simultaneous Localization and Mapping (SLAM)}}
 با استفاده از فیلترهای کالمن توسعه‌یافته و خنثی پرداخته شده است. ابتدا یک روش نمونه‌گیری برای فیلتر کالمن خنثی پیشنهاد شده است که پیچیدگی محاسباتی آن ثابت است و در همان حدود فیلتر کالمن توسعه‌یافته قرار می‌گیرد. همچنین یک UKF محدود شده با قابلیت مشاهده\footnote{\lr{Observability-Constrained UKF (OC-UKF)}} پیشنهاد شده که تضمین می‌کند فضای غیرقابل مشاهده سیستم مبتنی بر رگرسیون خطی UKF به همان اندازه فضای غیرخطی SLAM باشد.
 

\textbf {فیلترهای ذره‌ای}

فیلترهای ذره‌ای، که به عنوان روش‌های مونت کارلو ترتیبی\footnote{\lr{Sequential Monte Carlo}}
 نیز شناخته می‌شوند، جایگزین غیرپارامتری فیلترهای کالمن هستند. این فیلترها به خصوص برای مواجهه با سیستم‌های به شدت غیرخطی و نویزهای غیرگاوسی مفید هستند. ایده‌ی اصلی این است که توزیع پسین وضعیت ربات با استفاده از مجموعه‌ای از ذرات وزن‌دار نمایان شود. هر ذره نمایانگر یک فرضیه از وضعیت سیستم است و این ذرات بر اساس دینامیک سیستم و مشاهدات حسگرها به مرور زمان تکامل می‌یابند
\cite{fox2001particle}.

فیلترهای ذره‌ای نسبت به فیلترهای کالمن محاسبات بیشتری نیاز دارند اما در محیط‌های پیچیده انعطاف‌پذیری و دقت بیشتری ارائه می‌دهند. این فیلترها در سناریوهای مختلفی از جمله ناوبری در داخل و خارج از ساختمان، که محیط‌ها پویا و غیرقابل پیش‌بینی هستند، با موفقیت به کار گرفته شده‌اند
 \cite{fox2001particle, montemerlo2002conditional, kwok2003adaptive}.

در اینجا لازم است به برخی از محدودیت‌ها و چالش‌های استفاده از روش‌های مکان‌یابی احتمالی اشاره کنیم. یکی از مشکلات اصلی این روش‌ها، وابستگی به مدل‌های نویز و فرضیات دقیق است. برای مثال، فیلترهای کالمن و حتی نسخه‌های توسعه‌یافته آن‌ها به شدت بر این فرضیه تکیه دارند که نویزها دارای توزیع گاوسی هستند و سیستم‌ها به‌طور کامل با این فرض مطابقت دارند. اما در بسیاری از کاربردهای دنیای واقعی، شرایط محیطی و نویزهای سیستم ممکن است از این مدل‌ها پیروی نکنند، که این امر می‌تواند منجر به کاهش دقت و عملکرد روش‌های فیلتر بیزین و فیلترهای کالمن شود. همچنین، روش‌های فیلترهای ذره‌ای به دلیل تعداد زیادی از ذرات که برای حفظ دقت نیاز دارند، معمولاً از نظر محاسباتی سنگین‌تر هستند و منابع بیشتری مصرف می‌کنند. این مسئله به‌ویژه در سناریوهای زمان واقعی که نیاز به پردازش سریع و به‌موقع داده‌ها است، می‌تواند به محدودیت تبدیل شود.

علاوه بر این، یکی دیگر از معایب این روش‌ها، افزایش پیچیدگی در مواجهه با تعداد زیاد حسگرها و منابع داده است. ادغام داده‌های متعدد از حسگرهای مختلف برای بهبود دقت مکان‌یابی یک مزیت بزرگ است، اما همین موضوع می‌تواند چالش‌های محاسباتی بیشتری را به وجود بیاورد، به‌خصوص زمانی که منابع داده باهم ناسازگار باشند یا تداخل‌های احتمالی در آن‌ها رخ دهد. این مشکلات نیازمند روش‌هایی هستند که علاوه بر انعطاف‌پذیری، کارایی محاسباتی بالاتری داشته باشند و در عین حال قابلیت سازگاری با محیط‌های پیچیده و پویا را داشته باشند.

\subsection{روش‌های ترکیب حسگرها}

ترکیب حسگرها در مکان‌یابی رباتیک اهمیت زیادی دارد، زیرا اجازه می‌دهد داده‌های حاصل از چندین حسگر برای بهبود دقت و قابلیت اطمینان ادغام شوند. حسگرهای مختلف اطلاعات متفاوتی ارائه می‌دهند؛ برای مثال، حسگرهای مبتنی بر بینایی می‌توانند ویژگی‌های غنی محیطی را ارائه دهند، در حالی که حسگرهای اینرسی داده‌های حرکت قابل اطمینان را فراهم می‌کنند
\cite{srinivasan2007multiple}.

\textbf{مکان‌یابی مبتنی بر بینایی}

مکان‌یابی مبتنی بر بینایی از دوربین‌ها برای جمع‌آوری اطلاعات محیطی استفاده می‌کند. این رویکرد به ویژه در محیط‌هایی با ویژگی‌های دیداری متمایز موثر است. روش‌هایی مانند مکان‌یابی و نقشه‌برداری همزمان دیداری\footnote{Visual-SLAM}
 توسعه یافته‌اند تا نقشه‌هایی از محیط ایجاد کنند و به طور همزمان مکان ربات را تخمین بزنند
 \cite{westman2018underwater, hiebert2022introduction}.

روش‌های SLAM دیداری اغلب بر ویژگی‌هایی مانند خطوط یا گوشه‌ها در محیط تکیه می‌کنند تا نقشه‌ای ایجاد کنند. چالش اصلی در اینجا مقابله با تغییرات نور، انسدادها، و اشیاء پویا است که می‌تواند کیفیت داده‌های دیداری را کاهش دهد. برای کاهش این مشکلات، پژوهشگران داده‌های دیداری را با سایر حسگرها مانند لایدار یا واحدهای اندازه‌گیری اینرسی ترکیب کرده‌اند. کاربرد استفاده از این الگوریتم در حوزه رباتیک به طیف وسیعی از ربات‌ها همچون ربات‌های پرنده، خودران و محرک رسیده است
\cite{guan2021robot, khairuddin2015review, choset2001topological}.

\textbf{روش‌های مبتنی بر لایدار}

حسگرهای لایدار به دلیل توانایی ارائه اندازه‌گیری‌های دقیق فاصله در مکان‌یابی رباتیک به طور گسترده‌ای استفاده می‌شوند. این حسگرها پرتوهای لیزری را ارسال می‌کنند و زمانی که پرتو پس از برخورد با یک جسم بازمی‌گردد، زمان را اندازه می‌گیرند. این داده‌ها می‌توانند برای ایجاد یک نقشه دقیق از محیط و مکان‌یابی ربات در این نقشه استفاده شوند
 \cite{xu2019indoor}. 
روش‌های مبتنی بر لایدار به ویژه در محیط‌های ساختاریافته مانند راهروهای داخلی یا اتاق‌هایی با مرزهای مشخص موثر هستند. این روش‌ها اغلب با الگوریتم‌های احتمالی مانند فیلتر کالمن توسعه‌یافته یا فیلترهای ذره‌ای ترکیب می‌شوند تا دقت مکان‌یابی را افزایش دهند
 \cite{liu2022improved, blok2019robot}.
 
از طرفی دیگر، در حالی که این حسگرها اطلاعات لازم برای ایجاد نقشه‌های دقیق از محیط را فراهم می‌کنند، اما استفاده از داده‌های لایدار به‌تنهایی ممکن است تحت شرایط پیچیده محیطی با چالش‌هایی روبرو شود. در این راستا، استفاده از روش‌های مبتنی بر گراف جهت ترکیب داده‌های لایدار و سایر حسگرها نظیر حسگرهای دیداری، می‌تواند بهبود چشم‌گیری در دقت مکان‌یابی فراهم کند. یکی از مزایای گراف، توانایی آن در ترکیب اطلاعات حسگرهای مختلف به‌صورت هم‌زمان و ارائه‌ی تخمین‌های دقیق‌تر از مکان ربات است \cite{dai2022uav}. به‌ویژه در سناریوهای پیچیده و با وجود نویز، روش‌های مبتنی بر گراف توانایی بیشتری در مدیریت داده‌های چندمنبعی دارند و می‌توانند با بهره‌گیری از اطلاعات حسگرهای لایدار و دیداری، تخمین دقیقی از موقعیت ارائه دهند \cite{li2015gaussian}.
 
 استفاده از گراف در ترکیب داده‌های حسگر لایدار و دیداری می‌تواند به صورت موثری بر چالش‌هایی مانند هماهنگ‌سازی داده‌های ناهمزمان غلبه کند. این رویکرد به دلیل توانایی بالای گراف در استفاده از الگوریتم‌های بهینه‌سازی تکراری، به‌طور قابل‌ملاحظه‌ای نسبت به روش‌های متداول مانند فیلتر کالمن دقت بیشتری فراهم می‌کند \cite{song2021tightly}. همچنین، ترکیب حسگرهای دیداری و لایدار در قالب گراف، امکان مواجهه با مشکلات ناشی از نویز یا عدم‌حضور موقت حسگرها را به‌خوبی فراهم می‌آورد و در نهایت به بهبود مقاومت سیستم مکان‌یابی در محیط‌های پیچیده کمک می‌کند \cite{yang2022indoor}.
 


\subsection{روش‌های مکان‌یابی فعال}

رویکردهای سنتی مکان‌یابی اغلب منفعل هستند، به این معنا که به داده‌های حسگرهای موجود بدون تأثیرگذاری بر حرکات ربات تکیه می‌کنند. مکان‌یابی فعال، از سوی دیگر، شامل اتخاذ تصمیم‌هایی در مورد حرکات ربات برای بهبود دقت مکان‌یابی است. این رویکرد می‌تواند با هدایت ربات به سمت نواحی اطلاعاتی‌تر از محیط، عدم قطعیت را به طور قابل توجهی کاهش دهد
\cite{burgard1997active}.

\textbf{انتخاب اقدام مبتنی بر آنتروپی}

یکی از روش‌های کلیدی در مکان‌یابی فعال، کمینه‌سازی آنتروپی است که در آن ربات اقداماتی را انتخاب می‌کند که انتظار می‌رود در آینده عدم قطعیت در مکان آن را به حداقل برساند. این رویکرد در سناریوهای مختلفی با موفقیت به کار گرفته شده است، جایی که نشان داده شده است در محیط‌های با ویژگی‌های مبهم، عملکرد بهتری نسبت به روش‌های منفعل دارد
\cite{burgard1997active2}.
برای مثال، در محیط‌های اداری با راهروهای طولانی و بدون ویژگی‌های خاص، یک ربات ممکن است تصمیم بگیرد به سمت ناحیه‌ای با ویژگی‌های متمایزتر (مانند اتاقی با مبلمان) حرکت کند تا دقت مکان‌یابی خود را بهبود بخشد. فرآیند تصمیم‌گیری در این رویکرد اغلب توسط یک چارچوب احتمالی، مانند مکان‌یابی مارکوف، هدایت می‌شود که توزیع احتمالاتی را بر روی مکان‌های ممکن ربات حفظ می‌کند.

\subsection{مروری خاص بر مکان‌یابی ربات‌های کابلی}

بخش مکان‌یابی ربات‌های کابلی یکی از چالش‌های مهم در حوزه رباتیک است که به دلیل ساختار خاص این ربات‌ها، نیاز به رویکردهای پیشرفته و دقیق دارد. ربات‌های کابلی به‌عنوان یک نوع ربات موازی که از کابل‌ها به‌جای اتصالات صلب استفاده می‌کنند، به ویژه در محیط‌های بزرگ و باز مانند فیلم‌برداری‌های هوایی یا کاربردهای لجستیکی استفاده می‌شوند. با این حال، پیچیدگی‌های مکان‌یابی و تخمین وضعیت در این ربات‌ها به دلیل ماهیت انعطاف‌پذیر کابل‌ها و تأثیر نیروهای خارجی بر روی آن‌ها بسیار بالا است.

در مطالعات اخیر، ترکیب اندازه‌گیری‌های اینرسی با مدل‌های سینماتیکی برای بهبود دقت مکان‌یابی ربات‌های کابلی مورد توجه قرار گرفته است. به‌طور خاص، 
\cite{murtra2013imu}
نشان داده است که ترکیب داده‌های دینامیکی حاصل از حسگر اینرسی با راه‌حل‌های ایستای مدل‌های سینماتیکی می‌تواند منجر به بهبود قابل توجهی در دقت مکان‌یابی ربات‌های کابلی معلق شود. این روش به‌ویژه برای ربات‌های کابلی بزرگ مقیاس که در آن‌ها شکم‌دهی کابل‌ها تحت تأثیر وزن خود کابل و سایر نیروها یک چالش عمده است، تا حدودی موثر بوده است. همچنین،
\cite{kim2020robotic}
مطالعه‌ای که بر اساس ربات کابلی معلق و حسگرهای میدان مغناطیسی انجام داده است، یک رویکرد نوین برای مکان‌یابی دقیق ربات‌های کابلی در کاربردهای پزشکی ارائه داده است. در این روش، با استفاده از آرایه‌ای از حسگرهای اثر هال و یک ربات کابلی دوبعدی، مکان یک کپسول مغناطیسی درون بدن بیمار با دقت بالا تعیین می‌شود. این روش نه تنها دقت مکان‌یابی را افزایش داده، بلکه فضای کار ربات را نیز به‌طور موثری گسترش داده است.

مقاله
\cite{le2021cable}
 به بررسی دو روش جدید فیلتر کالمن توسعه‌یافته برای ترکیب داده‌های حسگرهای شتاب‌سنج و ژیروسکوپ با سینماتیک مستقیم در ربات‌های موازی کابلی جهت تخمین دقیق‌تر مکان و وضعیت بار پرداخته است. مقاله از دو رویکرد فیلتر کالمن توسعه‌یافته مبتنی بر زاویه اویلر\footnote{\lr{Euler angles}}
  و فیلتر کالمن توسعه‌یافته مبتنی بر بردار چرخش\footnote{\lr{Multiplicative Extended Kalman Filter (MEKF)}} استفاده می‌کند. همچنین، با استفاده از شبیه‌سازی‌های مونت کارلو، نتایج عددی نشان می‌دهد که استفاده از این فیلترها در مقایسه با محاسبات سینماتیک مستقیم به تنهایی، تخمین‌های دقیق‌تری از وضعیت بار ارائه می‌دهد. این تحقیقات نشان می‌دهند که با ترکیب روش‌های مختلف اندازه‌گیری و بهره‌گیری از مدل‌های دقیق‌تر سینماتیکی و دینامیکی، می‌توان به بهبود قابل توجهی در عملکرد و دقت مکان‌یابی ربات‌های کابلی دست یافت. این پیشرفت‌ها زمینه را برای کاربردهای وسیع‌تر این نوع ربات‌ها در صنایع مختلف فراهم می‌کند و پتانسیل‌های جدیدی را برای توسعه سیستم‌های رباتیک پیچیده‌تر به ارمغان می‌آورد.

\subsection{تحلیل مقایسه‌ای و بحث}

انتخاب روش مکان‌یابی بستگی به نیازهای خاص کاربرد رباتیک، شامل محیط، حسگرهای موجود، و منابع محاسباتی دارد. فیلترهای کالمن و نسخه‌های آن برای کاربردهایی که سیستم عمدتاً خطی است و نویز گاوسی است، مناسب هستند. با این حال، در محیط‌های پیچیده‌تر با ویژگی‌های غیرخطی بیشتر و نویزهای غیرگاوسی، فیلترهای ذره‌ای یک راه‌حل مقاوم‌تر ارائه می‌دهند. روش‌های مکان‌یابی فعال نمایانگر پیشرفت قابل توجهی در این زمینه هستند، زیرا به ربات اجازه می‌دهند تصمیمات آگاهانه‌ای در مورد اقدامات خود بگیرند تا عدم قطعیت را کاهش دهند. این روش‌ها به ویژه در محیط‌های پویا و یا محیط‌های به‌خوبی تعریف نشده، مفید هستند، جایی که روش‌های منفعل ممکن است با مشکل مواجه شوند.

روش‌های ترکیب حسگرها، به ویژه آن‌هایی که شامل لایدار و بینایی هستند، در بهبود دقت مکان‌یابی موثر بوده‌اند. ترکیب چندین حسگر به کاهش نقاط ضعف حسگرهای فردی کمک می‌کند و منجر به یک سیستم مکان‌یابی قابل اعتمادتر می‌شود. اما این ترکیب در پیاده‌سازی سختی‌های خاص خود را دارا می‌باشند. به همین دلیل، در سال‌های اخیر حل این مسئله‌های در حوزه گراف جا باز کرده‌اند. ترکیب حسگر‌ها و مقید‌سازی مسئله از مهم‌ترین مواردی هستند که این حوزه ‌را به این راستا کشیده‌اند.

\section{مروری بر پژوهش‌های انجام‌شده در ترکیب وظایف رباتیکی} 


در‌ حالیکه روش‌های کالیبراسیون و مکان‌یابی همزمان برای ربات‌ها کمتر مورد مطالعه قرار گرفته‌اند، این پارامترها نقش حیاتی در وظایف ناوبری و حرکت ربات‌ها ایفا می‌کنند. در بیشتر این مطالعات، از فیلترهای مختلفی مانند فیلتر کالمن و نسخه‌های پیشرفته آن برای بهبود دقت در مکان‌یابی و کالیبراسیون استفاده شده است. این روش‌ها اغلب پیچیده و نیازمند منابع محاسباتی بالا هستند، اما مزایای قابل توجهی در دقت و پایداری سیستم‌های رباتیک دارند.

\subsection{کالیبراسیون سنسورها و مکان‌یابی همزمان}
در مقالات اخیر، ترکیب کالیبراسیون سنسورها با مکان‌یابی ربات‌ها به عنوان یک مسئله مهم مطرح شده است. این روش‌ها از الگوریتم‌های فیلتر مبنا برای تخمین مکان و پارامترهای کالیبراسیون به صورت همزمان استفاده می‌کنند. به عنوان مثال، در \cite{kummerle2012simultaneous} یک روش کالیبراسیون همزمان با استفاده از فیلتر کالمن توسعه‌یافته و فیلتر کالمن خنثی ارائه شده است. این روش‌ها به منظور بهبود دقت در مکان‌یابی و کاهش خطاهای حاصل از عدم تطابق پارامترهای سنسور و مدل حرکتی ربات به کار می‌روند. 

به‌طور کلی، ترکیب فرآیندهای مکان‌یابی و کالیبراسیون حسگرها به دلیل عدم نیاز به طی کردن مراحل جداگانه برای کالیبراسیون حسگر، در حوزه رباتیک محبوبیت بیشتری پیدا کرده است. البته در بسیاری از موارد، انجام همزمان این دو فرآیند به دلیل تغییرات مداوم در پارامترهای کالیبراسیون حسگر، ضروری است. با مروری بر تحقیقات گذشته، مشاهده می‌شود که استفاده از این ایده در ربات‌هایی که دارای حسگرهای بینایی مانند دوربین هستند، بیشتر رایج است
\cite{zhou2014simultaneous, foxlin2002generalized}. 
روش‌های استفاده شده در این تحقیقات برای تعریف مسائل بهینه‌سازی کاری دشوار بوده‌است. 
\cite{reinke2019factor}
از آخرین تحقیقات انجام‌شده در این زمینه است که از ایده‌های جدیدتر مبتنی بر گراف استفاده کرده است. حرکت به سمت این دسته از الگوریتم‌ها برای این‌گونه ترکیب‌های رباتیکی، مسیری جدید در این حوزه ایجاد کرده است. 
  
 
\subsection{کالیبراسیون و مکان‌یابی همزمان ربات}

در قسمت‌های قبل موضوع کالیبراسیون و مکان‌یابی برای ربات‌ها به صورت مجزا به تفصیل مورد بررسی قرار گرفت. آنچه در این بخش مورد بررسی قرار می‌گیرد، تحقیقاتی است که هر دو الگوریتم را به‌طور همزمان در یک مسئله به کار گرفته و با حل یک مسئله یکپارچه، مکان‌یابی و کالیبراسیون را به صورت همزمان انجام می‌دهند. این رویکرد، نه تنها زمان صرف شده برای فرآیندهای کالیبراسیون جداگانه را کاهش می‌دهد، بلکه بهبود چشمگیری در دقت هر دو فرآیند ایجاد می‌کند.

با وجود تحقیقات گسترده در زمینه کالیبراسیون و مکان‌یابی، بررسی ترکیب این دو فرآیند و حل مسئله یکپارچه کمتر مورد توجه قرار گرفته است. یکی از دلایل این امر، پیچیدگی‌های قابل توجهی است که ترکیب این دو فرآیند به همراه دارد. در حالی که ترکیب کالیبراسیون حسگر و مکان‌یابی به صورت تئوری می‌تواند بار محاسباتی کمتری داشته باشد، اما با ادغام کالیبراسیون ربات، پیچیدگی‌های ریاضیاتی و محاسباتی مرتبط با سینماتیک و دینامیک ربات به طور چشمگیری افزایش می‌یابد. مقیدسازی مسئله نیز می‌تواند هدف دیگری در کنار این اهداف باشد که خود به پیچیدگی‌های ریاضیاتی مسئله اضافه می‌کند.

تحقیقات متعددی سعی در استفاده از روش‌های مبتنی بر فیلتر برای حل همزمان این دو مسئله داشته‌اند. به عنوان مثال، \cite{kummerle2011simultaneous} یکی از تحقیقات پایه در این زمینه است که با استفاده از روش فیلتر کالمن خنثی، یک مدل ساده‌شده از این ترکیب ارائه کرده است. در این تحقیق، نویسندگان تلاش کرده‌اند تا با انتخاب یک مسیر محاسباتی کمتر پیچیده، یک الگوریتم ترکیبی برای کالیبراسیون و مکان‌یابی ربات ارائه دهند. اما با وجود تلاش‌های انجام‌شده، این روش‌ها به دلیل پیچیدگی محاسباتی بالا و عدم توانایی در مدیریت بهینه‌ی منابع در مسائلی با قیود پیچیده، نتوانسته‌اند به یک راه‌حل جامع و کارا برای همه‌ی کاربردها تبدیل شوند.

در محیط‌هایی با عدم قطعیت‌های بالا، قیود مکانی پیچیده، یا تغییرات دینامیکی قابل توجه، روش‌های مبتنی بر فیلتر به دلیل فرضیات ساده‌سازانه‌ای که در مورد مدل‌های سیستم دارند، عملکرد بهینه‌ای ندارند. همچنین، همگرایی این روش‌ها در سیستم‌های با غیرخطی‌ بودن شدید و نویزهای غیرگاوسی با چالش‌های جدی مواجه است.
با توجه به این محدودیت‌ها، ترکیب کالیبراسیون و مکان‌یابی با استفاده از روش‌های مرسوم مبتنی بر فیلتر، به ویژه در مسائل با قیود پیچیده و نیاز به دقت بالا، مناسب و عملی به نظر نمی‌رسد. علاوه بر این، حل این مسائل پیچیده باید به گونه‌ای باشد که ایجاد تغییر در بخشی از مسئله، محاسبات قبلی را تحت تأثیر قرار ندهد؛ این امر به سهولت در افزودن قید به مسئله منجر می‌شود. در نتیجه، این محدودیت‌ها مسیر را برای استفاده از رویکردهای نوین‌تر، مانند الگوریتم‌های گراف‌مبنا، هموار می‌سازد. این الگوریتم‌ها، که در فصل بعد به تفصیل مورد بررسی قرار خواهند گرفت، با بهره‌گیری از ساختارهای داده‌ای کارآمدتر و مدل‌سازی دقیق‌تر، توانایی ترکیب همزمان و بهینه‌سازی فرآیندهای کالیبراسیون و مکان‌یابی را در محیط‌های پیچیده و دینامیکی فراهم می‌کنند.

\section{نتیجه‌گیری}
در حالی که روش‌های مبتنی بر فیلتر برای کالیبراسیون و مکان‌یابی ربات‌ها دقت و پایداری خوبی ارائه می‌دهند، پیچیدگی‌های پیاده‌سازی این روش‌ها می‌تواند محدودیت‌هایی ایجاد کند. این پیچیدگی‌ها، از جمله نیاز به منابع محاسباتی بالا و همچنین چالش‌های مرتبط با تغییرات دینامیکی و عدم قطعیت‌های محیطی، زمینه را برای بررسی روش‌های جایگزین مانند روش‌های مبتنی بر گراف فراهم می‌سازد. این روش‌های نوین، ممکن است با کاهش پیچیدگی‌های فرمول‌بندی، افزایش انعطاف‌پذیری مسئله و بهبود کارایی، راهکارهای بهتری برای حل مسئله مکان‌یابی و کالیبراسیون ارائه دهند. اگرچه این الگوریتم‌ها هنوز در حوزه کالیبراسیون به‌طور کامل جایگاه خود را نشان نداده‌اند، اما در تحقیقات اخیر به طور گسترده‌ای در زمینه‌های مکان‌یابی و SLAM مورد استفاده قرار گرفته‌اند. ویژگی‌های بارز این روش‌های مبتنی بر گراف در تلفیق این مسائل و بهره‌مندی از مزیت‌های آن‌ها، ما را به سمت استفاده از آن‌ها برای ایجاد یک مسئله یکپارچه شامل وظایف رباتیکی همچون کالیبراسیون حسگرها، کالیبراسیون ربات و مکان‌یابی به صورت همزمان هدایت می‌کند.

