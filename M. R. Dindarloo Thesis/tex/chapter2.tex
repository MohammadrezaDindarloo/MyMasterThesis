% !TeX root=../main.tex
\chapter{مروری بر مطالعات انجام شده}
این فصل، ابتدا مروری بر آخرین تحقیقات انجام‌شده در زمینه کالیبراسیون و روش‌های پیشنهادی آن‌ها خواهد داشت. در انتهای این بخش، تمرکز بیشتری بر روی ربات‌هایی خواهد بود که در این پایان‌نامه مورد تحقیق قرار گرفته‌اند. پس از مرور مباحث کالیبراسیون، به بررسی روش‌های مختلف مکان‌یابی پرداخته و نگاه دقیق‌تری به ربات‌های کابلی در این زمینه خواهیم داشت. در نهایت، مطالعه‌ای بر تحقیقات ارائه شده که به ترکیب وظایف رباتیکی پرداخته‌اند، انجام خواهد شد.

\section{مروری بر مطالعات انجام شده در حوزه کالیبراسیون} 



همانطور که بیان شد، کالیبراسیون نقشی کلیدی در بهبود دقت و کارایی ربات‌ها ایفا می‌کند. کالیبراسیون ربات‌ها با هدف بهبود دقت مکان‌یابی آن‌ها، از اهمیت ویژه‌ای برخوردار است. روش‌های مختلفی برای کالیبراسیون پارامترهای ربات‌ها در زمینه‌های مختلف ارائه شده است که هر یک دارای مزایا و معایب خاص خود هستند. علاوه بر روش ارائه شده در فصل قبل در دسته‌بندی سطوح کالیبراسیون، دیدگاهی دیگر در این زمینه وجود دارد که برای این مرور ادبیات برای ما مفید خواهد بود.

در این دیدگاه، اگر کالیبراسیون صرفاً از حسگرهای داخلی ربات استفاده کند، به آن خودکالیبراسیون\footnote{self-calibration} گفته می‌شود. و اگر از حسگر‌های خارجی اضافی برای کالیبراسیون استفاده شود، به آن کالیبراسیون خارجی\footnote{external-calibration} می‌گویند. در حالی که کالیبراسیون خارجی دقت بیشتری را به همراه دارد، اما از نظر هزینه و زمان‌بری، عملکرد ضعیف‌تری نسبت به خودکالیبراسیون دارد.



\textbf{ربات‌های صنعتی}

 دسته‌ای از روش‌های کالیبراسیون که در ربات‌های صنعتی مورد استفاده قرار گرفته‌اند در ادامه مورد بررسی قرار می‌گیرد. یکی از روش‌های مطرح شده توسط \cite{gan2019calibration}، روش کالیبراسیون پارامترهای سینماتیکی با استفاده از حسگرهای کششی است که به شناسایی موفقیت‌آمیز این پارامترها منجر شده است. \cite{park2011laser} روشی برای تخمین پارامترهای سینماتیکی با استفاده از حسگر لیزری ساختاریافته و یک دوربین ثابت ارائه دادند که اعتبار کالیبراسیون پارامترهای بازوی ربات انسان‌نما با 7 درجه آزادی و بازوی رباتیک با 4 درجه آزادی را به‌وسیله آزمایشات تایید کرده‌اند. 
 
 \cite{wang2012screw} از روش شناسایی محور پیچشی (SAI) بر اساس مدل نمایی (POE) استفاده کردند که دو آزمایش شبیه‌سازی نشان دادند این روش دارای دقت بالا و پایداری خوبی است. \cite{li2011optimal} یک روش جستجو برای تعیین تعداد بهینه حالات اندازه‌گیری برای بهبود دقت شناسایی ارائه کردند. با استفاده از این روش، پس از کالیبراسیون، خطاهای مکان به صورت قابل‌توجهی کاهش پیدا کرده‌است. روش‌های ارائه شده بیشتر بر مبنای روش‌های فیلتر‌مبنا هستند. این دسته از روش‌های بیان شده، از روش‌های کالیبراسیون خارجی استفاده می‌کنند.
 
 \textbf{ربات‌های چهارپا}
 
در حوزه ربات‌های چهارپا، کالیبراسیون آنلاین پارامترهای سینماتیکی از اهمیت ویژه‌ای برخوردار است. این کالیبراسیون به منظور بهبود دقت مکان‌یابی و کاهش خطاهای ناشی از تخمین اشتباه پارامترهای سینماتیکی انجام می‌شود. 
\cite{yang2022online}
 به بررسی روش‌هایی پرداخته که از داده‌های حسگرهای داخلی و خارجی برای به‌روزرسانی آنلاین پارامترهای کینماتیکی، مانند طول پاها، استفاده می‌کند. این روش‌ها بر مبنای کالیبراسیون خارجی می‌باشند. 
 
 یکی از روش‌های مطرح شده در این زمینه، استفاده از مدل‌های سینماتیکی ربات به همراه اندازه‌گیری‌های حسگرهای مفصلی، حسگرهای تماس پا و یک واحد اندازه‌گیری اینرسی (IMU) برای پیش‌بینی سرعت بدنه ربات است. در این روش، سرعت پیش‌بینی شده با سرعت اندازه‌گیری شده توسط سیستم‌های دیگر نظیر دوربین یا سیستم‌های تصویربرداری مقایسه شده و تفاوت بین آن‌ها برای به‌روزرسانی پارامترهای کینماتیکی به کار می‌رود. این روش قابلیت ترکیب در فیلترهای کالمن و یا تخمین‌گرهای مبتنی بر بهینه‌سازی پنجره‌ای را داراست. آزمایش‌های انجام شده بر روی ربات Unitree A1 نشان داده است که کالیبراسیون آنلاین پارامترهای سینماتیکی می‌تواند به طور قابل توجهی خطاهای مسافت‌پیمایی را کاهش داده و دقت مکان‌یابی را بهبود بخشد.
 
 در ادامه بررسی روش‌های کالیبراسیون در ربات‌های چهارپا، \cite{blochliger2017foot} روشی بر مبنای فیلتر کالمن و استفاده از برچسب‌هایی به صورت قیود ارائه کرده است. این روش با ادغام قیود مختلف در فرایند فیلتر کالمن، دقت کالیبراسیون را بهبود می‌بخشد و به شناسایی دقیق‌تر پارامترهای سینماتیکی ربات‌های چهارپا منجر می‌شود. 
 
 علاوه بر این، \cite{bloesch2013kinematic} نیز روشی برای کالیبراسیون سینماتیکی این نوع ربات‌ها به صورت دسته‌ای ارائه کرده است. این روش که بر مبنای دسته‌بندی داده‌های اندازه‌گیری شده است، به کالیبراسیون دقیق پارامترهای سینماتیکی می‌انجامد و به ربات اجازه می‌دهد تا با دقت بیشتری در محیط حرکت کند.  هر دو این روش‌های فیلتر‌مبنا با استفاده از حسگرهای خارجی به یک کالیبراسیون خارجی منجر شده‌اند. کالیبراسیون خارجی، گرچه دقت بالاتری را فراهم می‌کند، اما نیازمند تجهیزات اضافی و پیاده‌سازی پیچیده‌تری است.
 
 
 \textbf {ربات‌های جراح}
 
 کالیبراسیون در ربات‌های جراح به دلیل نیاز به دقت بالا در جراحی‌های کم‌تهاجمی
 \footnote{\lr{Minimal Invasive Surgery (MIS)}}،
 از اهمیت ویژه‌ای برخوردار است. بهبود دقت مکان‌یابی و کنترل ابزارهای جراحی از طریق کالیبراسیون، می‌تواند به کاهش خطاهای جراحی و افزایش ایمنی بیماران منجر شود. کالیبراسیون دقیق ابزارهای جراحی و بازوی رباتیک، به ویژه در جراحی‌هایی که نیاز به حرکت‌های دقیق و هماهنگ دارند، از اهمیت بسزایی برخوردار است.
 
\cite{roberti2020improving}
یک روش کالیبراسیون جدید برای ربات‌های جراحی با استفاده از مجموعه‌ ربات‌های تحقیقاتی da Vinci و دوربین RGB-D ارائه کرده‌اند. این روش کالیبراسیون شامل سه مرحله اصلی است: کالیبراسیون بازوهای رباتیک، کالیبراسیون دوربین، و کالیبراسیون دست-چشم
\footnote{\lr{Hand-Eye-Calibration}}.
 این روش توانسته است دقت کالیبراسیون در فضای سه‌بعدی را بهبود بخشد و در مقایسه با روش‌های سنتی، مانند روش Tsai، عملکرد بهتری داشته باشد.
 
 در 
 \cite{wang2017vision}،
روش‌های متعددی برای کالیبراسیون ربات‌های جراحی مورد بررسی قرار گرفته است. این روش‌ها شامل کالیبراسیون دست-چشم با استفاده از تصاویر دوبعدی و سه‌بعدی، و کالیبراسیون ابزارهای جراحی با استفاده از هندسه بازسازی تصویری است. همچنین، مقاله به استفاده از ابزارهای جراحی به عنوان ابزار کالیبراسیون و استفاده از روش‌های بسته برای حل معادلات کالیبراسیون اشاره دارد. این دو روش اخیر نیز کالیبراسیون‌های خارجی ارائه کرده‌اند. 

در تیم آزمایشگاهی ارس، ربات‌های جراح با ساختارهای متفاوتی توسعه یافته‌اند. یکی از این ربات‌ها، ربات جراح چشم ARASH:ASiST می‌باشد. این ربات با ساختار متوازی‌الاضلاع، یک نقطه دوران از راه دور\footnote{\lr{Remote Center of Motion (RCM)}} برای تسهیل عمل جراحی چشم ایجاد کرده است. روش‌هایی برای کالیبراسیون این ربات مورد بررسی قرار گرفته است. \cite{hassani2021kinematic} با استفاده از روش‌های غیرخطی حداقل مربعات خطا\footnote{\lr{ Nonlinear Least Square Error (NLSE)}}، روشی را برای کالیبراسیون این ربات جراح چشم ارائه کرده است. نتیجه کالیبراسیون، بهبود چشم‌گیری در نتایج ایجاد کرده است.

دومین ربات جراح چشم توسعه یافته در این مجموعه، ربات DIAMOND می‌باشد. این ربات با ساختار موازی خود، یک نقطه RCM برای این دسته از عمل‌های جراحی به صورت مکانیکی طراحی کرده است. \cite{dindarloo2023kinematic} روشی مبتنی بر پنجره‌های نسبتی داده برای کالیبراسیون استفاده کرده است. با استفاده از این روش، نیازمندی به کالیبراسیون دست-چشم حذف شده است. حل‌کننده ارائه شده برای این روش، الگوریتم لونبرگ-مارکوارت می‌باشد.

\textbf {ربات‌های کابلی}

 کالیبراسیون برای ربات‌های کابلی موازی
  \footnote{\lr{Cable-Driven Parallel Robot (CDPR)}}
  از اهمیت ویژه‌ای برخوردار است زیرا این ربات‌ها با استفاده از کابل‌های سبک و اقتصادی به انتقال انرژی و کنترل اجسام متحرک می‌پردازند. با این حال، یکی از چالش‌های اساسی در استفاده از کابل‌ها به عنوان رابط انتقال، پیچیدگی‌های ناشی از افتادگی کابل‌ها تحت تأثیر وزن آن‌ها است. این پدیده، کالیبراسیون دقیق این ربات‌ها را دشوار می‌سازد و مانع از کاربرد گسترده آن‌ها می‌شود. به همین دلیل در بسیاری از مطالعات انجام‌شده در کالیبراسیون این ربات‌ها، کابل‌ها را بدون وزرن به عنوان اجسام صلب در نظر گرفته‌اند. از آنجا که ربات‌های کابلی در بسیاری از کاربردها از جمله ماموریت‌های نجات، واقعیت مجازی، و عملیات در محیط‌های خطرناک به کار گرفته می‌شوند، داشتن کالیبراسیونی دقیق برای اطمینان از دقت و کارایی این ربات‌ها ضروری است. انتخاب استراتژی مناسب برای کالیبراسیون، چه با استفاده از حسگرهای داخلی ربات (خودکالیبراسیون) و چه با استفاده از ابزارهای مرجع خارجی، نقش کلیدی در بهبود عملکرد و دقت این سیستم‌ها ایفا می‌کند. 
 
نویسندگان در \cite{jin2018geometric} یک ابزار اندازه‌گیری طول تک‌بعدی را برای کالیبراسیون معرفی کرده‌اند. با این حال، محدودیت‌های عملکردی این حسگر باعث کاهش قابلیت کاربرد این روش در سناریوهای دنیای واقعی شده است. این روش یک کالیبراسیون خارجی را معرفی می کند. همانطور که بیان گردید، کالیبراسیون خودکار برای ربات‌های کابلی بزرگ‌مقیاس که با مشکل افتادگی کابل مواجه هستند، به‌طور مستقیم کمتر مورد بررسی قرار گرفته است. 

در \cite{borgstrom2009nims}، نویسندگان یک روش خودکالیبراسیون برای تخمین مکان اولیه یک ربات با چهار کابل و دو درجه آزادی (DoFs) معرفی کرده‌اند. به همین ترتیب، در \cite{ida2019automatic}، مکان اولیه یک ربات کابلی معلق و فروتحریک با استفاده از معادلات ایستا تخمین زده می‌شود. علاوه بر این، در \cite{pott2013cable_ForceSensorCalib} یک روش خودکالیبراسیون با استفاده از حسگرهای نیرو توسعه داده شده است، در حالی که نویسندگان \cite{darvin2018initial} روشی را برای کالیبراسیون مکان اولیه ربات و مقدار طول اولیه با استفاده از انکودرها پیشنهاد کرده‌اند. 
 
تنها تحقیقی که تاکنون برای خودکالیبراسیون ربات‌های کابلی با در نظر گرفتن اثر خم‌شدگی کابل‌ها منتشر شده است، توسط \cite{an2022all} ارائه شده است. روش بیان‌شده در این مقاله برای حل مسئله کالیبراسیون، از روش‌های مرسوم استفاده می‌کند. با این حال، در این تحقیق، پاسخ‌پذیری معادلات در قالب فرمول‌بندی ارائه شده برای ربات‌های با ابعاد بزرگ که در آن‌ها اثر خم‌شدگی نمود پیدا می‌کند، بررسی نشده است.







\section{مروری بر مطالعات انجام شده در حوزه مکان‌یابی} 

مکان‌یابی یکی از مسائل اساسی در رباتیک است که هدف آن تعیین مکان یک ربات در محیط مشخصی است. در طول سال‌ها، روش‌های مختلفی توسعه یافته‌اند که هر کدام دارای نقاط قوت و ضعف خاص خود هستند. این بخش به مرور روش‌های کلیدی مبتنی بر روش‌های احتمالی، ترکیب حسگرها، و استراتژی‌های مکان‌یابی می‌پردازد. در ادامه مروری بر هر یک از این روش‌ها و برخی از تحقیقاتی که در این زمینه انجام شده‌اند خواهیم‌داشت.


\subsection{روش‌های مکان‌یابی احتمالی}

روش‌های احتمالی به دلیل مقاومت در برابر عدم قطعیت
\footnote{uncertainty}
 و نویز در اندازه‌گیری حسگرها به طور گسترده‌ای در مکان‌یابی ربات‌های متحرک استفاده می‌شوند. این روش‌ها اغلب شامل استفاده از فیلترهای بیزین
\footnote{\lr{Bayesian-Filter}}
 ، از جمله فیلتر کالمن
\footnote{\lr{Kalman Filter(KF)}}
، فیلتر کالمن توسعه‌یافته
\footnote{\lr{Extended Kalman Filter (EKF)}}
، و فیلترهای ذره‌ای
\footnote{\lr{Particle-Filter}}
 هستند.


\textbf {فیلتر کالمن و نسخه‌های آن}

فیلتر کالمن یکی از ابزارهای محبوب برای سیستم‌های خطی است که در آن فرض نویز گاوسی است. این فیلتر با کمینه‌سازی خطای میانگین مربعات، برآورد بهینه‌ای از وضعیت سیستم ارائه می‌دهد. فیلتر کالمن توسعه‌یافته این قابلیت را به سیستم‌های غیرخطی گسترش می‌دهد، با این حال،  فیلتر کالمن توسعه‌یافته در مواجهه با سیستم‌های به شدت غیرخطی یا زمانی که نویز از توزیع گاوسی پیروی نمی‌کند، محدودیت‌هایی دارد
\cite{burgard1997active}
. برای رفع این مشکل، روش‌هایی مانند فیلتر کالمن توسعه‌یافته، فیلتر کالمن توسعه‌یافته مرحله‌ای
\footnote{\lr{Iterative Extended Kalman Filter (IEKF)}}
 و فیلتر کالمن نقطه سیگما
 \footnote{\lr{Sigma-Point Extended Kalman Filter (SPEKF)}}
  پیشنهاد شده‌اند که دقت را با در نظر گرفتن جملات مرتبه بالاتر در سری تیلور بهبود می‌بخشند.
استفاده از فیلترکالمن در طیف وسیعی از ربات‌ها دیده‌ شده‌است. شاید ربات‌های متحرک که در مکان‌یابی از انواع حسگر‌ها و الگوریتم‌ها استفاده می‌کنند، بیشترین‌ استفاد‌ه از فیلترکالمن را برای مکان‌یابی خود برده‌اند
\cite{liu2021cost, lin2018topological, negenborn2003robot}. 

با توجه به ماهیت غیرخطی پدیده‌های دنیای اطراف، استفاده از فیلترکالمن‌های بدون‌بو گسترش بیشتری داشته‌است. 
\cite{zhan2007iterated, guo2014square, xian2016square, filtermobile} 
 روش‌های مکان‌یابی مبتنی بر فیلتر کالمن بدون‌بو
\footnote{\lr{Unscented Kalman Filter (UKF)}}
  برای ردیابی و ناوبری اینرسی مورد بررسی قرار گرفته‌اند. این روش‌ها به منظور کشف مکان بلادرنگ اهداف متحرک در محیط‌های اینترنت اشیاءبه کار رفته‌اند. در این فرآیند، اطلاعات تولید شده از گره‌های حسگرهای مورد استفاده در اینترنت اشیا برای مکان‌یابی و ردیابی اهداف متحرک استفاده می‌شود.
  
 
در مقالات 
\cite{huang2013quadratic, kim2008unscented, cheng2014compressed}
، به بررسی روش‌های مختلف برای مسئله مکان‌یابی و نقشه‌برداری همزمان\footnote{\lr{Simultaneous Localization and Mapping (SLAM)}}
 با استفاده از فیلترهای کالمن توسعه‌یافته و بدون‌بو پرداخته شده است. ابتدا یک روش نمونه‌گیری برای فیلتر کالمن بدون‌بو پیشنهاد شده است که پیچیدگی محاسباتی آن ثابت است و در همان حدود فیلتر کالمن توسعه‌یافته قرار می‌گیرد. همچنین یک UKF محدود شده با قابلیت مشاهده\footnote{\lr{Observability-Constrained UKF (OC-UKF)}} پیشنهاد شده که تضمین می‌کند فضای غیرقابل مشاهده سیستم مبتنی بر رگرسیون خطی UKF به همان اندازه فضای غیرخطی SLAM باشد.
 

\textbf {فیلترهای ذره‌ای}

فیلترهای ذره‌ای، که به عنوان روش‌های مونت کارلو ترتیبی
\footnote{\lr{Sequential Monte Carlo}}
 نیز شناخته می‌شوند، جایگزین غیرپارامتری فیلترهای کالمن هستند. این فیلترها به خصوص برای مواجهه با سیستم‌های به شدت غیرخطی و نویزهای غیرگاوسی مفید هستند. ایده‌ی اصلی این است که توزیع پسین وضعیت ربات با استفاده از مجموعه‌ای از ذرات وزن‌دار نمایان شود. هر ذره نمایانگر یک فرضیه از وضعیت سیستم است و این ذرات بر اساس دینامیک سیستم و مشاهدات حسگرها به مرور زمان تکامل می‌یابند
\cite{fox2001particle}.

فیلترهای ذره‌ای نسبت به فیلترهای کالمن محاسبات بیشتری نیاز دارند اما در محیط‌های پیچیده انعطاف‌پذیری و دقت بیشتری ارائه می‌دهند. این فیلترها در سناریوهای مختلفی از جمله ناوبری در داخل و خارج از ساختمان، که محیط‌ها پویا و غیرقابل پیش‌بینی هستند، با موفقیت به کار گرفته شده‌اند
 \cite{fox2001particle, montemerlo2002conditional, kwok2003adaptive}.

\subsection{روش‌های ترکیب حسگرها}

ترکیب حسگرها در مکان‌یابی رباتیک اهمیت زیادی دارد، زیرا اجازه می‌دهد داده‌های حاصل از چندین حسگر برای بهبود دقت و قابلیت اطمینان ادغام شوند. حسگرهای مختلف اطلاعات متفاوتی ارائه می‌دهند؛ برای مثال، حسگرهای مبتنی بر بینایی می‌توانند ویژگی‌های غنی محیطی را ارائه دهند، در حالی که حسگرهای اینرسی داده‌های حرکت قابل اطمینان را فراهم می‌کنند
\cite{srinivasan2007multiple}.

\textbf{مکان‌یابی مبتنی بر بینایی}

مکان‌یابی مبتنی بر بینایی از دوربین‌ها برای جمع‌آوری اطلاعات محیطی استفاده می‌کند. این رویکرد به ویژه در محیط‌هایی با ویژگی‌های بصری متمایز موثر است. روش‌هایی مانند مکان‌یابی و نقشه‌برداری همزمان بصری (SLAM بصری) توسعه یافته‌اند تا نقشه‌هایی از محیط ایجاد کنند و به طور همزمان مکان ربات را تخمین بزنند
 \cite{westman2018underwater, hiebert2022introduction}.

روش‌های SLAM بصری اغلب بر ویژگی‌هایی مانند خطوط یا گوشه‌ها در محیط تکیه می‌کنند تا نقشه‌ای ایجاد کنند. چالش اصلی در اینجا مقابله با تغییرات نور، انسدادها، و اشیاء پویا است که می‌تواند کیفیت داده‌های بصری را کاهش دهد. برای کاهش این مشکلات، پژوهشگران داده‌های بصری را با سایر حسگرها مانند لیدار یا واحدهای اندازه‌گیری اینرسی ترکیب کرده‌اند. کاربرد استفاده از این الگوریتم در حوزه رباتیک به طیف وسیعی از ربات‌ها همچون ربات‌های پرنده، خودران و محرک رسیده است
\cite{guan2021robot, khairuddin2015review, choset2001topological}.

\textbf{روش‌های مبتنی بر لیزر}

حسگرهای لیدار به دلیل توانایی ارائه اندازه‌گیری‌های دقیق فاصله در مکان‌یابی رباتیک به طور گسترده‌ای استفاده می‌شوند. این حسگرها پرتوهای لیزری را ارسال می‌کنند و زمانی که پرتو پس از برخورد با یک جسم بازمی‌گردد، زمان را اندازه می‌گیرند. این داده‌ها می‌توانند برای ایجاد یک نقشه دقیق از محیط و مکان‌یابی ربات در این نقشه استفاده شوند
 \cite{xu2019indoor}. 
روش‌های مبتنی بر لیزر به ویژه در محیط‌های ساختاریافته مانند راهروهای داخلی یا اتاق‌هایی با مرزهای مشخص موثر هستند. این روش‌ها اغلب با الگوریتم‌های احتمالی مانند فیلتر کالمن توسعه‌یافته یا فیلترهای ذره‌ای ترکیب می‌شوند تا دقت مکان‌یابی را افزایش دهند
 \cite{liu2022improved, blok2019robot}.
 
 روش‌های مبتنی بر گراف با توجه به مزایایی که دارند، در سال‌های اخیر در ابعاد زیادی از ربات‌ها، مورد استفاده قرار گرفته‌اند. 
 \cite{yang2022indoor, song2021tightly, li2015gaussian, dai2022uav} بااستفاده از روش‌های مبتنی بر گراف، مسئله مکان‌یابی را با ایجاد داده‌های بینایی با دیگر داده‌ها مورد استفاده قرار داده‌اند.

\subsection{روش‌های مکان‌یابی فعال}

رویکردهای سنتی مکان‌یابی اغلب منفعل هستند، به این معنا که به داده‌های حسگرهای موجود بدون تأثیرگذاری بر حرکات ربات تکیه می‌کنند. مکان‌یابی فعال، از سوی دیگر، شامل اتخاذ تصمیم‌هایی در مورد حرکات ربات برای بهبود دقت مکان‌یابی است. این رویکرد می‌تواند با هدایت ربات به سمت نواحی اطلاعاتی‌تر از محیط، عدم قطعیت را به طور قابل توجهی کاهش دهد
\cite{burgard1997active}.

\textbf{انتخاب اقدام مبتنی بر آنتروپی}

یکی از روش‌های کلیدی در مکان‌یابی فعال، کمینه‌سازی آنتروپی است که در آن ربات اقداماتی را انتخاب می‌کند که انتظار می‌رود در آینده عدم قطعیت در مکان آن را به حداقل برساند. این رویکرد در سناریوهای مختلفی با موفقیت به کار گرفته شده است، جایی که نشان داده شده است در محیط‌های با ویژگی‌های مبهم، عملکرد بهتری نسبت به روش‌های منفعل دارد
\cite{burgard1997active2}.
برای مثال، در محیط‌های اداری با راهروهای طولانی و بدون ویژگی‌های خاص، یک ربات ممکن است تصمیم بگیرد به سمت ناحیه‌ای با ویژگی‌های متمایزتر (مانند اتاقی با مبلمان) حرکت کند تا دقت مکان‌یابی خود را بهبود بخشد. فرآیند تصمیم‌گیری در این رویکرد اغلب توسط یک چارچوب احتمالی، مانند مکان‌یابی مارکوف، هدایت می‌شود که توزیع احتمالاتی را بر روی مکان‌های ممکن ربات حفظ می‌کند.

\subsection{مروری خاص بر مکان‌یابی ربات‌های کابلی}

بخش مکان‌یابی ربات‌های کابلی یکی از چالش‌های مهم در حوزه رباتیک است که به دلیل ساختار خاص این ربات‌ها، نیاز به رویکردهای پیشرفته و دقیق دارد. ربات‌های کابلی به‌عنوان یک نوع ربات موازی که از کابل‌ها به‌جای اتصالات سخت استفاده می‌کنند، به ویژه در محیط‌های بزرگ و باز مانند فیلم‌برداری‌های هوایی یا کاربردهای لجستیکی استفاده می‌شوند. با این حال، پیچیدگی‌های مکان‌یابی و تخمین وضعیت در این ربات‌ها به دلیل ماهیت انعطاف‌پذیر کابل‌ها و تأثیر نیروهای خارجی بر روی آن‌ها بسیار بالا است.

در مطالعات اخیر، ترکیب اندازه‌گیری‌های اینرسی با مدل‌های سینماتیکی برای بهبود دقت مکان‌یابی ربات‌های کابلی مورد توجه قرار گرفته است. به‌طور خاص، 
\cite{murtra2013imu}
نشان داده است که ترکیب داده‌های دینامیکی حاصل از حسگر اینرسی با راه‌حل‌های ایستای مدل‌های سینماتیکی می‌تواند منجر به بهبود قابل توجهی در دقت مکان‌یابی ربات‌های کابلی معلق شود. این روش به‌ویژه برای ربات‌های کابلی بزرگ مقیاس که در آن‌ها شکم‌دهی کابل‌ها تحت تأثیر وزن خود کابل و سایر نیروها یک چالش عمده است، تا حدودی موثر بوده است. همچنین،
\cite{kim2020robotic}
مطالعه‌ای که بر اساس ربات کابلی معلق و حسگرهای میدان مغناطیسی انجام داده است، یک رویکرد نوین برای مکان‌یابی دقیق ربات‌های کابلی در کاربردهای پزشکی ارائه داده است. در این روش، با استفاده از آرایه‌ای از حسگرهای اثر هال و یک ربات کابلی دوبعدی، مکان یک کپسول مغناطیسی درون بدن بیمار با دقت بالا تعیین می‌شود. این روش نه تنها دقت مکان‌یابی را افزایش داده، بلکه فضای کار ربات را نیز به‌طور موثری گسترش داده است.

مقاله
\cite{le2021cable}
 به بررسی دو روش جدید فیلتر کالمن توسعه‌یافته برای ترکیب داده‌های حسگرهای شتاب‌سنج و ژیروسکوپ با سینماتیک مستقیم در ربات‌های موازی کابلی جهت تخمین دقیق‌تر مکان و وضعیت بار پرداخته است. مقاله از دو رویکرد فیلتر کالمن توسعه‌یافته مبتنی بر زاویه اویلر
 \footnote{\lr{Euler angles}}
  و فیلتر کالمن توسعه‌یافته مبتنی بر بردار چرخش\footnote{\lr{Multiplicative Extended Kalman Filter (MEKF)}} استفاده می‌کند. همچنین، با استفاده از شبیه‌سازی‌های مونت کارلو، نتایج عددی نشان می‌دهد که استفاده از این فیلترها در مقایسه با محاسبات سینماتیک مستقیم به تنهایی، تخمین‌های دقیق‌تری از وضعیت بار ارائه می‌دهد. این تحقیقات نشان می‌دهند که با ترکیب روش‌های مختلف اندازه‌گیری و بهره‌گیری از مدل‌های دقیق‌تر سینماتیکی و دینامیکی، می‌توان به بهبود قابل توجهی در عملکرد و دقت مکان‌یابی ربات‌های کابلی دست یافت. این پیشرفت‌ها زمینه را برای کاربردهای وسیع‌تر این نوع ربات‌ها در صنایع مختلف فراهم می‌کند و پتانسیل‌های جدیدی را برای توسعه سیستم‌های رباتیک پیچیده‌تر به ارمغان می‌آورد.

\subsection{تحلیل مقایسه‌ای و بحث}

انتخاب روش مکان‌یابی بستگی به نیازهای خاص کاربرد رباتیک، شامل محیط، حسگرهای موجود، و منابع محاسباتی دارد. فیلترهای کالمن و نسخه‌های آن برای کاربردهایی که سیستم عمدتاً خطی است و نویز گاوسی است، مناسب هستند. با این حال، در محیط‌های پیچیده‌تر با غیرخطی بودن‌ها و نویزهای غیرگاوسی، فیلترهای ذره‌ای یک راه‌حل مقاوم‌تر ارائه می‌دهند. روش‌های مکان‌یابی فعال نمایانگر پیشرفت قابل توجهی در این زمینه هستند، زیرا به ربات اجازه می‌دهند تصمیمات آگاهانه‌ای در مورد اقدامات خود بگیرند تا عدم قطعیت را کاهش دهند. این روش‌ها به ویژه در محیط‌های پویا و یا محیط‌های به‌خوبی تعریف نشده، مفید هستند، جایی که روش‌های منفعل ممکن است با مشکل مواجه شوند.

روش‌های ترکیب حسگرها، به ویژه آن‌هایی که شامل لیدار و بینایی هستند، در بهبود دقت مکان‌یابی موثر بوده‌اند. ترکیب چندین حسگر به کاهش نقاط ضعف حسگرهای فردی کمک می‌کند و منجر به یک سیستم مکان‌یابی قابل اعتمادتر می‌شود. اما این ترکیب در پیاده‌سازی سختی‌های خاص خود را دارا می‌باشند. به همین دلیل، در سال‌های اخیر حل این مسئله‌های در حوزه گراف جا باز کرده‌اند. ترکیب حسگر‌ها و مقید‌سازی مسئله از مهم‌ترین مواردی هستند که این حوزه ‌را به این راستا کشیده‌اند.

\section{مروری بر مطالعات انجام‌شده در ترکیب وظایف رباتیکی} 


در‌ حالیکه روش‌های کالیبراسیون و مکان‌یابی همزمان برای ربات‌ها کمتر مورد مطالعه قرار گرفته‌اند، این پارامترها نقش حیاتی در وظایف ناوبری و حرکت ربات‌ها ایفا می‌کنند. در بیشتر این مطالعات، از فیلترهای مختلفی مانند فیلتر کالمن و نسخه‌های پیشرفته آن برای بهبود دقت در مکان‌یابی و کالیبراسیون استفاده شده است. این روش‌ها اغلب پیچیده و نیازمند منابع محاسباتی بالا هستند، اما مزایای قابل توجهی در دقت و پایداری سیستم‌های رباتیک دارند.

\subsection{کالیبراسیون سنسورها و مکان‌یابی همزمان}
در مقالات اخیر، ترکیب کالیبراسیون سنسورها با مکان‌یابی ربات‌ها به عنوان یک مسئله مهم مطرح شده است. این روش‌ها از الگوریتم‌های فیلتر مبنا برای تخمین مکان و پارامترهای کالیبراسیون به صورت همزمان استفاده می‌کنند. به عنوان مثال، در \cite{kummerle2012simultaneous} یک روش کالیبراسیون همزمان با استفاده از فیلتر کالمن توسعه‌یافته و فیلتر کالمن بدون‌بو ارائه شده است. این روش‌ها به منظور بهبود دقت در مکان‌یابی و کاهش خطاهای حاصل از عدم تطابق پارامترهای سنسور و مدل حرکتی ربات به کار می‌روند. 

به‌طور کلی، ترکیب فرآیندهای مکان‌یابی و کالیبراسیون حسگرها به دلیل عدم نیاز به طی کردن مراحل جداگانه برای کالیبراسیون حسگر، در حوزه رباتیک محبوبیت بیشتری پیدا کرده است. البته در بسیاری از موارد، انجام همزمان این دو فرآیند به دلیل تغییرات مداوم در پارامترهای کالیبراسیون حسگر، ضروری است. با مروری بر تحقیقات گذشته، مشاهده می‌شود که استفاده از این ایده در ربات‌هایی که دارای حسگرهای بینایی مانند دوربین هستند، بیشتر رایج است
\cite{zhou2014simultaneous, foxlin2002generalized}. 
روش‌های استفاده شده در این تحقیقات برای تعریف مسائل بهینه‌سازی کاری دشوار بوده‌است. 
\cite{reinke2019factor}
از آخرین تحقیقات انجام‌شده در این زمینه است که از ایده‌های جدیدتر مبتنی بر گراف استفاده کرده است. حرکت به سمت این دسته از الگوریتم‌ها برای این‌گونه ترکیب‌های رباتیکی، مسیری جدید در این حوزه ایجاد کرده است. 
  
 
\subsection{کالیبراسیون و مکان‌یابی همزمان ربات}

در قسمت‌های قبل موضوع کالیبراسیون و مکان‌یابی برای ربات‌ها به صورت مجزا به تفصیل مورد بررسی قرار گرفت. آنچه در این بخش مورد بررسی قرار می‌گیرد، تحقیقاتی است که هر دو الگوریتم را به‌طور همزمان در یک مسئله به کار گرفته و با حل یک مسئله یکپارچه، مکان‌یابی و کالیبراسیون را به صورت همزمان انجام می‌دهند. این رویکرد، نه تنها زمان صرف شده برای فرآیندهای کالیبراسیون جداگانه را کاهش می‌دهد، بلکه بهبود چشمگیری در دقت هر دو فرآیند ایجاد می‌کند.

با وجود تحقیقات گسترده در زمینه کالیبراسیون و مکان‌یابی، بررسی ترکیب این دو فرآیند و حل مسئله یکپارچه کمتر مورد توجه قرار گرفته است. یکی از دلایل این امر، پیچیدگی‌های قابل توجهی است که ترکیب این دو فرآیند به همراه دارد. در حالی که ترکیب کالیبراسیون حسگر و مکان‌یابی به صورت تئوری می‌تواند بار محاسباتی کمتری داشته باشد، اما با ادغام کالیبراسیون ربات، پیچیدگی‌های ریاضیاتی و محاسباتی مرتبط با سینماتیک و دینامیک ربات به طور چشمگیری افزایش می‌یابد. مقیدسازی مسئله نیز می‌تواند هدف دیگری در کنار این اهداف باشد که خود به پیچیدگی‌های ریاضیاتی مسئله اضافه می‌کند.

تحقیقات متعددی سعی در استفاده از روش‌های مبتنی بر فیلتر برای حل همزمان این دو مسئله داشته‌اند. به عنوان مثال، \cite{kummerle2011simultaneous} یکی از تحقیقات پایه در این زمینه است که با استفاده از روش فیلتر کالمن بدون‌بو، یک مدل ساده‌شده از این ترکیب ارائه کرده است. در این تحقیق، نویسندگان تلاش کرده‌اند تا با انتخاب یک مسیر محاسباتی کمتر پیچیده، یک الگوریتم ترکیبی برای کالیبراسیون و مکان‌یابی ربات ارائه دهند. اما با وجود تلاش‌های انجام‌شده، این روش‌ها به دلیل پیچیدگی محاسباتی بالا و عدم توانایی در مدیریت بهینه‌ی منابع در مسائلی با قیود پیچیده، نتوانسته‌اند به یک راه‌حل جامع و کارا برای همه‌ی کاربردها تبدیل شوند.

در محیط‌هایی با عدم قطعیت‌های بالا، قیود مکانی پیچیده، یا تغییرات دینامیکی قابل توجه، روش‌های مبتنی بر فیلتر به دلیل فرضیات ساده‌سازانه‌ای که در مورد مدل‌های سیستم دارند، عملکرد بهینه‌ای ندارند. همچنین، همگرایی این روش‌ها در سیستم‌های با غیرخطی‌ بودن شدید و نویزهای غیرگاوسی با چالش‌های جدی مواجه است.
با توجه به این محدودیت‌ها، ترکیب کالیبراسیون و مکان‌یابی با استفاده از روش‌های مرسوم مبتنی بر فیلتر، به ویژه در مسائل با قیود پیچیده و نیاز به دقت بالا، مناسب و عملی به نظر نمی‌رسد. علاوه بر این، حل این مسائل پیچیده باید به گونه‌ای باشد که ایجاد تغییر در بخشی از مسئله، محاسبات قبلی را تحت تأثیر قرار ندهد؛ این امر به سهولت در افزودن قید به مسئله منجر می‌شود. در نتیجه، این محدودیت‌ها مسیر را برای استفاده از رویکردهای نوین‌تر، مانند الگوریتم‌های گراف‌مبنا، هموار می‌سازد. این الگوریتم‌ها، که در فصل بعد به تفصیل مورد بررسی قرار خواهند گرفت، با بهره‌گیری از ساختارهای داده‌ای کارآمدتر و مدل‌سازی دقیق‌تر، توانایی ترکیب همزمان و بهینه‌سازی فرآیندهای کالیبراسیون و مکان‌یابی را در محیط‌های پیچیده و دینامیکی فراهم می‌کنند.

\section{نتیجه‌گیری}
در حالی که روش‌های مبتنی بر فیلتر برای کالیبراسیون و مکان‌یابی ربات‌ها دقت و پایداری خوبی ارائه می‌دهند، پیچیدگی‌های پیاده‌سازی این روش‌ها می‌تواند محدودیت‌هایی ایجاد کند. این پیچیدگی‌ها، از جمله نیاز به منابع محاسباتی بالا و همچنین چالش‌های مرتبط با تغییرات دینامیکی و عدم قطعیت‌های محیطی، زمینه را برای بررسی روش‌های جایگزین مانند روش‌های مبتنی بر گراف فراهم می‌سازد. این روش‌های نوین، ممکن است با کاهش پیچیدگی‌های فرمول‌بندی، افزایش انعطاف‌پذیری مسئله و بهبود کارایی، راهکارهای بهتری برای حل مسئله مکان‌یابی و کالیبراسیون ارائه دهند. اگرچه این الگوریتم‌ها هنوز در حوزه کالیبراسیون به‌طور کامل جایگاه خود را نشان نداده‌اند، اما در تحقیقات اخیر به طور گسترده‌ای در زمینه‌های مکان‌یابی و SLAM مورد استفاده قرار گرفته‌اند. ویژگی‌های بارز این روش‌های مبتنی بر گراف در تلفیق این مسائل و بهره‌مندی از مزیت‌های آن‌ها، ما را به سمت استفاده از آن‌ها برای ایجاد یک مسئله یکپارچه شامل وظایف رباتیکی همچون کالیبراسیون حسگرها، کالیبراسیون ربات و مکان‌یابی به صورت همزمان هدایت می‌کند.

