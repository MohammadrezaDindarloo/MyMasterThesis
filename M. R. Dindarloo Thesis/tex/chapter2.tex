% !TeX root=../main.tex
\chapter{مروری بر مطالعات انجام شده}
این فصل، ابتدا مروری بر آخرین تحقیقات انجام‌شده در زمینه کالیبراسیون و روش‌های پیشنهادی آن‌ها خواهد داشت. در انتهای این بخش، تمرکز بیشتری بر روی ربات‌هایی خواهد بود که در این پایان‌نامه مورد تحقیق قرار گرفته‌اند. پس از مرور مباحث کالیبراسیون، به بررسی روش‌های مختلف موقعیت‌یابی پرداخته و نگاه دقیق‌تری به ربات‌های کابلی در این زمینه خواهیم داشت. در نهایت، مطالعه‌ای بر تحقیقات ارائه شده که به ترکیب وظایف رباتیکی پرداخته‌اند، انجام خواهد شد. با توجه به مزیت‌هایی که روش‌های گراف‌مبنا نسبت به روش‌های مرسوم دارند(در فصل آتی این مزیت ها به تفصیل مورد بررسی قرار خواهند گرفت)، تمرکز برای ترکیب این وظایف و حسگرها، این روش‌های بر پایه گراف خواهدبود. 

\section{مروری بر مطالعات انجام شده در حوزه کالیبراسیون} 



همانطور که بیان شد، کالیبراسیون نقشی کلیدی در بهبود دقت و کارایی ربات‌ها ایفا می‌کند. کالیبراسیون ربات‌ها با هدف بهبود دقت موقعیت‌یابی آن‌ها، از اهمیت ویژه‌ای برخوردار است. روش‌های مختلفی برای کالیبراسیون پارامترهای ربات‌ها در زمینه‌های مختلف ارائه شده است که هر یک دارای مزایا و معایب خاص خود هستند. علاوه بر روش ارائه شده در فصل قبل در دسته‌بندی سطوح کالیبراسیون، دیدگاهی دیگر در این زمینه وجود دارد که برای این مرور ادبیات برای ما مفید خواهد بود.

در این دیدگاه، اگر کالیبراسیون صرفاً از حسگرهای داخلی ربات استفاده کند، به آن خودکالیبراسیون\footnote{self-calibration} گفته می‌شود. و اگر از حسگر‌های خارجی اضافی برای کالیبراسیون استفاده شود، به آن کالیبراسیون خارجی\footnote{external-calibration} می‌گویند. در حالی که کالیبراسیون خارجی دقت بیشتری را به همراه دارد، اما از نظر هزینه و زمان‌بری، عملکرد ضعیف‌تری نسبت به خودکالیبراسیون دارد.



\textbf{ربات‌های صنعتی}

 دسته‌ای از روش‌های کالیبراسیون که در ربات‌های صنعتی مورد استفاده قرار گرفته‌اند در ادامه مورد بررسی قرار می‌گیرد. یکی از روش‌های مطرح شده توسط \cite{gan2019calibration}، روش کالیبراسیون پارامترهای سینماتیکی با استفاده از حسگرهای کششی است که به شناسایی موفقیت‌آمیز این پارامترها منجر شده است. \cite{park2011laser} روشی برای تخمین پارامترهای سینماتیکی با استفاده از حسگر لیزری ساختاریافته و یک دوربین ثابت ارائه دادند که اعتبار کالیبراسیون پارامترهای بازوی ربات انسان‌نما با 7 درجه آزادی و بازوی رباتیک با 4 درجه آزادی را به‌وسیله آزمایشات تایید کرده‌اند. 
 
 \cite{wang2012screw} از روش شناسایی محور پیچشی (SAI) بر اساس مدل نمایی (POE) استفاده کردند که دو آزمایش شبیه‌سازی نشان دادند این روش دارای دقت بالا و پایداری خوبی است. \cite{li2011optimal} یک روش جستجو برای تعیین تعداد بهینه حالات اندازه‌گیری برای بهبود دقت شناسایی ارائه کردند. با استفاده از این روش، پس از کالیبراسیون، خطاهای موقعیت به صورت قابل‌توجهی کاهش پیدا کرده‌است. روش‌های ارائه شده بیشتر بر مبنای روش‌های فیلتر‌مبنا هستند. این دسته از روش‌های بیان شده، از روش‌های کالیبراسیون خارجی استفاده می‌کنند.
 
 \textbf{ربات‌های چهارپا}
 
در حوزه ربات‌های چهارپا، کالیبراسیون آنلاین پارامترهای سینماتیکی از اهمیت ویژه‌ای برخوردار است. این کالیبراسیون به منظور بهبود دقت مکان‌یابی و کاهش خطاهای ناشی از تخمین اشتباه پارامترهای سینماتیکی انجام می‌شود. 
\cite{yang2022online}
 به بررسی روش‌هایی پرداخته که از داده‌های حسگرهای داخلی و خارجی برای به‌روزرسانی آنلاین پارامترهای کینماتیکی، مانند طول پاها، استفاده می‌کند. این روش‌ها بر مبنای کالیبراسیون خارجی می‌باشند. 
 
 یکی از روش‌های مطرح شده در این زمینه، استفاده از مدل‌های سینماتیکی ربات به همراه اندازه‌گیری‌های حسگرهای مفصلی، حسگرهای تماس پا و یک واحد اندازه‌گیری اینرسی (IMU) برای پیش‌بینی سرعت بدنه ربات است. در این روش، سرعت پیش‌بینی شده با سرعت اندازه‌گیری شده توسط سیستم‌های دیگر نظیر دوربین یا سیستم‌های تصویربرداری مقایسه شده و تفاوت بین آن‌ها برای به‌روزرسانی پارامترهای کینماتیکی به کار می‌رود. این روش قابلیت ترکیب در فیلترهای کالمن و یا تخمین‌گرهای مبتنی بر بهینه‌سازی پنجره‌ای را داراست. آزمایش‌های انجام شده بر روی ربات Unitree A1 نشان داده است که کالیبراسیون آنلاین پارامترهای سینماتیکی می‌تواند به طور قابل توجهی خطاهای مسافت‌پیمایی را کاهش داده و دقت مکان‌یابی را بهبود بخشد.
 
 در ادامه بررسی روش‌های کالیبراسیون در ربات‌های چهارپا، \cite{blochliger2017foot} روشی بر مبنای فیلتر کالمن و استفاده از برچسب‌هایی به صورت قیود ارائه کرده است. این روش با ادغام قیود مختلف در فرایند فیلتر کالمن، دقت کالیبراسیون را بهبود می‌بخشد و به شناسایی دقیق‌تر پارامترهای سینماتیکی ربات‌های چهارپا منجر می‌شود. 
 
 علاوه بر این، \cite{bloesch2013kinematic} نیز روشی برای کالیبراسیون سینماتیکی این نوع ربات‌ها به صورت دسته‌ای ارائه کرده است. این روش که بر مبنای دسته‌بندی داده‌های اندازه‌گیری شده است، به کالیبراسیون دقیق پارامترهای سینماتیکی می‌انجامد و به ربات اجازه می‌دهد تا با دقت بیشتری در محیط حرکت کند.  هر دو این روش‌های فیلتر‌مبنا با استفاده از حسگرهای خارجی به یک کالیبراسیون خارجی منجر شده‌اند. کالیبراسیون خارجی، گرچه دقت بالاتری را فراهم می‌کند، اما نیازمند تجهیزات اضافی و پیاده‌سازی پیچیده‌تری است.
 
 
 \textbf {ربات‌های جراح}
 
 کالیبراسیون در ربات‌های جراح به دلیل نیاز به دقت بالا در جراحی‌های کم‌تهاجمی
 \footnote{(MIS) Minimal Invasive Surgery}،
 از اهمیت ویژه‌ای برخوردار است. بهبود دقت موقعیت‌یابی و کنترل ابزارهای جراحی از طریق کالیبراسیون، می‌تواند به کاهش خطاهای جراحی و افزایش ایمنی بیماران منجر شود. کالیبراسیون دقیق ابزارهای جراحی و بازوی رباتیک، به ویژه در جراحی‌هایی که نیاز به حرکت‌های دقیق و هماهنگ دارند، از اهمیت بسزایی برخوردار است.
 
\cite{roberti2020improving}
یک روش کالیبراسیون جدید برای ربات‌های جراحی با استفاده از مجموعه‌ ربات‌های تحقیقاتی da Vinci و دوربین RGB-D ارائه کرده‌اند. این روش کالیبراسیون شامل سه مرحله اصلی است: کالیبراسیون بازوهای رباتیک، کالیبراسیون دوربین، و کالیبراسیون دست-چشم
\footnote{Hand-Eye-Calibration}.
 این روش توانسته است دقت کالیبراسیون در فضای سه‌بعدی را بهبود بخشد و در مقایسه با روش‌های سنتی، مانند روش Tsai، عملکرد بهتری داشته باشد.
 
 در 
 \cite{wang2017vision}،
روش‌های متعددی برای کالیبراسیون ربات‌های جراحی مورد بررسی قرار گرفته است. این روش‌ها شامل کالیبراسیون دست-چشم با استفاده از تصاویر دوبعدی و سه‌بعدی، و کالیبراسیون ابزارهای جراحی با استفاده از هندسه بازسازی تصویری است. همچنین، مقاله به استفاده از ابزارهای جراحی به عنوان ابزار کالیبراسیون و استفاده از روش‌های بسته برای حل معادلات کالیبراسیون اشاره دارد. این دو روش اخیر نیز کالیبراسیون‌های خارجی ارائه کرده‌اند. 

در تیم آزمایشگاهی ارس، ربات‌های جراح با ساختارهای متفاوتی توسعه یافته‌اند. یکی از این ربات‌ها، ربات جراح چشم ARASH:ASiST می‌باشد. این ربات با ساختار متوازی‌الاضلاع، یک نقطه دوران از راه دور\footnote{(RCM) Remote Center of Motion} برای تسهیل عمل جراحی چشم ایجاد کرده است. روش‌هایی برای کالیبراسیون این ربات مورد بررسی قرار گرفته است. \cite{hassani2021kinematic} با استفاده از روش‌های غیرخطی حداقل مربعات خطا\footnote{(NLSE) Nonlinear Least Square Error}، روشی را برای کالیبراسیون این ربات جراح چشم ارائه کرده است. نتیجه کالیبراسیون، بهبود چشم‌گیری در نتایج ایجاد کرده است.

دومین ربات جراح چشم توسعه یافته در این مجموعه، ربات DIAMOND می‌باشد. این ربات با ساختار موازی خود، یک نقطه RCM برای این دسته از عمل‌های جراحی به صورت مکانیکی طراحی کرده است. \cite{dindarloo2023kinematic} روشی مبتنی بر پنجره‌های نسبتی داده برای کالیبراسیون استفاده کرده است. با استفاده از این روش، نیازمندی به کالیبراسیون دست-چشم حذف شده است. حل‌کننده ارائه شده برای این روش، الگوریتم لونبرگ-مارکوارت می‌باشد.

\textbf {ربات‌های کابلی}

 کالیبراسیون برای ربات‌های کابلی از اهمیت ویژه‌ای برخوردار است زیرا این ربات‌ها با استفاده از کابل‌های سبک و اقتصادی به انتقال انرژی و کنترل اجسام متحرک می‌پردازند. با این حال، یکی از چالش‌های اساسی در استفاده از کابل‌ها به عنوان رابط انتقال، پیچیدگی‌های ناشی از افتادگی کابل‌ها تحت تأثیر وزن آن‌ها است. این پدیده، کالیبراسیون دقیق این ربات‌ها را دشوار می‌سازد و مانع از کاربرد گسترده آن‌ها می‌شود. به همین دلیل در بسیاری از مطالعات انجام‌شده در کالیبراسیون این ربات‌ها، کابل‌ها را بدون وزرن به عنوان اجسام صلب در نظر گرفته‌اند. از آنجا که ربات‌های کابلی در بسیاری از کاربردها از جمله ماموریت‌های نجات، واقعیت مجازی، و عملیات در محیط‌های خطرناک به کار گرفته می‌شوند، داشتن کالیبراسیونی دقیق برای اطمینان از دقت و کارایی این ربات‌ها ضروری است. انتخاب استراتژی مناسب برای کالیبراسیون، چه با استفاده از حسگرهای داخلی ربات (خودکالیبراسیون) و چه با استفاده از ابزارهای مرجع خارجی، نقش کلیدی در بهبود عملکرد و دقت این سیستم‌ها ایفا می‌کند. 
 
نویسندگان در \cite{jin2018geometric} یک ابزار اندازه‌گیری طول تک‌بعدی را برای کالیبراسیون معرفی کرده‌اند. با این حال، محدودیت‌های عملکردی این حسگر باعث کاهش قابلیت کاربرد این روش در سناریوهای دنیای واقعی شده است. این روش یک کالیبراسیون خارجی را معرفی می کند. همانطور که بیان گردید، کالیبراسیون خودکار برای ربات‌های کابلی بزرگ‌مقیاس که با مشکل افتادگی کابل مواجه هستند، به‌طور مستقیم کمتر مورد بررسی قرار گرفته است. 

در \cite{borgstrom2009nims}، نویسندگان یک روش خودکالیبراسیون برای تخمین موقعیت اولیه یک ربات با چهار کابل و دو درجه آزادی (DoFs) معرفی کرده‌اند. به همین ترتیب، در \cite{ida2019automatic}، موقعیت اولیه یک ربات کابلی معلق و فروتحریک با استفاده از معادلات استاتیک تخمین زده می‌شود. علاوه بر این، در \cite{pott2013cable_ForceSensorCalib} یک روش خودکالیبراسیون با استفاده از حسگرهای نیرو توسعه داده شده است، در حالی که نویسندگان \cite{darvin2018initial} روشی را برای کالیبراسیون مکان اولیه ربات و مقدار طول اولیه با استفاده از انکودرها پیشنهاد کرده‌اند. 
 
تنها کاری که تاکنون برای خودکالیبراسیون ربات‌های کابلی با در نظر گرفتن اثر خم‌شدگی کابل‌ها منتشر شده است، توسط \cite{an2022all} ارائه شده است. روش بیان‌شده در این مقاله برای حل مسئله کالیبراسیون، از روش‌های مرسوم استفاده می‌کند. با این حال، در این تحقیق، پاسخ‌پذیری معادلات در قالب فرمول‌بندی ارائه شده برای ربات‌های با ابعاد بزرگ که در آن‌ها اثر خم‌شدگی نمود پیدا می‌کند، بررسی نشده است.


\section{مروری بر مطالعات انجام شده در حوزه مکان‌یابی} 





\section{مروری بر مطالعات انجام‌شده در ترکیب وظایف رباتیکی با استفاده از الگوریتم‌های گراف‌مبنا} 

reinke2019factor ----- کالیبراسیون دوربین و ربات با فکتور گراف