% در این فایل، دستورها و تنظیمات مورد نیاز، آورده شده است.
%-------------------------------------------------------------------------------------------------------------------
% دستوراتی که پوشه پیش‌فرض زیرفایل‌های tex را مشخص می‌کند.
%\makeatletter
%\def\input@path{{./tex/}}
%\makeatother
% در ورژن جدید زی‌پرشین برای تایپ متن‌های ریاضی، این سه بسته، حتماً باید فراخوانی شود
\usepackage{amsthm,amssymb,amsmath}
% بسته‌ای برای تنطیم حاشیه‌های بالا، پایین، چپ و راست صفحه
\usepackage[a4paper, top=40mm, bottom=40mm, outer=25mm, inner=35mm]{geometry}
% بسته‌‌ای برای ظاهر شدن شکل‌ها و تعیین آدرس تصاویر
\usepackage[final]{graphicx}
\graphicspath{{./img/}}



% بسته‌های مورد نیاز برای نوشتن کدها، رنگ‌آمیزی آنها و تعیین پوشهٔ کدها
%%man ezafe kardam
\usepackage{subfig}

\usepackage{diagbox}
\usepackage{multirow}
\usepackage{rotating}
\usepackage[final]{listings}
\usepackage[usenames,dvipsnames,svgnames,table]{xcolor}
\lstset{inputpath=./code/}
\usepackage[notransparent]{svg}
\usepackage[final]{pdfpages}
%\usepackage{tabularray}
\usepackage{booktabs, makecell, multirow, tabularx}
% بسته‌ای برای رسم کادر
\usepackage{framed} 
% بسته‌‌ای برای چاپ شدن خودکار تعداد صفحات در صفحه «معرفی پایان‌نامه»
\usepackage{lastpage}
% بسته‌ٔ لازم برای: ۱. تغییر شماره‌گذاری صفحات پیوست. ۲. تصحیح باگ آدرس وب حاوی '%' در مراجع
\usepackage{etoolbox}
\usepackage{amsmath}

%%%%%%%%%%%%%%%%%%%%%%%%%%%%%%%%%%%%
%%% دستورات وابسته به استیل مراجع:
%% اگر از استیل‌های natbib (plainnat-fa، asa-fa، chicago-fa) استفاده می‌کنید، خط زیر را فعال و بعدی‌اش را غیرفعال کنید.
%\usepackage{natbib}
%\newcommand{\citelatin}[1]{\cite{#1}\LTRfootnote{\citeauthor*{#1}}}
%\newcommand{\citeplatin}[1]{\citep{#1}\LTRfootnote{\citeauthor*{#1}}}


%% اگر از سایر استیل‌ها استفاده می‌کنید، خط بالا را غیرفعال و خط‌های زیر را فعال کنید.
\let\citep\cite
\let\citelatin\cite
\let\citeplatin\cite
%%%%%%%%%%%%
%% بررسی حالت پیش نویس
\usepackage{ifdraft}
\ifdraft
{%
	% بسته‌ٔ ایجاد لینک‌های رنگی با امکان جهش
	\usepackage[unicode=true,pagebackref=true,
colorlinks,linkcolor=blue,citecolor=blue,final]{hyperref}
	%\usepackage{todonotes}
	\usepackage[firstpage]{draftwatermark}
	\SetWatermarkText{\ \ \ \rl{پیش‌نویس}}
	\SetWatermarkScale{1.2}
}
{ 
	\usepackage[pagebackref=false]{hyperref}
	%\usepackage[disable]{todonotes} % final without TODOs
}

\usepackage[obeyDraft]{todonotes}
\setlength{\marginparwidth}{2cm}

%%%%%%%%%%%%
%%% تصحیح باگ: اگر در مراجع، آدرس وب حاوی '%' بوده و pagebackref فعال باشد، دستورات زیر باید بیایند:
%% برای استیل‌های natbib مثل plainnat-fa، asa-fa، chicago-fa
\makeatletter
\let\ORIG@BR@@lbibitem\BR@@lbibitem
\apptocmd\ORIG@BR@@lbibitem{\endgroup}{}{}
\def\BR@@lbibitem{\begingroup\catcode`\%=12 \ORIG@BR@@lbibitem}
\makeatother
%% برای سایر استیل‌ها
\makeatletter
\let\ORIG@BR@@bibitem\BR@@bibitem
\apptocmd\ORIG@BR@@bibitem{\endgroup}{}{}
\def\BR@@bibitem{\begingroup\catcode`\%=12 \ORIG@BR@@bibitem}
\makeatother
%%%%%%%%%%%%%%%%%%%%%%%%%%%%%%%%%%%%

% بسته‌ لازم برای تنظیم سربرگ‌ها
\usepackage{fancyhdr}
%\usepackage{enumitem}
\usepackage{setspace}
% بسته‌های لازم برای نوشتن الگوریتم
\usepackage{algorithm}
\usepackage{algorithmic}
% بسته‌های لازم برای رسم بهتر جداول
\usepackage{tabulary}
\usepackage{tabularx}
\usepackage{rotating}
% بسته‌های لازم برای رسم تنظیم بهتر شکل‌ها و زیرشکل‌ها
\usepackage[export]{adjustbox}
\usepackage{subfig}
\usepackage[subfigure]{tocloft}
% بسته‌ای برای رسم نمودارها و نیز صفحه مالکیت اثر
\usepackage{tikz}
% بسته‌ای برای ظاهر شدن «مراجع» و «نمایه» در فهرست مطالب
\usepackage[nottoc]{tocbibind}
% دستورات مربوط به ایجاد نمایه
\usepackage{makeidx}
\makeindex


% بسته‌ای برای افزودن تورفتگی به ابتدای اولین پاراگراف هر بخش
\usepackage{indentfirst}

% بسته زیر باگ ناشی از فراخوانی بسته‌های زیاد را برطرف می‌کند.
\usepackage{morewrites}
%%%%%%%%%%%%%%%%%%%%%%%%%%
%khodam
\RequirePackage{zref-perpage}
\zmakeperpage{footnote}

%%% بسته ایجاد واژه‌نامه با xindy
\usepackage[xindy,toc,acronym,nonumberlist=true]{glossaries}

% فراخوانی بسته زی‌پرشین (باید آخرین بسته باشد)
\usepackage[extrafootnotefeatures, localise=on, displaymathdigits=persian]{xepersian}
\paragraphfootnotes

\makeatletter
% تعریف قلم فارسی و انگلیسی و مکان قلم‌ها
\if@irfonts
\settextfont[Path={./font/}, BoldFont={IRLotusICEE_Bold.ttf}, BoldItalicFont={IRLotusICEE_BoldIranic.ttf}, ItalicFont={IRLotusICEE_Iranic.ttf},Scale=1.2]{IRLotusICEE.ttf}
% LiberationSerif or FreeSerif as free equivalents of Times New Roman
\setlatintextfont[Path={./font/}, BoldFont={LiberationSerif-Bold.ttf}, BoldItalicFont={LiberationSerif-BoldItalic.ttf}, ItalicFont={LiberationSerif-Italic.ttf},Scale=1]{LiberationSerif-Regular.ttf}
% چنانچه می‌خواهید اعداد در فرمول‌ها، انگلیسی باشد، خط زیر را غیرفعال کنید
% و گزینهٔ displaymathdigits=persian را از خط ۱۰۹ حذف کنید.
\setdigitfont[Path={./font/}, Scale=1.2]{IRLotusICEE.ttf}
% تعریف قلم‌های فارسی و انگلیسی اضافی برای استفاده در بعضی از قسمت‌های متن
\setiranicfont[Path={./font/}, Scale=1.3]{IRLotusICEE_Iranic.ttf}				% ایرانیک، خوابیده به چپ
\setmathsfdigitfont[Path={./font/}]{IRTitr.ttf}
%\defpersianfont\titlefont[Path={./font/}, Scale=1]{IRTitr.ttf}
\defpersianfont\titlefont[Path={./font/}, Scale=1]{IRLotusICEE.ttf}
% برای تعریف یک قلم خاص عنوان لاتین، خط بعد را فعال و ویرایش کنید و خط بعد از آن را غیرفعال کنید.
% \deflatinfont\latintitlefont[Scale=1]{LiberationSerif}
\font\latintitlefont=cmssbx10 scaled 2300 %cmssbx10 scaled 2300
\else
\settextfont{XB Niloofar}
\setlatintextfont{Junicode}
% چنانچه می‌خواهید اعداد در فرمول‌ها، انگلیسی باشد، خط زیر را غیرفعال کنید
% و گزینهٔ displaymathdigits=persian را از خط ۱۰۹ حذف کنید.
\setdigitfont{XB Niloofar}
% تعریف قلم‌های فارسی و انگلیسی اضافی برای استفاده در بعضی از قسمت‌های متن
% \setmathsfdigitfont{XB Titre}
\defpersianfont\titlefont{XB Titre}
\deflatinfont\latintitlefont[Scale=1.1]{Junicode}
\fi
\makeatother

% برای استفاده از قلم نستعلیق خط بعد را فعال کنید.
\defpersianfont\nast[Path={./font/}, Scale=2]{IranNastaliq}
% فونت‌های جدید را در این جا وارد کنید.
%\defpersianfont\newfont[Path={./font/}, Scale=2]{newfont}

%%%%%%%%%%%%%%%%%%%%%%%%%%
% راستچین شدن todonotes
\presetkeys{todonotes}{align=right,textdirection=righttoleft}{}
\makeatletter
\providecommand\@dotsep{5}
\def\listtodoname{فهرست کارهای باقیمانده}
\def\listoftodos{\noindent{\Large\vspace{10mm}\textbf{\listtodoname}}\@starttoc{tdo}}
\renewcommand{\@todonotes@MissingFigureText}{شکل}
\renewcommand{\@todonotes@MissingFigureUp}{شکل}
\renewcommand{\@todonotes@MissingFigureDown}{جاافتاده}
\makeatother
% دستوری برای حذف کلمه «چکیده»
% \renewcommand{\abstractname}{}
% دستوری برای حذف کلمه «abstract»
%\renewcommand{\latinabstract}{}
% دستوری برای تغییر نام کلمه «اثبات» به «برهان»
\renewcommand\proofname{\textbf{برهان}}
% دستوری برای تغییر نام کلمه «کتاب‌نامه» به «مراجع»
\renewcommand{\bibname}{مراجع}
% دستوری برای تعریف واژه‌نامه انگلیسی به فارسی
\newcommand\persiangloss[2]{#1\dotfill\lr{#2}\\}
% دستوری برای تعریف واژه‌نامه فارسی به انگلیسی 
\newcommand\englishgloss[2]{#2\dotfill\lr{#1}\\}
% تعریف دستور جدید «\پ» برای خلاصه‌نویسی جهت نوشتن عبارت «پروژه/پایان‌نامه/رساله»
\newcommand{\پ}{پروژه/پایان‌نامه/رساله }

%\newcommand\BackSlash{\char`\\}

%%%%%%%%%%%%%%%%%%%%%%%%%%
% \SepMark{-}

% تعریف و نحوه ظاهر شدن عنوان قضیه‌ها، تعریف‌ها، مثال‌ها و ...
\theoremstyle{definition}
\newtheorem{definition}{تعریف}[section]
\theoremstyle{theorem}
\newtheorem{theorem}[definition]{قضیه}
\newtheorem{lemma}[definition]{لم}
\newtheorem{proposition}[definition]{گزاره}
\newtheorem{corollary}[definition]{نتیجه}
\newtheorem{remark}[definition]{ملاحظه}
\theoremstyle{definition}
\newtheorem{example}[definition]{مثال}

%\renewcommand{\theequation}{\thechapter-\arabic{equation}}
%\def\bibname{مراجع}
\numberwithin{algorithm}{chapter}
\def\listalgorithmname{فهرست الگوریتم‌ها}
\def\listfigurename{فهرست تصاویر}
\def\listtablename{فهرست جداول}

% دستور های لازم برای تعریف ترجمهٔ دستورات الگوریتم
\makeatletter
\renewcommand{\algorithmicrequire}{\if@RTL\textbf{ورودی:}\else\textbf{Require:}\fi}
\renewcommand{\algorithmicensure}{\if@RTL\textbf{خروجی:}\else\textbf{Ensure:}\fi}
\renewcommand{\algorithmicend}{\if@RTL\textbf{پایان}\else\textbf{end}\fi}
\renewcommand{\algorithmicif}{\if@RTL\textbf{اگر}\else\textbf{if}\fi}
\renewcommand{\algorithmicthen}{\if@RTL\textbf{آنگاه}\else\textbf{then}\fi}
\renewcommand{\algorithmicelse}{\if@RTL\textbf{وگرنه}\else\textbf{else}\fi}
\renewcommand{\algorithmicfor}{\if@RTL\textbf{برای}\else\textbf{for}\fi}
\renewcommand{\algorithmicforall}{\if@RTL\textbf{برای هر}\else\textbf{for all}\fi}
\renewcommand{\algorithmicdo}{\if@RTL\textbf{انجام بده}\else\textbf{do}\fi}
\renewcommand{\algorithmicwhile}{\if@RTL\textbf{تا زمانی که}\else\textbf{while}\fi}
\renewcommand{\algorithmicloop}{\if@RTL\textbf{تکرار کن}\else\textbf{loop}\fi}
\renewcommand{\algorithmicrepeat}{\if@RTL\textbf{تکرار کن}\else\textbf{repeat}\fi}
\renewcommand{\algorithmicuntil}{\if@RTL\textbf{تا زمانی که}\else\textbf{until}\fi}
\renewcommand{\algorithmicprint}{\if@RTL\textbf{چاپ کن}\else\textbf{print}\fi}
\renewcommand{\algorithmicreturn}{\if@RTL\textbf{بازگردان}\else\textbf{return}\fi}
\renewcommand{\algorithmicand}{\if@RTL\textbf{و}\else\textbf{and}\fi}
\renewcommand{\algorithmicor}{\if@RTL\textbf{و یا}\else\textbf{or}\fi} % TODO add better translate
\renewcommand{\algorithmicxor}{\if@RTL\textbf{یا}\else\textbf{xor}\fi} % TODO add better translate
\renewcommand{\algorithmicnot}{\if@RTL\textbf{نقیض}\else\textbf{not}\fi}
\renewcommand{\algorithmicto}{\if@RTL\textbf{تا}\else\textbf{to}\fi}
\renewcommand{\algorithmicinputs}{\if@RTL\textbf{ورودی‌ها}\else\textbf{inputs}\fi}
\renewcommand{\algorithmicoutputs}{\if@RTL\textbf{خروجی‌ها}\else\textbf{outputs}\fi}
\renewcommand{\algorithmicglobals}{\if@RTL\textbf{متغیرهای عمومی}\else\textbf{globals}\fi}
\renewcommand{\algorithmicbody}{\if@RTL\textbf{انجام بده}\else\textbf{do}\fi}
\renewcommand{\algorithmictrue}{\if@RTL\textbf{درست}\else\textbf{true}\fi}
\renewcommand{\algorithmicfalse}{\if@RTL\textbf{نادرست}\else\textbf{false}\fi}
\renewcommand{\algorithmicendif}{\algorithmicend\textbf{ شرط }\algorithmicif}
\renewcommand{\algorithmicendfor}{\algorithmicend\textbf{ حلقهٔ }\algorithmicfor}
\renewcommand{\algorithmicendwhile}{\algorithmicend\textbf{ حلقهٔ }\algorithmicwhile}
\renewcommand{\algorithmicendloop}{\algorithmicend\textbf{ حلقهٔ }\algorithmicloop}
\renewcommand{\algorithmiccomment}[1]{\{{\itshape #1}\}}
\makeatletter

%%%%%%%%%%%%%%%%%%%%%%%%%%%%
%%% دستورهایی برای سفارشی کردن سربرگ صفحات:
%\newcommand{\SetHeader}[1]{
% دستور زیر معادل با گزینه twoside است.
%\csname@twosidetrue\endcsname
\pagestyle{fancy}
%% دستورات زیر سبک صفحات fancy را تغییر می‌دهد:
% O=Odd, E=Even, L=Left, R=Right
% در صورت oneside بودن، عنوان فصل، سمت چپ ظاهر می‌شود.
\fancyhead{}
\fancyhead[OR]{\small\leftmark}
\fancyhead[ER]{\small\leftmark}
\fancyhead[OL,EL]{\thepage}
\fancyfoot{} % حذف محتوای پیش‌فرض پانویس
%\fancyhead[ER]{\small\rightmark}
%\fancyhead[OR]{\footnotesize}
%\fancyhead[EL]{\footnotesize\rightmark}
\renewcommand{\headrulewidth}{0.75pt}
% شکل‌دهی شماره و عنوان فصل در سربرگ
\renewcommand{\chaptermark}[1]{\markboth{\@chapapp~\thechapter:\ #1}{}}
\makeatletter
%\renewcommand{\rightmark}[1]{\@title}
%\makeatother
%}
% شکل‌دهی سربرگ در صفحات اولیه هر فصل
\patchcmd{\chapter}{\thispagestyle{plain}}{\thispagestyle{empty}}{}{}
%%%%%%%%%%%%%%%%%%%%%%%%%%%%
%\def\MATtextbaseline{1.5}
%\renewcommand{\baselinestretch}{\MATtextbaseline}
\doublespacing
%%%%%%%%%%%%%%%%%%%%%%%%%%%%%
% دستوراتی برای اضافه کردن کلمه «فصل» در فهرست مطالب

\newlength\mylenprt
\newlength\mylenchp
\newlength\mylenapp

\renewcommand\cftpartpresnum{\partname~}
\renewcommand\cftchappresnum{\chaptername~}
\renewcommand\cftchapaftersnum{:}

\settowidth\mylenprt{\cftpartfont\cftpartpresnum\cftpartaftersnum}
\settowidth\mylenchp{\cftchapfont\cftchappresnum\cftchapaftersnum}
\settowidth\mylenapp{\cftchapfont\appendixname~\cftchapaftersnum}
\addtolength\mylenprt{\cftpartnumwidth}
\addtolength\mylenchp{\cftchapnumwidth}
\addtolength\mylenapp{\cftchapnumwidth}

\setlength\cftpartnumwidth{\mylenprt}
\setlength\cftchapnumwidth{\mylenchp}	

\makeatletter
{\def\thebibliography#1{\chapter*{\refname\@mkboth
   {\uppercase{\refname}}{\uppercase{\refname}}}\list
   {[\arabic{enumi}]}{\settowidth\labelwidth{[#1]}
   \rightmargin\labelwidth
   \advance\rightmargin\labelsep
   \advance\rightmargin\bibindent
   \itemindent -\bibindent

   \listparindent \itemindent
   \parsep \z@
   \usecounter{enumi}}
   \def\newblock{}
   \sloppy
   \sfcode`\.=1000\relax}}
   
%اگر مایلید در شماره گذاری حرفی و ابجد به جای آ از الف استفاده شود دستورات زیر را فعال کنید.   
%\def\@Abjad#1{%
%  \ifcase#1\or الف\or ب\or ج\or د%
%           \or هـ\or و\or ز\or ح\or ط%
%           \or ی\or ک\or ل\or م\or ن%
%           \or س\or ع\or ف\or ص%
%           \or ق\or ر\or ش\or ت\or ث%
%            \or خ\or ذ\or ض\or ظ\or غ%
%            \else\@ctrerr\fi}
%
% \def\abj@num@i#1{%
%   \ifcase#1\or الف\or ب\or ج\or د%
%            \or هـ‍\or و\or ز\or ح\or ط\fi

%   \ifnum#1=\z@\abjad@zero\fi}   
%  
%   \def\@harfi#1{\ifcase#1\or الف\or ب\or پ\or ت\or ث\or

% ج\or چ\or ح\or خ\or د\or ذ\or ر\or ز\or ژ\or س\or ش\or ص\or ض\or ط\or ظ\or ع\or غ\or

% ف\or ق\or ک\or گ\or ل\or م\or ن\or و\or ه\or ی\else\@ctrerr\fi}

%
\makeatother

%%% امکان درج کد در سند
% در این قسمت رنگ، قلم و قالب‌بندی قسمت‌های مختلف یک کد تعیین می‌شود. 
\lstdefinestyle{myStyle}{
	basicstyle=\ttfamily, % whole listing /w verbatim font
	keywordstyle=\color{blue}\bfseries, % bold black keywords
	identifierstyle=, % nothing happens
	commentstyle=\color{LimeGreen}, % green comments
	stringstyle=\ttfamily\color{red}, % red typewriter font for strings
	showstringspaces=false % no special string spaces
	breaklines=true,
	breakatwhitespace=false,
	numbers=right, % line number formats
	numberstyle=\footnotesize\lr,
	numbersep=-10pt,
	frame=single,
	captionpos=b,
	captiondirection=RTL
}
\lstset{style=myStyle} % command to set default style
\def\lstlistingname{\rl{برنامهٔ}}
\def\lstlistlistingname{\rl{فهرست برنامه‌ها}}


% for numbering subsubsections
\setcounter{secnumdepth}{3}
%to include subsubsections in the table of contents
\setcounter{tocdepth}{3}

\makeatletter
\renewcommand{\p@subfigure}{\thefigure.}
\makeatother
