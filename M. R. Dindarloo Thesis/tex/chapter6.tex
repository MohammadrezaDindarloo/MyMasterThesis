% !TeX root=../main.tex


\chapter{نتیجه‌گیری و پیشنهادات برای آینده}
%\thispagestyle{empty} 
\section{نتیجه‌گیری}

در این پایان‌نامه، در فصل اول و دوم، ابتدا مروری بر مسائل بهینه‌سازی در ادبیات ربات‌ها صورت گرفت. مشاهده شد که این مسائل با توجه به اهداف کاربردی ربات‌ها، می‌توانند دارای ماهیت‌های متفاوت، اما ضروری باشند. با نگاهی اجمالی به ادبیات موضوع، مشخص شد که این مسائل بهینه‌سازی می‌توانند در راستای فرمول‌بندی یک مسئله کالیبراسیون دینامیکی یا سینماتیکی، مکان‌یابی، ردیابی، نقشه‌برداری، یا ترکیبی از این موارد باشند.

در فصل سوم، با آغاز فرمول‌بندی‌های مرسوم برای ایجاد یک مسئله کالیبراسیون و مکان‌یابی همزمان، به نقطه ضعف‌هایی برخوردیم که نه تنها ایجاد فرمول‌بندی مورد نظر را دشوار می‌کردند، بلکه در حل چنین مسائل پیچیده‌ای به صورت زمان واقعی با تعداد بسیار زیادی داده، ناتوان بودند. از سوی دیگر، انعطاف‌پذیری الگوریتم برای افزودن قیود جدید به مسئله بدون تغییر در فرمول‌بندی، برای ما بسیار حائز اهمیت بود، زیرا مقیدسازی سینماتیکی می‌تواند دقت مکان‌یابی را به‌طور قابل توجهی افزایش دهد. با در نظر گرفتن تمامی این نکات و استفاده از آخرین روش‌ها و حل‌کننده‌های موجود در رباتیک، الگوریتم‌های گراف مبنا را به‌عنوان راه‌حلی مناسب در فضای حل احتمالاتی این مسائل انتخاب کردیم. بدین ترتیب، از مکان‌یابی یک ربات شروع کردیم و با توسعه گراف عامل مناسب، مسائل کالیبراسیون در سطوح مختلف و همچنین قیود ضروری، به گراف توسعه‌یافته متصل گردید. در انتهای این فصل، گراف عامل جامع پیشنهادی خود برای حل مسئله کالیبراسیون و مکان‌یابی ربات‌ها، با تمرکز بر رفع مشکلات روش‌های مرسوم، معرفی شد.

در ادامه، برای صحت‌سنجی روش پیشنهادی، از میان ربات‌های موجود، ربات کابلی آسان‌نصب انتخاب گردید. علت این انتخاب، اهمیت انجام همزمان فرآیند کالیبراسیون و مکان‌یابی این ربات‌ها در کوتاه‌ترین زمان ممکن برای ایجاد مفهوم یک ربات آسان‌نصب بود. بدین ترتیب، در فصل چهارم، با استفاده از ربات کابلی صلب توسعه‌یافته در مجموعه آزمایشگاهی ارس، الگوریتم پیشنهادی مورد بررسی قرار گرفت. در این پیاده‌سازی، اهمیت قیود سینماتیکی در بهبود نتایج مکان‌یابی یکی از چالش‌های مورد بررسی بود. علاوه بر این، کالیبراسیون سینماتیکی به‌صورت خودکار برای ربات، در حالی که مکان‌یابی در حال انجام بود، صورت گرفت و نتایج قابل قبولی از نظر سرعت الگوریتم و توانایی آن در برآورد اهداف ما به‌دست آمد.

در نهایت، برای محک‌زدن روش پیشنهادی و همچنین نمایش انعطاف‌پذیری آن، گامی جدید در ادبیات ربات‌های کابلی برداشته شد. فصل پنجم، به حل همان مسئله ربات‌های کابلی که در فصل چهارم مطرح شده بود، می‌پردازد. اما آنچه به‌عنوان نوآوری این فصل و نقطه پایانی کار معرفی گردید، افزودن یکی از پیچیده‌ترین قیود دینامیکی به مسئله ربات‌های کابلی برای مدل‌سازی واقعی‌تر کابل و گسترش روش موردنظر برای همه ربات‌های کابلی بدون نگرانی از ابعاد ربات و مسئله خم‌شدگی کابل‌ها بود. نتایج کالیبراسیون برای دو دسته ربات کوچک‌مقیاس و بزرگ‌مقیاس در یک شبیه‌ساز اجزای محدود مورد بررسی قرار گرفت. بررسی  و مقایسه این نتایج با به‌کارگیری روش‌های فصل چهارم، علاوه بر بهبود قابل توجه، نقطه عطفی برای تحقیقات در زمینه ترکیب سنسورها و کاربرد آن‌ها در مسائل ربات‌های کابلی بود.

\section{پیشنهادات برای آینده}

در مسیر انجام این پایان‌نامه، موانع متعددی به وجود آمد که گذر از آنها افق‌های جدیدی را برای نویسنده نمایان کرد و چشم‌اندازهای تازه‌ای را در جهت ایجاد نقطه‌ای مفید در این حوزه مشخص ساخت. علاوه بر نظراتی که در راستای توسعه الگوریتم ایجاد شده است، نظرات متعددی نیز در بخش پیاده‌سازی به وجود آمدند، چرا که پیچیدگی حل چنین مسائلی در ربات‌های کابلی باعث شده است تحقیقات مفید و کارآمدی در این زمینه کمتر صورت گیرد. برخی از این پیشنهادات در گروه در حال پیش‌برد می‌باشند و برخی به عنوان کار‌های بعدی مدنظر خواهند بود. 

\begin{itemize}
	
\item {\textbf{ایجاد گراف‌های عامل با استفاده از کتابخانه :SymForce} 
در این پایان‌نامه، از کتابخانه GTSAM برای ایجاد گراف‌های عامل استفاده شده است. یکی از چالش‌برانگیزترین بخش‌های حل مسئله، محاسبه ژاکوبین مدل بود. نرم‌افزارهای متعددی مانند Maple و MATLAB، و همچنین کتابخانه‌های مختلفی همچون SymPy در پایتون و AutoDiff در C++ مورد استفاده قرار گرفتند. با این حال، حجم بسیار بالای معادلات دینامیکی کابل‌ها، مانع از به دست آوردن پاسخ در این بسترها شد. 

در نهایت، با استفاده از کتابخانه SymForce که مشتق‌گیری‌ها را با روش‌های مشتق‌گیری خودکار
\footnote{\lr{Automatic Differentiation}} 
 انجام می‌دهد، موفق شدیم تمامی مشتق‌های مورد نیاز را در زمانی کمتر از ۳۰ ثانیه محاسبه کنیم. 
پیشنهاد ما در این بخش، نه تنها استفاده از این کتابخانه برای به دست آوردن ژاکوبین‌های مورد نیاز است، بلکه پیشنهاد می‌شود که گراف‌های عامل توسعه‌یافته نیز با استفاده از روش‌هایی که به تازگی توسط این کتابخانه ارائه‌شده، ایجاد شوند. ما معتقدیم که این رویکرد می‌تواند فرآیند ایجاد گراف‌های عامل را به طور قابل توجهی ساده‌تر کند.}



\item \textbf{پیاده‌سازی ماژول بهینه‌ساز سینماتیک-ایستا ربات کابلی توسعه‌یافته در راستای بهبود دقت ردیابی:} 
در زمینه ربات‌های کابلی، ادغام داده‌های سینماتیکی می‌تواند به طور قابل توجهی در بهبود نتایج ردیابی کمک کند. در فصل چهارم این پایان‌نامه، بخشی از نتایج این موضوع را اثبات کردند. وقتی به حوزه ربات‌های مقیاس‌بزرگ وارد می‌شویم، در نظر گرفتن دینامیک کابل، همانطور که در فصل پنجم مشاهده شد، بسیار حائز اهمیت می‌شود. 

در همین راستا، مقاله \cite{allak2022kinematics} در سال‌های اخیر سعی در توسعه روشی برای ردیابی ربات‌های کابلی با در نظر گرفتن جرم کابل، به صورت منابع باز داشته است. حل‌کننده‌ی پیشنهادی آنها، CERES، علیرغم توانایی بالای خود در حل مسئله، عیوب روش‌های مرسوم را دارد. 

در قسمتی از کارهای انجام شده توسط ما، بررسی حل‌کننده مورد استفاده توسط این مقاله برای ربات‌های مقیاس‌بزرگ بوده است. در حالی که نتایج حل بهینه‌سازی توسط گراف عامل پیشنهادی ما و حل پیشنهادی این مقاله برای ربات‌های کوچک‌مقیاس مشابه بود، افزایش ابعاد ربات باعث واگرایی در الگوریتم ارائه شده در مقاله می‌شود. این در حالی است که گراف عامل پیشنهادی به خوبی برای ربات‌های مقیاس‌بزرگ نتیجه را دنبال می‌کند.

علاوه بر این، در این مقاله برای ادغام نتایج بهینه‌سازی سینماتیکی با حسگر‌های مختلف از فیلتر کالمن استفاده شده است که خود نیاز به ایجاد ساختاری جدا و اتصال این دو، و همچنین ادغام مناسب داده‌های حسگر اینرسی-بینایی دارد. روش پیشنهادی ما استفاده از ماژول طراحی‌شده است. استفاده از این ماژول نه‌تنها منجر به همگرایی در ربات‌های با ابعاد بزرگ‌تر می‌شود، بلکه ادغام حسگرها برای قسمت ردیابی بسیار راحت‌تر و منجر به حل دقیق‌تری می‌شود. افزون بر این موارد، گره‌هایی که به تازگی برای ادغام حسگرهای اینرسی در این حوزه معرفی شده‌اند، بسیار ارزشمند خواهند بود. لازم به ذکر است که این ماژول بااستفاده از معادلات ارائه شده در فصل پنجم ایجاد شده است که یکی از ویژگی‌های مهم آن حذف نیاز به حسگر نیروِ، با استفاده از قید‌های سینماتیک وارون می‌باشد. 

\item \textbf{کالیبراسیون حسگر UWB با استفاده از گراف عامل پیشنهادی:} 
حسگرهای UWB به دلیل ویژگی‌های خاص خود در کاربردهای مختلف تعیین مکان در حوزه‌های نقشه‌برداری و رباتیک مورد استفاده قرار می‌گیرند. این حسگرها را می‌توان در دو سطح کالیبراسیون بررسی کرد.

سطح اول، حذف بایاس‌های اولیه است. این نوع کالیبراسیون، مشابه روش ارائه‌شده در فصل چهارم است. به عبارتی، اگر در گراف عامل فصل چهارم حسگرهای نیرو حذف شوند و مقادیر اندازه‌گیری حسگرهای UWB جایگزین مقادیر اندازه‌گیری انکودر شوند، نه‌تنها قادر به تعیین این آفست‌ها خواهیم بود، بلکه مکان اتصال انکرها می‌تواند به عنوان مکان پولی‌ها در نظر گرفته شده و یک مسئله بهینه‌سازی ایجاد شود.

سطح دوم کالیبراسیون این حسگرها، در نظر گرفتن این بایاس اولیه به عنوان پارامتری متغیر با زمان است. به عبارتی، گراف عامل فصل چهارم را در حالتی ایجاد کنیم که بایاس حسگر UWB همانند متغیر سینماتیکی مجموعه، همانطور که در انتهای فصل سوم بررسی شد، در حال تغییر باشد. بدین ترتیب، کالیبراسیون با دقت بالایی به‌صورت زمان واقعی انجام می‌شود. انجام این سطح کالیبراسیون نسبت به سطح اول، حذف بایاس‌های متغیر با زمان سیستم می‌باشد که در سطح اول از آنها صرف‌نظر می‌شود. این پیشنهاد در مرحله اجرا توسط اعضای تیم آزمایشگاهی ارس می‌باشد.


\item \textbf{پیاده‌سازی الگوریتم گراف پیشنهادی برای ربات‌های واقعی مقیاس بزرگ:} 
تمامی موارد بررسی‌شده در فصل چهارم بر روی ربات کابلی واقعی توسعه‌یافته در تیم آزمایشگاهی ارس پیاده‌سازی شده‌اند. در فصل پنجم، این پیاده‌سازی واقعی نیازمند یک ربات در ابعاد ورزشگاه بزرگ است که با توجه به امکانات فعلی، پیاده‌سازی واقعی برای ما مقدور نبوده است. همچنین هیچ‌گونه دیتاست‌ای از چنین ربات با ابعاد بزرگ در اینترنت نیز در دسترس نیست. به همین دلیل، از داده‌های شبیه‌سازی نرم‌افزار RecurDyn استفاده کردیم. البته که نتایج خود، نویزهای شدیدی که برای نزدیک شدن به یک پیاده سازی واقعی در نظر گرفته ایم. پیشنهاد ما پیاده‌سازی مجدد روش‌های پیشنهادی در فصل پنجم بر روی یک ربات واقعی است. تمامی کدهای مورد نیاز برای این پیاده‌سازی به‌صورت منابع باز در Github قرار داده شده است.

\item \textbf{ایجاد شبیه‌ساز مناسب ربات کابلی با استفاده از توابع آماده شده:} 
ما برای توسعه یک ربات کابلی با ابعاد بزرگ نرم‌افزار های متعددی مورد بررسی قرار دادیم که از انجام این شبیه‌سازی باز ماندند. بهترین نرم‌افزاری که با محاسبات بسیار طولانی و زمان‌بر ما را به جواب رساندند، نرم‌افزار RecurDyn بوده است. پیشنهاد ما توسعه یک محیط گرافیکی مناسب برای معرفی یک نرم‌افزار شبیه‌ساز کابل با در نظر گرفتم جرم کابل، با استفاده از توابع سینماتیکی و مکان‌یابی معرفی شده در فصل‌های اخیر می‌باشد. چنین نرم‌افزاری که دارای سرعت بالا باشد طبق آخرین تحقیقات ما توسعه نیافته است.

\item \textbf{ایجاد یک شبیه‌ساز مناسب ربات کابلی با استفاده از توابع آماده‌شده:}
ما برای توسعه یک ربات کابلی با ابعاد بزرگ، نرم‌افزارهای متعددی را مورد بررسی قرار دادیم، اما اغلب آن‌ها نتوانستند شبیه‌سازی مورد نظر ما را انجام دهند. بهترین نرم‌افزاری که با وجود محاسبات طولانی و زمان‌بر، توانست به نتایج مطلوب برسد، نرم‌افزار RecurDyn بود. پیشنهاد ما توسعه یک محیط گرافیکی مناسب برای معرفی یک نرم‌افزار شبیه‌ساز کابل با در نظر گرفتن جرم کابل، و استفاده از توابع سینماتیکی و مکان‌یابی معرفی‌شده در فصل‌های اخیر است. تاکنون، طبق آخرین تحقیقات ما، چنین نرم‌افزاری که دارای سرعت بالا باشد، توسعه نیافته است.

\item \textbf{توسعه فرمول‌بندی ریاضی بیان شده به ریاضیات اجزای محدود به جای مدل‌های اسیتا معرفی‌شده:}
مدل توسعه‌یافته برای کابل‌ها در این پایان‌نامه، مبتنی بر مدل‌های ریاضی است که در حالت ایستا ربات و ثابت بودن کابل‌ها ایجاد شده‌اند. ادغام این الگوریتم با بیان ریاضی فعلی و قیدهای دینامیکی در ربات‌های کابلی به‌صورت مستقیم امکان‌پذیر نخواهد بود. یکی از پیشنهادهای مناسب در امتداد این کار، تغییر فرمول‌بندی به مدلی جامع‌تر است که قابلیت ادغام با تمامی قیدهای دینامیکی ربات را داشته باشد. این تغییر می‌تواند به استفاده مؤثرتر از ریاضیات اجزای محدود به جای مدل‌های ایستا منجر شود که راه‌حلی برای ما خواهد بود.

\item \textbf{بررسی مسیر مناسب برای جمع‌آوری داده کافی برای کالیبراسیون دقیق:}
همان‌طور که در نتایج مشاهده شد، برای ربات کابلی با مقیاس بزرگ، افزایش تعداد داده‌ها منجر به نتیجه‌ای بهتر در کالیبراسیون خواهد شد، چرا که پارامترهای غیرخطی بسیاری در مدل درگیر هستند. مسیری که در آزمایشات ما طی شد، یک مسیر تصادفی بود که شامل قسمت‌های مختلفی از فضای کاری ربات می‌شد. پیشنهاد ما بررسی این موضوع برای یافتن مسیری بهینه، و نه تصادفی، جهت تسریع فرآیند کالیبراسیون است. این موضوع می‌تواند به بهبود دقت و کاهش زمان کالیبراسیون کمک شایانی کند.


\item \textbf{ترکیب حسگر‌ها و ماژول‌های آماده مختلف:}
در نهایت، این پایان‌نامه بستری کامل برای حل مسائل کالیبراسیون و مکان‌یابی ربات‌ها ایجاد کرده است. بررسی تحقیقات تکمیلی جهت بهبود نتایج و استفاده از ماژول‌های معرفی شده در حوزه رباتیک و به‌ویژه شاخه SLAM می‌تواند نقطه عطف تازه‌ای بین روش‌های موجود در ربات‌های کابلی و روش‌های موجود در دیگر ربات‌ها مانند ربات‌های خودران باشد. این نکته به‌عنوان پیشنهاد ما مطرح می‌شود زیرا اولین هدف ما، حرکت به سمت الگوریتمی منعطف بوده است تا تحقیقات در این حوزه‌ها بیشتر به هم متصل شوند.


\end{itemize}
