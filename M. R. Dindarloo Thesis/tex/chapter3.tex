% !TeX root=../main.tex
\chapter{موقعیت‌یابی و کالیبراسیون به صورت همزمان یک ربات کابلی با در نظر گرفتن کابل‌ها به صورت جسم صلب}

\section{مقدمه}
در این فصل ابتدا نگاهی اجمالی بر حل مسئله سینماتیک یک ربات چهار کابلی فروتحریک معلق خواهیم داشت. سپس با استفاده از معادلاتی که در سینماتیک ربات استخراج شده شده است  

در این فصل پایه و اساس کارهای روح الله باید گفته شود.



همانطور که در فصل قبل ذکر شد، اگرچه سنسورهای فضای مفصل سریع و ارزان هستند، اما زمانی که از آنها برای اندازه‌گیری مقادیر مجری نهایی استفاده می‌شود، دقت مدل  سینماتیکی برای تعیین دقت قابل دستیابی بسیار مهم است. 


علاوه بر این، در زمینه تلفیق و ترکیب اندازه‌گیری‌ها، هم‌ثبت داده‌ها \cite{pott2013cable} اولین گام اساسی است. به عبارت دیگر، سنسورها باید اندازه‌گیری‌های خود را در یک چارچوب مختصات یکپارچه ارائه دهند. اهمیت هم‌ثبت به دلیل فرض اساسی نویز گاوسی با میانگین صفر در الگوریتم‌های تلفیق داده‌ها است. نکته دوم که باید به آن توجه داشت این است که برای روبات‌های قابل استقرار، الگوریتم کالیبراسیون پیشنهادی نباید به سنسورهای گران  قیمت و سخت برای نگهداری نیاز داشته باشد. علاوه بر این، فرآیند کالیبراسیون باید به اندازه‌ای ساده باشد که اجرای آن در مکان‌های مختلف آسان باشد. بنابراین با اینکه کالیبراسیون موضوعی است که بسیاری از پژوهشگران به آن علاقه‌مند هستند، اما مفهوم بهره‌گیری از چندین سنسور برای بهبود نتایج کمتر مورد توجه قرار گرفته است. علاوه بر این، الگوریتم کالیبراسیون یکپارچه و قابل گسترش در ادبیات برای روبات‌های کابلی وجود ندارد. بنابراین در این فصل، رویکردی چند سنسوره برای توسعه الگوریتم کالیبراسیون آگاه به نیرو برای روبات‌های کابلی ارائه شده است. علاوه بر این، یک سیستم کالیبراسیون یکپارچه بر اساس گراف‌های فاکتور و کتابخانه GTSAM پیشنهاد شده است. خواننده می‌تواند برای جزئیات بیشتر در مورد گراف‌های فاکتور و کاربردهای آنها در روباتیک به پیوست A مراجعه کند.
 \lr{dsdfsd}
یسبسیبس

