% !TeX root=../main.tex
\chapter{پیاده سازی رویکرد مبتنی بر گراف جهت مکان‌یابی و کالیبراسیون همزمان برای ربات‌های کابلی خم‌شده}
%\thispagestyle{empty} 
این پایان‌نامه به بررسی دقیق رویکردهای مختلف در حوزه رباتیک برای فرمول‌بندی ریاضی یک مسئله بهینه‌سازی مقید پرداخته است. از تحلیل مزایا و معایب روش‌های مرسوم تا رویکردهای جدید مبتنی بر گراف، روشی جامع برای فرمول‌بندی و حل مسئله بهینه‌سازی کالیبراسیون و مکان‌یابی همزمان ربات‌ها ارائه شده است. در فصل گذشته، عملکرد این روش بر روی یک ربات کابلی مقید که کابل‌های آن به صورت صلب در نظر گرفته شده بود، ارزیابی شد. پس از معرفی فرمول‌بندی سینماتیکی، گراف مربوط به آن ساخته و نتایج پیاده‌سازی بررسی شد. آنچه از ابتدای این پایان‌نامه به عنوان هدفی مهم معرفی گردید، ماژولاریتی و انعطاف‌پذیری روش، با قیدهای متفاوت و گسترده بود. انتخاب ربات کابلی به عنوان مورد مورد مطالعه، نیز به دلیل امکان پیاده‌سازی و ارزیابی همین هدف بوده است.

در این فصل، بدون تغییر در فرمول‌بندی‌های ارائه شده در فصل قبل، به مسئله قیدهای دینامیکی کابل افزوده خواهد شد. علیرغم پیچیدگی این مدل‌ها، حل‌کننده همچنان با دقت و سرعت بالا به نتایج مطلوب دست خواهد یافت. تاکنون تحقیقات بسیاری بر مدل‌سازی کابل‌های خم شده انجام شده است که نتایج دقیقی به دست داده‌اند. ما نیز برای حل مسئله کالیبراسیون و مکان‌یابی نیازمند افزودن این قیدها به مسئله هستیم. با این حال، پیچیدگی‌های این مدل‌ها باعث شده است که در برخی کارهای اخیر به جای حل مستقیم مسئله با این معادلات، از شبکه‌های عمیق استفاده شود که به دلیل مشکلات خاص خود، دقت و اطمینان کافی ندارند.

روش ما برای حل این چالش، استفاده از همان مقیدسازی‌هایی است که برای کابل‌های صلب انجام شده بود. در پایان، با حل این مسئله، مزایای این رویکرد را بار دیگر خواهیم دید؛ رویکردی که با دقت و قدرت بالا، مسئله کالیبراسیون و مکان‌یابی همزمان ربات‌ها را، حتی در شرایطی که کابل‌ها صلب نیستند، به نحوی که حل آنها در روش‌های مرسوم دشوار است، به سرانجام می‌رساند.



\section{نمادها و تعاریف} \label{subsec:Assm}
این فصل یک ربات موازی کابلی معلق با شش درجه آزادی ($m=6$) و چهار کابل فعال ($n=4$) را مورد بررسی قرار می‌دهد. از آنجایی که $m>n$، این ربات فروتحریک\footnote{underactuated}
 است و یک ساختار نامقید\footnote{under-constrained}
تشکیل می‌دهد~\cite{ida2021natural}. 
شکل~\ref{fig:frame_robot} ساختار این ربات را نشان می‌دهد که برای وضوح بیشتر تنها یک کابل در آن نمایش داده شده است. 
دستگاه‌مختصات $\mathcal{L}$ به بدنه متحرک ربات متصل است، در حالی که دستگاه‌مختصات جهانی $\mathcal{G}$ به طور ثابت به پایه ربات متصل شده است. مکان پنجه ربات نسبت به دستگاه‌مختصات جهانی با
 $(\bm{p},\bm{R}) \in SE(3)$
نشان داده می‌شود، که در آن
 $\bm{p} \in \mathbb{R}^3$
بردار انتقال از $\mathcal{G}$ به $\mathcal{L}$ است و
 $\bm{R} \in SO(3)$
جهت‌گیری $\mathcal{L}$ نسبت به $\mathcal{G}$ است. کابل $i$ از پایه ربات در نقطه $\bm{p}_{a_i}$ که در دستگاه‌مختصات جهانی تعریف شده است، جدا می‌شود و به پنجه ربات در نقطه $\bm{p}_{b_i}$ که در دستگاه‌مختصات بدنه محلی بیان شده است، متصل می‌شود.

\begin{figure}[t]
	\centering
	\includegraphics[width=0.6\textwidth]{img/robot_frame.pdf}
	\caption{دیاگرام پنجه ربات متصل به یک کابل  خم‌شده}
	\label{fig:frame_robot}
\end{figure}

ما تغییر شکل کابل را در یک صفحه عمودی دو بعدی $\Psi$ مدل‌سازی می‌کنیم که پولی $\bm{p}_{a_i}$ و نقطه اتصال $\bm{p}_{b_i}$ در پنجه ربات را در بر می‌گیرد. دستگاه‌مختصات
$\mathcal{P}$
روی این صفحه در نقطه $\bm{p}_{a_i}$ قرار دارد و با بردارهای واحد $\hat{\bm{s}}_z$ که موازی با محور $z$ جهانی است و $\hat{\bm{s}}_c$ که در جهت کابل بر روی صفحه $x_{\mathcal{P}}-y_{\mathcal{P}}$  دستگاه‌مختصات $\mathcal{P}$ قرار دارد، تعریف می‌شود.
به طور خاص،  $\hat{\bm{s}}_c = \frac{\bm{b}_{xy_i} - \bm{a}_{xy_i}}{\|\bm{b}_{xy_i} - \bm{a}_{xy_i}\|}$، که در آن $\bm{b}_{xy_i}$ و $\bm{a}_{xy_i}$ به ترتیب اجزای $x-y$ بردارهای $\bm{b}_{i}$ و $\bm{a}_{i}$ در دستگاه‌مختصات جهانی هستند.

\section{معادلات مدل کابل خم‌شده} \label{seq:modeling}

معادلات زنجیره‌ای اثر خم شدن کابل غیرقابل ارتجاع با جرم غیر قابل اغماض را همانطور که در~\cite{pott2013cable} توصیف شده است، به صورت زیر است:

\begin{equation} \label{eq:z(cx)}
	z_i(x_c) = \frac{f_{h,i}}{g_c} \cdot \left( \cosh \left( \frac{g_c}{f_{h,i}} \cdot (x_c + C_{1,i}) \right) - C_{2,i} \right)
\end{equation} 

در این معادله، شکل خم شدن کابل با تابع $z_i(x_c)$ تعریف شده است. ثابت‌های زنجیره‌ای، $C_{1,i}$ و $C_{2,i}$ با توجه به شرایط مرزی نقطه انتهایی $z_i(0)=(\bm{p}_{a})_z$ و $z'_i(L_i)=-\frac{f_v}{f_h}$ تعیین می‌شوند، که در آن $z'_i(L_i)$ شیب معادله (\ref{eq:z(cx)}) در \(x_c = L_i = \|(\bm{b})_{xy_i} - (\bm{a})_{xy_i}\|\) است و به صورت زیر دارای ‌حل بسته هستند:

\begin{equation}  \label{eq:C1}
	C_{1,i} = \frac{f_{h,i}}{g_c} \cdot \operatorname{asinh} \left( \frac{-f_{v,i}}{f_{h,i}} \right) - L_i  
\end{equation}

\begin{equation}  \label{eq:C2}
	C_{2,i} = \cosh \left(C_{1,i} \cdot \frac{g_c}{f_{h,i}} \right) - \frac{g_c}{f_{h,i}} \cdot (\bm{p}_a)_z
\end{equation}

همانطور که در شکل~\ref{fig:frame_robot} نشان داده شده است، $f_{h,i}$ و $f_{v,i}$ اجزای افقی و عمودی نیروی کابل $\mathbf{f}_{c_i}$ در دستگاه‌مختصات $\mathcal{P}$ هستند. علاوه بر این، $g_c=g.\rho_c$ است که در آن $g$ و $\rho_c$ به ترتیب شتاب گرانشی و جرم در واحد طول کابل هستند. در نهایت، طول منحنی معادله (\ref{eq:z(cx)}) به عنوان طول واقعی کابل $l_{C,i}$ تعریف می‌شود و به صورت زیر محاسبه می‌شود:

\begin{equation}  \label{eq:lcat}
	\begin{aligned}
		l_{C,i} &= \int_0^{L_i} \sqrt{1 + \left(\frac{dz}{dx}\right)^2} \, dx \\ &= \frac{f_{h,i}}{g_c} \cdot \left( \sinh \left( \frac{g_c}{f_{h,i}} (L_i + C_{1,i}) \right) - \sinh \left( \frac{g_c}{f_{h,i}} \cdot C_{1,i} \right) \right) 
	\end{aligned}
\end{equation}


\section{سینماتیک ربات}   
تحلیل سینماتیکی یک ربات موازی کابلی فروتحریک شامل هر دو معادلات هندسی و حالت ایستای آن است، که به طور معروف به تحلیل سینماتیک-ایستا معروف است~\cite{carricato2013direct}. نیروی پیچشی\footnote{wrench}
 پنجه ربات
 $\mathbf{w}_{ee} \in \mathbb{R}^6$ 
به نیروهای کابل $\mathbf{f}_c \in \mathbb{R}^4 $ از طریق ماتریس ژاکوبی $\bm{J} \in \mathbb{R}^{4\times6}$ مرتبط می‌شود:
\begin{equation} \label{eq:static}
	\mathbf{w}_{ee} = \bm{J}^T \mathbf{f}_{c}  
\end{equation}


در این اینجا، فرض می‌کنیم که $\mathbf{w}_{ee}$ تنها توسط گرانش ایجاد شده است و به صورت زیر محاسبه می‌شود:
\begin{equation} \label{eq:example}
	\mathbf{w}_{ee} = m_{e}g \begin{bmatrix} \hat{\bm{s}}_z \\ \bm{b}_{\text{com}} \times \hat{\bm{s}}_z \end{bmatrix}
\end{equation}

که در آن $m_e$ جرم پنجه ربات و $\bm{b}_{com}$ جابجایی بین مبدا دستگاه‌مختصات $\mathcal{L}$ و مرکز جرم (CoM) انتهای ربات است. 
برای پیوند دادن این معادلات حالت ایستا با مدل زنجیره‌ای در معادله \eqref{eq:lcat}، هر جزء نیروی کابل به عنوان یک جفت افقی و عمودی نمایش داده می‌شود:  
\begin{equation}
	\mathbf{f}_{c_{i}}= \begin{bmatrix} f_{h,i} & f_{v,i} \end{bmatrix} ^T 
\end{equation}

برای هر $\mathbf{f}_{c_{i}}$، ستون $i^{ام}$ مربوطه از ماتریس ژاکوبی $\bm{J}^T$ به صورت زیر بیان می‌شود:
\begin{equation}
	\bm{J}^T_i= \begin{bmatrix} -\hat{\bm{s}}_{c,i} & \hat{\bm{s}}_{z} \\ -\bm{R}\bm{b}_{i} \times \hat{\bm{s}}_{c,i} & \bm{R}\bm{b}_{i} \times \hat{\bm{s}}_{z}  \end{bmatrix}
\end{equation}

که در آن، ماتریس چرخش $\bm{R}$، بردارهای واحد $\hat{\bm{s}}_{c,i}, \hat{\bm{s}}_{z}$، و بردار اتصال پنجه $\bm{b}_i$ در مختصات محلی، در بخش
\ref{subsec:Assm} 
تعریف شده‌اند. توجه داشته باشید که معادلات حالت ایستا در 
\ref{eq:static}
 یک مسئله نامعین\footnote{underdetermined}
است که در آن تعداد معادلات کمتر از تعداد متغیرها است. همانطور که در \cite{allak2022kinematics} پیشنهاد شده است، ما تمام نیروهای کابل را بر اساس یک کابل مرجع بیان می‌کنیم. همانطور که در بخش~\ref{sec:calib_factor} مشاهده خواهد شد، این انتخاب، تعداد حسگرهای نیرو مورد نیاز را به تنها یک عدد کاهش می‌دهد که برای ما نیز از اهمیت بالایی برخودار است. همانطور که در~\cite{borgstrom2009nims, allak2022kinematics} ارائه شده است، با تقسیم ماتریس ژاکوبین و بردار نیرو، می‌توان
 \ref{eq:static} 
 را به صورت زیر نوشت:
\begin{equation} \label{eq:wrench-jac-force}
	\mathbf{w}_{ee} = \begin{bmatrix} \bm{J}^T_{\text{ref}} & \bm{J}^T_{\text{res}} \end{bmatrix} \cdot \begin{bmatrix} \mathbf{f}_{c_{\text{ref}}} \\ \mathbf{f}_{c_{\text{res}}} \end{bmatrix} 
\end{equation} 

که نتیجه می‌دهد:
\begin{equation} \label{eq:jacobian-distribution}
	\mathbf{w}_{ee} = \bm{J}_{\text{ref}}^T \cdot \mathbf{f}_{c_{\text{ref}}} + \bm{J}_{\text{res}}^T \cdot \mathbf{f}_{c_{\text{res}}} 
\end{equation}

که در آن، $\bm{J}^T_{\text{ref}}$ نمایانگر یک زیرماتریس $6\times2$ شامل دو ستون اول $\bm{J}^T$ است که مربوط به نیروی کابل مرجع $\mathbf{f}_{c_{\text{ref}}}$ است. به دنبال آن، $\bm{J}^T_{\text{res}}$ به عنوان زیرماتریس باقی‌مانده $6\times6$ تعریف می‌شود و $\mathbf{f}_{c_{\text{res}}}$ نمایانگر نیروهای کابل باقی‌مانده است. ما می‌توانیم $\mathbf{f}_{c_{\text{res}}}$ را در معادله~\eqref{eq:jacobian-distribution} به صورت زیر بازنویسی کنیم:

\begin{equation} \label{eq:force_base_cable_one}
	\mathbf{f}_{c_{\text{res}}} = (\bm{J}_{\text{res}}^T)^{-1} \left( \mathbf{w}_{ee} - \bm{J}_{\text{ref}}^T \cdot \mathbf{f}_{c_{\text{ref}}} \right)
\end{equation}

با این فرمول‌بندی، تعداد متغیرها از 8 به 2 کاهش می‌یابد، در حالی که معادلات حالت ایستا به‌طور ضمنی در معادله \eqref{eq:static} نهفته می‌شود و نیاز افزودن یک قید جداگانه را حذف می کند~\cite{borgstrom2009nims}.




\section{گراف عامل کالیبراسیون و مکان‌یابی همزمان سینماتیک-ایستا} \label{sec:calib_factor}

در این بخش، با استفاده از روابط استخراج شده در قسمت قبل، و همچنین فرمول‌بندی سینماتیکی تعریف شده در فصل پیشین، رویکردی با یک فرمول‌بندی یکپارچه ایجاد می‌شود. رویکرد ما یک گراف عامل با گره‌های متغیر
$\bm{X}(k) \in SE(3)$، $l^0_i$، $\bm{p}_{a_i} \in \mathbb{R}^3$، $\Delta\mathcal{\bm{R}}(k) \in SO(3)$، و $f^{ref}_{h}(k), \ f^{ref}_{v}(k) \in \mathbb{R}$
تعریف می‌کند. این گره‌ها به ترتیب، نمایانگر مکان‌های پنجه ربات، مقادیر اولیه طول کابل‌ها، مکان‌های نقاط پولی، و تغییرات در جهت‌گیری ربات، و همچنین نیروی کابل مرجع در محورهای افقی و عمودی در حالت‌های ایستا ربات هستند. در ساختار فروتحریک ما، همه ترکیب‌های مکانی و جهت‌گیری قابل اجرا نیستند. در اینجا، $\Delta\mathcal{\bm{R}}(k)$ متغیری است که مقدار اولیه چرخش پنجه ربات را تغییر می‌دهد. علاوه بر این، کابل مرجع به عنوان کابلی که کاربر حسگر نیرو برای مقاصد کالیبراسیون بر روی آن تعبیه شده است، تعیین می‌شود.

\begin{figure} [t]
	\centering
	\includegraphics[width=0.55\textwidth]{img/CALIBRATION_GRAPH.pdf}
	\caption{گراف عامل کالیبراسیون و مکان‌یابی همزمان ربات کابلی با کابل‌های خم‌شده}
	\label{fig:calibration_FG}
\end{figure}

شکل~\ref{fig:calibration_FG} ساختار این گراف عامل در راستای کالیبراسیون خودکار و همچنین مکان‌یابی همزمان برای ساختار تعریف شده را که برای دو نمونه از وضعیت‌های ایستا $0$ و $i$ نمایش داده شده است، نشان می‌دهد. متغیرهای بهینه‌سازی در این گراف با دایره‌های برچسب‌خورده با نام‌های پارامتر‌ها در رنگ‌های مختلف نمایان شده‌اند.
علاوه بر این، عامل‌ها با مربع‌های رنگی به تصویر کشیده شده‌اند، که عامل انکودر و یا همان طول کابل خم‌شده به رنگ زرد، عامل مکان اتصال کابل به پولی به رنگ قرمز، عامل اندازه‌گیری نیرو به رنگ سبز، و عامل پیشین مکان به رنگ آبی است. هر عامل بر اساس معادلات سینماتیک و مدل ریاضی تعریف شده برای کابل توصیف‌شده در بخش
\ref{seq:modeling}
 فرمول‌بندی شده و به شرح زیر تعریف می‌شوند:

\subsection{عامل طول کابل خم‌شده}
این عامل رابطه‌ای بین اندازه‌گیری‌های انکودر و طول واقعی کابل از معادله
\ref{eq:lcat}
 ایجاد می‌کند. این قید اندازه‌گیری برای کابل $i^{ام}$ به صورت زیر فرمول‌بندی می‌شود:
\begin{equation}
	f(z^{enc}_{i}, \bm{\zeta})[k] = l_{C,i}[k] + l^0_i - z^{enc}_{i}[k]
\end{equation}

که در آن $l_{C,i}$ نمایانگر طول واقعی کابل در اثر خم‌شدگی از نیروی وزن آن  به‌طوریکه در معادله~\ref{eq:lcat} تعریف شده است، می‌باشد. همچنین $l^0_i$ نشان‌دهنده مقدار طول اولیه کابل است، و $z^{enc}_{i}$ نمایانگر اندازه‌گیری نسبی انکودر مربوط به کابل $i^{ام}$ است. علاوه بر این، $k$ بیان‌کننده شاخص نمونه داده‌های زمانی است.

\subsection{عامل مکان اتصال کابل به پولی}
این عامل تضمین می‌کند که ارتفاع محاسبه‌شده نقطه اتصال کابل روی پنجه ربات، که از مکان پنجه ربات استنتاج شده، با ارتفاع کابل خم‌شده پیش‌بینی‌شده از پولی مربوطه مطابقت داشته باشد. تابع خطا برای پولی $i^{ام}$ به صورت زیر بیان می‌شود:
\begin{equation}
	f(\bm{\zeta})[k] = (p_{b,i})_z [k] - z_i(L_i)[k]
\end{equation}
که در آن، $(p_{b,i})_z$ به ارتفاع نقطه اتصال کابل $i^{ام}$ در مختصات جهانی اشاره دارد، و $z_i(L_i)$ نمایانگر شکل خم‌شدگی کابل در $L_i$ به‌طوریکه در معادله~\ref{eq:z(cx)} توصیف شده است.

\subsection{عامل اندازه‌گیری نیرو}
این عامل نرم نیروی افقی و عمودی را به‌گونه‌ای محدود می‌کند که نزدیک به اندازه‌گیری نیرو از حسگر تعبیه‌شده روی کابل مرجع، در نزدیکی پنجه ربات باشد. توجه داشته باشید که این محدودیت تنها برای کابل مرجع با یک تابع هزینه به صورت زیر مورد نیاز است:
\begin{equation}
	f(f^{m}, \bm{\zeta})[k] = \| [f^{ref}_{h}[k]~~f^{ref}_{v}[k]]^T \| - f^{m}[k]
\end{equation}
در اینجا، $f^{ref}_{h}$ و $f^{ref}_{v}$ به ترتیب نمایانگر نیروهای مرجع کابل در جهت‌های افقی و عمودی هستند، $\|.\|$ نشان‌دهنده نرم اقلیدسی است، و $f^{m}$ مقدار  نیروی کابل مرجع است که توسط حسگر نیرو اندازه‌گیری شده است.

\subsection{عامل پیشین مکان}
عوامل انکودر و اندازه‌گیری نیرو با گره‌های متغیر مکان $\bm{X}(k)$ مرتبط هستند که نمایانگر وضعیت‌های ایستا ربات در فرآیند کالیبراسیون به‌طوریکه توسط یک سیستم محلی‌سازی مبتنی بر بینایی اندازه‌گیری شده است. 
هر مکان به حالات تعادلی مربوط می‌شود که در آن ربات از طریق چهار کابل خود ثابت است. نمونه‌هایی از وضعیت‌های ایستا در شکل~\ref{fig:calibration_FG} با نشانگرهای $0$ و $i$ برچسب‌گذاری شده‌اند. این عامل پیشین  مکان نیز برای تعریف صفر ربات مورد استفاده قرار می‌گیرد.



\section{نتایج شبیه سازی} \label{sec:results}
این بخش به منظور اعتبارسنجی مدل و روش کالیبراسیون پیشنهادی از طریق شبیه‌سازی اجزای محدود\footnote{\lr{Finite Element (FE)}}
 سیستم ارائه شده است. ابتدا اعتبار فرمول‌بندی‌های سینماتیک-ایستا بررسی می‌شود و سپس نتایج کالیبراسیون برای دو ربات کابلی کوچک مقیاس و بزرگ مقیاس نشان داده می‌شود. برای شبیه‌سازی‌های اجزای محدود از نرم‌افزار RecurDyn~\cite{functionbay} استفاده خواهیم کرد، مدل گراف عامل خود را با استفاده از کتابخانه GTSAM~\cite{dellaert2012factor} پیاده‌سازی می کنیم و همچنین از SymForce~\cite{Martiros-RSS-22} برای استخراج مشتق عامل‌ها و ژاکوبین‌های مربوطه استفاده می‌کنیم.


نرم‌افزار \lr{RecurDyn} یک نرم‌افزار مهندسی به کمک رایانه\footnote{\lr{Computer-Aided Engineering (CAE)}}
است که توسط شرکت \lr{FunctionBay} توسعه داده شده و در شبیه‌سازی دینامیک چندجسمی\footnote{\lr{Multibody Dynamics (MBD)}}
 سیستم‌های مکانیکی که از اجسام سخت یا انعطاف‌پذیر متصل به هم تشکیل شده‌اند، تخصص دارد. این نرم‌افزار به مهندسان اجازه می‌دهد حرکت و تعاملات این سیستم‌ها را شبیه‌سازی و تحلیل کنند تا رفتار آن‌ها در شرایط واقعی را پیش‌بینی کنند. \lr{RecurDyn} با ابزارهای تحلیل اجزا محدود\footnote{\lr{Finite Element Analysis (FEA)}}
  برای تحلیل اجسام انعطاف‌پذیر که تحت تنش ها تغییر شکل می‌دهند، یکپارچه شده و از هم‌زمان‌سازی با سایر ابزارهای \lr{CAE} و سیستم‌های کنترلی پشتیبانی می‌کند. همچنین این نرم‌افزار امکان سفارشی‌سازی از طریق برنامه‌نویسی را فراهم می‌کند. \lr{RecurDyn} به طور گسترده در صنایع مختلفی مانند خودروسازی، رباتیک، هوافضا و ماشین‌آلات صنعتی برای بهینه‌سازی عملکرد و طراحی سیستم‌ها استفاده می‌شود. ما نیز از این نرم‌افزار برای شبیه‌سازی اجزا محدود کابل استفاده می‌کنیم.

\subsection{صحت‌سنجی مدل}
برای صحت‌سنجی دقت مدل سینماتیک-ایستا، همان‌طور که در بخش 
\ref{seq:modeling}
توضیح داده شده است، گراف عامل سینماتیک-ایستا توسعه داده شده را با استفاده از عامل‌های پیشین بر روی متغیرهای استخراج شده از شبیه‌ساز مقید می‌کنیم. دو سناریوی ربات کابلی معلق کوچک و بزرگ مقیاس برای صحت‌سنجی این مدل انجام شده است. هدف از طراحی این دو سناریوی مجزا، بررسی دقت الگوریتم برای طیف وسیعی از پیاده‌سازی‌ها می‌باشد. شکل~\ref{fig:recurdyn_small} و شکل~\ref{fig:recurdyn_large} این سناریوهای ربات‌ها را در محیط شبیه‌ساز RecurDyn نشان می‌دهند. هر دو ربات چهار کابل را به چهار گوشه بالای یک جعبه مستطیلی  متصل کرده‌اند. همچنین ابعاد ربات که از فواصل بین پولی ها که در شکل مشخص شده‌اند به دست آمده، $(12.5, 4.5, 28.5)$ متر برای ربات کوچک و $(240, 220, 50)$ متر برای ربات بزرگ‌تر در نظر گرفته شده است. همان‌طور که در جدول~\ref{tab:Model_verification} ذکر شده است، جرم پنجه برای ربات بزرگ $34 \text{ Kg}$ و برای ربات کوچک $4.4 \text{ Kg}$ تنظیم شده است. چگالی طول کابل‌ها برای ربات کوچک $10.2 \text{ g/m}$ و برای ربات بزرگ‌تر $72.4 \text{ g/m}$ است، که به نسبت جرم پنجه ربات به کابل $4.14$ برای ربات کوچک و $0.62$ برای ربات بزرگ‌تر نتیجه می‌دهد. همان‌طور که در~\cite{pott2013cable} پیشنهاد شده است، این شرایط نشان‌دهنده تأثیر قابل توجه خم‌شدگی برای ربات بزرگ‌تر است.

\begin{figure} [t]
	\centering
	\includegraphics[width=0.45\textwidth]{img/E2_small.pdf}
	\caption{سناریوی ربات کوچک‌مقیاس در محیط شبیه‌ساز RecurDyn}
	\label{fig:recurdyn_small}
\end{figure}

\begin{figure} [b]
	\centering
	\includegraphics[width=0.8\textwidth]{img/E1_large.pdf}
	\caption{سناریوی ربات بزرگ‌مقیاس در محیط شبیه‌ساز RecurDyn}
	\label{fig:recurdyn_large}
\end{figure}

دو ردیف اول جدول~\ref{tab:Model_verification} درصد خطاهای میانگین در طول کابل پیش‌بینی شده (MPE-L) و همچنین این خطا برای نیرو (MPE-F) محاسبه شده در 5 مکان ایستای تصادفی را ارائه می‌دهد. این مقادیر نشان‌دهنده‌ی تطابق نزدیک بین نیروی کابل و مقادیر طول کابل خم‌شده محاسبه شده از مدل ریاضی ارائه شده و مقادیر مربوطه از شبیه‌ساز می‌باشد. به طور خاص، برای ربات بزرگ مقیاس، این خطاهای MPE-L و MPE-F به ترتیب برابر با
 $0.0089\%$ 
 و 
 $0.9154\%$ 
 ارزیابی شده است، که بسیار کوچک‌تر از دامنه‌های طول کابل و نیرو‌های ایجادشده برای هر کابل، مطابق مقادیر گزارش شده در جدول، می‌باشد. همان‌طور که به طور شهودی انتظار می‌رود، این تطابق برای ربات کابلی کوچک‌تر دقیق‌تر است. مقادیر مربوط به طول کابل و نیروهای پیش‌بینی شده در ستون سوم این جدول نشان‌دهنده این موضوع می‌باشند.

\begin{table}
	\centering
	\caption{صحت‌سنجی مدل}
	\label{tab:Model_verification}
	\renewcommand{\arraystretch}{1.45} % Increase row spacing
	\footnotesize
	\begin{tabular}{c|c|c}
		\toprule
		\rowcolor{gray!10}
		\hline
		\text{سناریو} & \text{‌بزرگ‌مقایس} & \text{‌کوچک‌مقیاس} \\
		\midrule
		(\%) MPE-L  & $8.8947\times10^{-3}$ & $7.3595\times10^{-3}$ \\
		\hline
		(\%) MPE-F  & $0.9154$ & $0.9046$ \\
		\hline
		$[l_{\min}, l_{\max}](m)$ & $[134.7, 200.7]$ & $[10.8, 25.9]$ \\
		\hline
		$[f_{\min}, f_{\max}](N)$ & $[489.1, 954.3]$ & $[12.4, 26.2]$ \\
		\hline
		اندازه ربات ($\text{m}^3$) & $240\times220\times50 $ & $12.5\times 4.5 \times 28.5$ \\
		\hline
		جرم پنجه ربات ($\text{Kg}$) & $34.0$ & $4.4$ \\
		\hline
		\multirow{2}{*}{مشخصات کابل} & \multirow{2}{*}{\begin{tabular}[c]{@{}c@{}} $\rho = 1.44~~(\text{g}/\text{cm}^2)$ \\ 
				$\text{شعاع} = 4.0~~(\text{mm})$ \end{tabular}} & \multirow{2}{*}{\begin{tabular}[c]{@{}c@{}} $\rho = 1.44~~(\text{g}/\text{cm}^2)$ \\ $\text{شعاع} = 1.5~~(\text{mm})$  \end{tabular}} \\
		& & \\
		\bottomrule
	\end{tabular}
\end{table}


\subsection{نتایج نهایی کالیبراسیون با گراف عامل توسعه‌ داده‌شده}
در قسمت، روش کالیبراسیون را که در بخش 
\ref{sec:calib_factor}
توضیح داده شده است، پیاده‌سازی می‌کنیم و اهمیت مدل‌سازی خم‌شدگی کابل را از طریق نتایج شبیه‌سازی نشان می‌دهیم. علاوه بر این، به طور خلاصه به مسئله مقداردهی اولیه کالیبراسیون پرداخته و یک راه‌حل احتمالی را همچون فصل قبل، پیشنهاد می‌کنیم.

هدف ما در فرآیند کالیبراسیون سینماتیکی، تعیین مکان‌های نقطه‌های پولی و طول اولیه کابل‌ها با استفاده از اندازه‌گیری‌های مجموعه‌ای از مکان‌های پنجه ربات، اندازه‌گیری‌های طول نسبی کابل و تنها مقادیر کشش کابل مرجع در نقطه اتصال پنجه است. طبق آزمایشات و نتایجی که از طیف وسیعی از داده‌ها استخراج شده است، توزیع نمونه‌ها بایستی به طور جامع فضای‌کاری ربات را پوشش دهد. تحلیل و توسعه مسیر مناسب برای جمع‌آوری کمترین تعداد داده در جهت انجام فرآیند کالیبراسیون موفق، به موضوع تحقیق آینده برای تکمیل این کار محول می‌شود.

ما داده‌های حسگر شبیه‌سازی شده خود را از نرم‌افزار RecurDyn دریافت کردیم و تغییرات نویز گاوسی میانگین صفر را برای وارد کردن نویز پیش‌بینی‌شده حسگر معرفی کردیم. به طور خاص، ما انحراف استاندارد\footnote{\lr{Standard Deviation {STD}}} 
$10mm$ 
برای طول کابل، $5N$ برای حسگر نیرو در ربات بزرگ‌مقیاس و $1N$ برای حسگر ربات کوچک‌مقیاس اعمال کردیم. مکان‌های پنجه ربات با $\Delta \mathbf{T} = \exp(\hat{\boldsymbol{\xi}})$ که در آن $\hat{\boldsymbol{\xi}} \in \mathfrak{se}(3)$ عنصر جبر لی\footnote{\lr{Lie algebra}}
 متناظر با بردار پیچش\footnote{\lr{screw vector}}
  $\boldsymbol{\xi} \in \mathbb{R}^6$
 است، دچار تغییر شده‌اند. این بردارهای پیچش تغییرات، از توزیع گاوسی میانگین صفر $\boldsymbol{\xi}\sim\mathcal{N}(\mathbf{0}, \boldsymbol{\Sigma})$ با ماتریس کوواریانس $\boldsymbol{\Sigma}$ با انحراف $0.005$ متر در درجات انتقالی و $0.01$ رادیان در درجات آزادی چرخشی نمونه‌برداری شده‌اند.
برای مقداردهی اولیه گراف عامل، مکان‌های نقطه‌های پولی از شبیه‌ساز به عنوان داده‌های مرجع را با دامنه $1$ متر برای ربات کوچک و $10$ متر برای  ربات بزرگ دچار تغییر کردیم. علاوه بر این، این تغییرات برای مقادیر اولیه طول کابل‌ها، $10$ متر و $80$ متر برای ربات کوچک و بزرگ، به ترتیب، تنظیم شد. ما تخمین اولیه برای نیروی کابل مرجع در محور عمودی را به عنوان:
\begin{equation}  \label{eq:fv_0}
	f^{{ref}}_{v_0} = \frac{m_eg}{4}
\end{equation}
  در نظر گرفتیم و برای محور افقی نیز به صورت زیر:
\begin{equation}  \label{eq:fh_0}
	f^{ref}_{h_0}=\frac{f^{{ref}}_{v_0}}{\tan(\alpha)} 
\end{equation}
که در آن $\alpha$ به صورت زیر تعریف می‌شود:
\begin{equation}  \label{eq:alpha}
	\alpha = \arccos\left(\frac{\hat{\bm{s}}_{c_{ref}}^T \cdot [\bm{p}_{b_{\text{ref}}} - \bm{p}_{a_{\text{ref}}}]}{\| \bm{p}_{b_{\text{ref}}} - \bm{p}_{a_{\text{ref}}} \|}\right)
\end{equation}
در این تعریف، $\hat{\bm{s}}_{c_{ref}}$ بردار واحدی است که در بخش \ref{seq:modeling} تعریف شده است و از تخمین‌های اولیه مکان پولی استفاده می‌کند. این نقاط باید به پولی‌های مربوط به کابل مرجع متقابل مربوط ‌شوند.

نتایج کالیبراسیون برای مکان‌های پولی و طول اولیه کابل در جدول~\ref{tab:calibration_results_sag} ارائه شده است. این نتایج مربوط به سناریوهای نشان داده شده در شکل~\ref{fig:recurdyn_small} و شکل~\ref{fig:recurdyn_large} است، که در آن مکان‌های اولیه پولی با ستاره‌های قرمز و مکان‌های بهبودیافته پولی‌ با دایره‌های زرد برای هر دو مورد نشان داده شده است.
در این جدول، سه سناریوی مختلف کالیبراسیون برای ربات بزرگ‌مقیاس با تعداد مختلف نقاط نمونه‌برداری شده (یا همان تعداد داده‌ها برای کالیبراسیون) برای فرآیند بهینه‌سازی گزارش شده است. همان‌طور که انتظار می‌رود، دقت کالیبراسیون با افزایش تعداد داده‌های نمونه‌برداری بهبود می‌یابد. بنابراین جدول~\ref{tab:calibration_results_sag} نتایج دقیق‌تری را با 35 نمونه داده برای ربات بزرگ و 8 نمونه برای ربات کوچک ارائه می‌دهد. ما معتقدیم که برای ربات بزرگ‌تر، پارامترهای مربوط به خم‌شدگی کابل تأثیر عمیق‌تری بر دقت کالیبراسیون دارند. این موضوع به نوبه خود تعداد مؤثر پارامترهای مدل را افزایش می‌دهد و نیاز به نقاط نمونه‌برداری بیشتری برای شناسایی مناسب دارد.

\begin{table}
	\centering
	\caption{نتایج میانگین خطای مطلق کالیبراسیون با استفاده از گراف عامل توسعه‌ داده‌شده}
	\label{tab:calibration_results_sag}
	\renewcommand{\arraystretch}{1.3} % Increase row spacing
	\footnotesize
	\begin{tabular}{c|c|c|c|c}
		\toprule
		\rowcolor{gray!10}
		\hline
		\text{میانگین خطا(متر)} & \text{مقیاس‌بزرگ (9 داده)} & \text{مقیاس‌بزرگ (18 داده)} & \text{مقیاس‌بزرگ (35 داده)} & \text{مقیاس‌کوچک (8 داده)} \\
		\midrule
		$\text{پولی(متر)}$ &  $0.387$ &  $0.226$  &   $0.191$  &   $0.149$ \\
		\hline
		$\text{طول اولیه کابل(متر)}$ &  $0.380$  &  $0.217$  &   $0.173$  &   $0.115$ \\
		\bottomrule
	\end{tabular}
\end{table}

حل شدن مسئله بهینه‌سازی کالیبراسیون در کنار زنجیره‌ای از پارامتر‌های مکان‌یابی که به مدل کابل‌، داده‌های حسگریِ پنجه ربات و وضعیت مفصل‌های ربات مقید شده است، حلی دقیق‌تر از مسئله را برای ما فراهم می‌کند. حرکت ربات در فضا می تواند منجر به ایجاد وضعیت‌های ایستا شود که در هر کدام از این وضعیت‌ها، قیدهای مربوط به سینماتیک به زنجیره‌ مکان‌یابی متصل می‌شود. وجود تعدادی محدود از این قیود می تواند نقش مهمی در بهبود مکان ربات پس از مدتی حرکت در فضا را ایجاد کند و از خطاهای جمع‌شونده که از نویز حسگر‌ها القا می‌شود، جلوگیری شود. عملکرد این قید همچون نشانگرهایی که در SLAM مورد استفاده قرار می‌گیرند، در اینجا نیز بسیار کارآمد هستند. از آنجایی که هدف ما در این فصل توسعه پایه و اساس الگوریتم مد نظر و مقیدسازی آن با قیدهای پیچیده‌تر بود، از ایجاد پیچیدگی بیشتر در قسمت مکان‌یابی اجتناب گردید. افزودن قیود مکان‌یابی و توسعه بیشتر الگوریتم در این راستا، کار دشواری نخواهد بود. 

موضوع دیگری که در فرمول‌بندی تعریف شده مورد توجه قرار دارد، صادق بودن معادلات طول خم‌شده کابل در حالت ایستای ربات است. به همین دلیل، داده‌هایی که برای کالیبراسیون مورد استفاده قرار داده‌ شد، داده‌‌های حالت‌های ایستای ربات بودند. البته، حل مسئله در فضایی جامع‌تر، و نه محدود به داد‌های ایستا، نیازمند تغییر فرمول‌بندی به معادلات در فضای اجزای محدود می باشد. این توسعه و تحقیق به کارهای آینده سپرده شده‌است.


\section{بحث و گفت‌وگو} \label{sec:discussion}

\subsection{اهمیت در نظر گرفتن اثر خم‌شدگی کابل}
برای بررسی اهمیت در نظر گرفتن اثر خم‌شدگی کابل، فرآیند کالیبراسیون را برای ربات بزرگ‌تر با استفاده از یک گراف عامل ساده‌سازی شده که در آن مدل کابل بی‌وزن صلب به کار رفته بود و در فصل پیشین معرفی شد، انجام دادیم. به عبارتی دیگر، در این آزمایش، گراف عامل مربوط به خم‌شدگی کابل از گراف توسعه‌یافته در این فصل حذف شد. 

نتایج نشان داد که این ساده‌سازی منجر به افزایش قابل توجه خطای میانگین مطلق از $0.19$ به $2.34$ متر شد. این افزایش چشمگیر در خطا، بیانگر اهمیت حیاتی در نظر گرفتن اثر خم‌شدگی کابل در مدل‌سازی و کالیبراسیون است. در واقع، خم‌شدگی کابل نه تنها به دلیل تاثیر مستقیم بر طول واقعی کابل و نیروهای اعمالی بلکه به دلیل تاثیر غیرمستقیم آن بر دقت نهایی مکان‌یابی نیز بسیار مهم است.

این تفاوت قابل توجه در کیفیت کالیبراسیون نشان می‌دهد که نادیده گرفتن چنین پارامترهای فیزیکی مهم می‌تواند به کاهش قابل توجه دقت مدل منجر شود و بر عملکرد کلی سیستم تأثیر منفی بگذارد. علاوه بر این، در سیستم‌های بزرگ‌مقیاس که تغییرات کوچک در مدل می‌تواند تأثیرات بزرگی داشته باشد، اهمیت این نکته دوچندان می‌شود. این نتایج تأکید می‌کنند که برای دستیابی به دقت بالا در فرآیندهای کالیبراسیون و مکان‌یابی، مدل‌سازی دقیق و جامع از پارامترهای فیزیکی کابل‌ها، از جمله خم‌شدگی، ضروری است. این نکته نه تنها برای ربات‌های کابلی بلکه برای هر سیستم مکانیکی دیگری که به دقت بالا نیاز دارد، قابل تعمیم است.


\subsection{نکات مربوط به روش مقداردهی اولیه}
همان‌طور که در~\cite{khorrambakht2023graph} ذکر شده است، یکی از نگرانی‌های اصلی در حل مسئله بهینه‌سازی کالیبراسیون غیرمقعر و غیرخطی، مقداردهی اولیه صحیح آن است. اگر مقادیر اولیه به اندازه کافی نزدیک به راه‌حل جهانی مسئله نباشند، نتیجه ممکن است به شدت منحرف شود یا بهینه‌ساز حتی ممکن است واگرا شود. چارچوب ارائه شده در~\cite{khorrambakht2023graph} که در فصل پیشین نیز مورد استفاده قرار گرفت، تلاش می‌کند تا این مسئله مقداردهی اولیه را با استفاده از خروجی تقریبی یک الگوریتم بهینه‌سازی جهانی مونت کارلو حل کند. با این حال،~\cite{khorrambakht2023graph} مدل کابل صلب بدون اثرات خم‌شدگی را فرض می‌کند. ما معتقدیم که این الگوریتم به طور مستقیم می‌تواند برای مقداردهی اولیه مسئله کالیبراسیون توسعه یافته ارائه شده در این فصل به کار رود. همان‌طور که قبلاً ذکر شد، دقت کالیبراسیون ربات کابل بزرگ با مدل کابل ساده شده $2.34$ متر بود. این مقدار به طور قابل توجهی کوچکتر از تغییرات ما در طول آزمایشات کالیبراسیون ($10$ متر) است. این نشان می‌دهد که ما می‌توانیم با خیال راحت الگوریتم بهینه‌سازی خود را با خروجی‌های الگوریتم مشابهی که در~\cite{khorrambakht2023graph} ارائه شده است، مقداردهی اولیه کنیم. تأیید این فرضیه برای موارد خاص ارائه شده در این فصل به دلیل محدودیت‌های شبیه‌ساز مورد استفاده برای تولید تصاویر/داده‌های حسگر LiDAR مورد نیاز برای اجرای این الگوریتم امکان‌پذیر نبود. بررسی این ایده با استفاده از شبیه‌سازهای واقع‌گرایانه موضوع تحقیق آینده‌ی ما است.



\section{نتیجه‌گیری}
این فصل با هدف توسعه گراف عامل معرفی شده در فصل قبل و بررسی میزان دشواری افزودن قید به یک مسئله فرمول‌بندی شده در فضای گراف، تدوین گردید. همان‌طور که مشاهده شد، مقید‌سازی این مسئله، حتی با وجود قیدهای بسیار پیچیده، در این فضای گراف پایه به‌طور نسبی ساده است. این فصل توانست شرایطی از کالیبراسیون و مکان‌یابی یک ربات واقعی را شبیه‌سازی کرده و نتایج ارزشمندی را برای اولین بار در فضای ربات‌های کابلی مقیاس بزرگ با استفاده از روش‌های بهینه‌سازی گراف‌پایه ارائه دهد.

نتایج حاصل نشان داد که مدل‌سازی دقیق اثر خم‌شدگی کابل‌ها در ربات‌های کابلی، تأثیر قابل‌توجهی در بهبود دقت کالیبراسیون و مکان‌یابی دارد. با استفاده از گراف عامل پیشنهادی، امکان بررسی و مدل‌سازی دقیق‌تر شرایط واقعی کابل‌های خم‌شده فراهم شده است که در نهایت منجر به کاهش خطاهای ناشی از فرضیات ساده‌سازی‌شده در مدل‌های پیشین شد. این نتایج تأکید می‌کند که توجه به جزئیات دینامیکی و فیزیکی کابل‌ها، اهمیت ویژه‌ای در بهبود دقت فرآیندهای کالیبراسیون و مکان‌یابی ربات‌ها دارد.

فرمول‌بندی توسعه داده‌شده در این فصل، با بهره‌گیری از معادلات سینماتیک-ایستا و استفاده از گراف عامل، توانسته است مکان‌های پولی و طول‌های اولیه کابل‌ها را با دقت بالایی تخمین بزند. نتایج شبیه‌سازی‌ها نشان داد که دقت کالیبراسیون به تعداد داده‌های نمونه‌برداری شده بستگی دارد و با افزایش این داده‌ها، نتایج دقیق‌تری به دست می‌آید. این رویکرد نه تنها عملکرد ربات‌های کابلی را در شرایط مختلف بهبود بخشید، بلکه اهمیت توجه به اثرات پیچیده‌تر دینامیکی مانند خم‌شدگی کابل در فرآیندهای کالیبراسیون را برجسته کرد.

در نهایت، تحقیقات انجام‌شده در این فصل، راه را برای مطالعات بیشتر در زمینه بهبود الگوریتم‌های کالیبراسیون ربات‌ها با در نظر گرفتن اثرات دینامیکی هموار کرده است. این نتایج می‌توانند به‌طور بالقوه به پیشرفت‌های قابل‌توجهی در صنایع مختلف از جمله ربات‌های خودران، حمل‌ونقل و فضانوردی منجر شوند. با این حال، علیرغم ایجاد یک شالوده محکم برای پیش‌برد اهداف تحقیقاتی آینده، برخی از مسائل مانند ادغام داده‌های حسگری اضافی با دقت بالا برای اهداف همجوشی حسگرها و بهبود روش‌های مقداردهی اولیه با استفاده از راه‌حل‌های کارآمد معرفی شده در این حوزه، نیازمند بررسی‌های عمیق‌تر در تحقیقات آینده هستند. بهبود و توسعه این جنبه‌ها می‌تواند نتایج حاصل از این تحقیق را به سطوح جدیدی از دقت و کاربرد برساند و در نهایت، زمینه‌ساز نوآوری‌های آینده در حوزه ربات‌های کابلی شود.



