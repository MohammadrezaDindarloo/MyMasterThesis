% !TeX root=../main.tex
% در این فایل، عنوان پایان‌نامه، مشخصات خود و چکیده پایان‌نامه را به انگلیسی، وارد کنید.

%%%%%%%%%%%%%%%%%%%%%%%%%%%%%%%%%%%%
\latinuniversity{K. N. Toosi University of Technology}
\latincollege{...}
\latinfaculty{Faculty of Electrical Engineering}
\latindepartment{Control Group}
\latinsubject{Electrical Engineering}
%\latinfield{field}
\latintitle{‫‪Development‬‬ ‫‪of‬‬ ‫‪a‬‬ ‫‪Graph-Based‬‬ ‫‪Unified‬‬ ‫‪Optimization‬‬ ‫‪Framework‬‬ ‫‪for‬‬ ‫‪Robot‬‬ ‫‪Calibration‬‬ ‫‪and‬‬ ‫‪State‬‬ ‫‪Estimation‬‬}
\firstlatinsupervisor{Prof. Hamid D. Taghirad}
%\secondlatinsupervisor{Second Supervisor}
\firstlatinadvisor{Prof. Philippe Cardou}
\secondlatinadvisor{Dr. Seyed Ahmad Khalilpour}
\latinname{Mohammadreza}
\latinsurname{Dindarloo}
\latinthesisdate{Winter 2024}
\latinkeywords{Factor Graph, Calibration, Localization, SLAM}
\en-abstract{
In today's world, robot manufacturers are focused on simplifying usage and enhancing their performance accuracy, which includes reducing repetitive processes such as sensor and robot calibration. On the other hand, due to the importance of precise localization, recent research has shifted towards novel methods based on graph theory and the integration of data from various sensors.
In this thesis, we start from this familiar point and review recent methods, developing a graph-based framework that not only performs localization effectively but also handles sensor calibration and even goes a step further by simultaneously calibrating the robot without the need for separate processes. This approach creates a unified and flexible problem formulation where adding any constraints does not affect the previous formulations.
The chosen algorithm for this formulation is factor graph, which excels in managing computational complexity through optimized matrices and graph-based architecture, leading to faster convergence and increased stability of the results. Moreover, factor graphs facilitate the seamless integration of heterogeneous sensor data, resulting in improved accuracy and reduced dependency on a single data source.
To validate the proposed method, we consider a cable-driven parallel suspended robot. We first create a unified problem in the graph space that simultaneously performs visual sensor calibration, localization, and kinematic calibration of the robot without requiring prior processes, thus achieving the concept of easy deployment. Then, to assess the flexibility and evaluate our graph, we incorporate the cable sag equations, which are among the most complex equations in robotics, into the graph and analyze the results. The results demonstrate the high capability of this method in creating a flexible formulation that not only increases the speed of the process but also enhances the accuracy of localization and calibration.
}
